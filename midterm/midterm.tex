
\documentstyle{article}

% \def\brac#1{{\tt <}#1{\tt >}}
\def\brac#1{$<$#1$>$}
\def\Int{{\tt int}}
\def\int{\brac{\Int}}
\def\int{\brac{\Int}}
\def\Shortint{{\tt short~int}}
\def\shortint{\brac{\Shortint}}
\def\Longint{{\tt long~int}}
\def\longint{\brac{\Longint}}
\def\Float{{\tt float}}
\def\float{\brac{\Float}}
\def\Double{{\tt double}}
\def\double{\brac{\Double}}
\def\Char{{\tt char}}
\def\chr{\brac{\Char}}
\def\Void{{\tt void}}
\def\void{\brac{\Void}}

\def\ptr#1{pointer~to #1}
\def\p2#1{\brac{\ptr#1}}
\def\Ano#1#2{array~of {#1}~#2s}
\def\ano#1#2{\brac{\Ano#1#2}}
\def\Ao#1{array~of #1}
\def\ao#1{\brac{\Ao#1}}

\def\breakhere{\mbox{$\otimes$}}
\parskip 8pt


\topmargin -0.5in
\textheight 9in

% \def\baselinestretch{2}

\title{Midterm Exam}
\author{CSE 110}
\date{21 July 1992}

\begin{document}
\setcounter{page}{0}
\begin{center}
{\Large \bf CSE 110 Midterm Exam}\\
\vspace{.2in}
{\large M--J. Dominus}\\
\vspace{.2in}
21 July, 1992
\end{center}

\begin{verbatim}




             Name ___________________________________________________________


             Penn ID Number _________________________________________________


             Tracing:             30 pts

        1.1.                      ____________   10 pts
             
        1.2.                      ____________   10 pts
             
        1.3.                      ____________   10 pts
             
             Writing Code:        50 pts

        2.1.                      ____________   10 pts
             
        2.2.                      ____________   20 pts
             
        2.3.                      ____________   20 pts
             
             Debugging:           20 pts

        3.1.                      ____________    5 pts
             
        3.2.                      ____________    5 pts
             
        3.3.                      ____________    5 pts
             
        3.4.                      ____________    5 pts
             



            Total                 ____________  100 pts





\end{verbatim}

\newpage

Welcome to the exam. 

I will award partial credit for partially correct answers that show
thought or insight.  If you can't satisfy one of the requirements of a
programming problem, ignore that part and work on writing a good program
that satisfies the other requirements.  It's much better to turn in a
program that does a few things well than it is to turn in a program that
does everything badly.

Make sure to make it clear whhich parts of the things you write are the
ones I should grade.  Be sure to label each answer with a number of the
problem it's the answer to.

\section{Tracing}

\subsection{10 Points}

What does this program fragment print?
\begin{flushleft}
\verb% int name[]={5, 23, 119}; % \\*
\verb% int *p, *q;% \\*
\verb% % \\*
\verb% p = name;% \\*
\verb% q = name + 1;% \\*
\verb! printf("%d %d %d\n", *name, *p, *q);! \\*
\verb% % \\*
\verb% *(p++);% \\*
\verb% (*q)++;% \\*
\verb! printf("%d %d\n", *p, q[0]);!
\end{flushleft}

What are the contents of the array {\tt name} when this code is finished?

\subsection{10 points}

What does this program print?
\input hail.tex

\subsection{10 points}

What does this program print?
\input confusticate.tex

\section{Writing Code}

\subsection{10 points}

Suppose we want to compute the value of $3^4$, which is $3\cdot 3\cdot
3\cdot 3$, or 81.  C, unlike some languages, has no operator for this,
so we'll write a function.  Write the function {\tt pow}, which accepts
an integer {\tt n} and a non-negative integer {\tt p}, and which
computes and returns the value of {\tt n} raised to the {\tt p} power,
or $n^p$.  Have {\tt pow} return a \longint\ value, since the return
values are likely to be large.

\subsection{20 points}

Write the function {\tt strrev}, whose argument is a string, and which
reverses its argument in place.  That means that if we do:
\begin{flushleft}
\verb% char word[] = "Foo"; % \\*
\verb! printf("%s\n", word); ! \\*
\verb% strrev(word);% \\*
\verb! printf("%s\n", word); ! \\*
\end{flushleft}

\noindent the output should be {\tt Foo}, followed by {\tt ooF}.

\subsection{20 points}

Write a function {\tt get\_int} which accepts three arguments: {\tt min}
and {\tt max}, which are \int s, and {\tt prompt}, which is a string.
{\tt get\_int} should print the string {\tt prompt} to prompt the user
to enter numeric input, and try to read an \int\ value from the user.
The input must be between {\tt min} and {\tt max}, inclusive.  {\tt
get\_int} should repeatedly prompt and get input until the user enters
an integer between {\tt min} and {\tt max}.  {\tt get\_int} should
handle non-numeric input gracefully.  When the user finally enters a
valid input, {\tt get\_int} should return that value as its return
value.

\section{Debugging}

\subsection{5 points}

Suppose that {\tt getline} is a function which reads input stream up to
a newline character, somehow finds enough space in memory (an \ao\Char)
to store the line there, stores the line in the space, and returns a
pointer to the first character in the space.  Suppose also that {\tt
getline} somehow terminates the program if it reaches {\tt EOF}.

Let's use {\tt getline} to write a program which copies its input, a
line at a time, to the screen:

\begin{flushleft}
\verb% char * getline(void);% \\*
\verb% int main(void) % \\*
\verb% {% \\*
\verb%   char *s;% \\*
\verb% % \\*
\verb%   while(1) {% \\*
\verb%     s = getline(); % \\*
\verb%     printf(s); % \\*
\verb%   }% \\*
\verb% }% \\*
\end{flushleft}

This program, as written, has a very bad bug that will cause undefined
behavior under certain circumstances.  (Typically the program's behavior
under these circumstances will be garbage output or immediate
termination.)

Find the bug, explain the circumstances under which it occurs, and
write correct code to replace the buggy code above.

\subsection{5 points} 
Dafydd Painter, a world-famous nostrilologist, is writing C programs to
help him with his work.  In one of them he has the directive

\begin{flushleft}
\verb% 	#define NUMBER_OF_NOSTRILS 2% \\*
\end{flushleft}

    He explains: ``This constant represents the number of nostrils
possessed by the typical human being.  It appears many places in my
program.  I represented the number of nostrils with a manifest constant
so that if the value ever changes, it'll be easy to change the code.''

    Of course that's silly, because we don't expect the typical number
of nostrils to change.  Nevertheless, there is a good reason for
{\tt\#define}'ing {\tt NUMBER\_OF\_NOSTRILS}.  What is it?

\subsection{5 points}

Farina Granville, noted quinquagintaseptologist, tried to write a
function that would set the value of a variable to 57:

\begin{flushleft}
\verb% void set_to_57(int var)% \\*
\verb% { % \\*
\verb%   var = 57;% \\*
\verb% }% \\*
\end{flushleft}

Of course, it didn't work.  You explained to her that in order for the
function to change the value of a variable, it must know where that
variable is, and so Farina will have to pass the address of the variable
she wants to change into the function {\tt set\_to\_57}.

Farina is obstinate:  ``I don't want to pass in a pointer,'' she says.
``People shouldn't have to know about pointers to use this simple
function.  I'll have the function itself get the address once it's
called.'' Then she writes this:

\begin{flushleft}
\verb% void set_to_57(int var)% \\*
\verb% {% \\*
\verb%   int *p;      /* Address of argument */% \\*
\verb% % \\*
\verb%   p = &var;% \\*
\verb%   *p = 57; % \\*
\verb% } % \\*
\end{flushleft}

Does this work?  If so, how?  If not, why not?



\subsection{5 points}

Kelvin R. R\'eaumur, a noted temperaturologist, wants to write a
function which accepts a Celsius temperature as an argument and which
returns the equivalent Fahrenheit temperature.  The formula for
converting from degrees Celsius to degrees Fahrenheit is:
\[ ^\circ{\rm F} = {^\circ{\rm C}}\cdot{9\over 5}\ + 32 \]
Here is what Dr. R\'eaumur tried:

\begin{flushleft}
\verb% double celsius_to_fahrenheit(double c)% \\*
\verb% {% \\*
\verb%   double f;% \\*
\verb% % \\*
\verb%   f = c * (9/5) + 32;% \\*
\verb%   return f;% \\*
\verb% }% \\*
\end{flushleft}

First, Dr.~R\'eaumur tested the function by passing in 0, the Celsius
freezing point of water, and it correctly returned 32, the Fahrenheit
freezing point of water.  Heartened, he then tried it on 37, which is
normal human body temperature, but the return value was 69, instead of
98.6 as it ought to have been.

What is wrong with this function?  Fix it.


\end{document}
