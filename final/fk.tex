
\documentstyle{article}

% \def\brac#1{{\tt <}#1{\tt >}}
\def\brac#1{$<$#1$>$}
\def\Int{{\tt int}}
\def\int{\brac{\Int}}
\def\int{\brac{\Int}}
\def\Shortint{{\tt short~int}}
\def\shortint{\brac{\Shortint}}
\def\Longint{{\tt long~int}}
\def\longint{\brac{\Longint}}
\def\Float{{\tt float}}
\def\float{\brac{\Float}}
\def\Double{{\tt double}}
\def\double{\brac{\Double}}
\def\Char{{\tt char}}
\def\chr{\brac{\Char}}
\def\Void{{\tt void}}
\def\void{\brac{\Void}}

\def\ptr#1{pointer~to #1}
\def\p2#1{\brac{\ptr#1}}
\def\Ano#1#2{array~of {#1}~#2s}
\def\ano#1#2{\brac{\Ano#1#2}}
\def\Ao#1{array~of #1}
\def\ao#1{\brac{\Ao#1}}

\def\breakhere{\mbox{$\otimes$}}
\parskip 8pt


\topmargin -0.5in
\textheight 9in

% \def\baselinestretch{2}
\begin{document}

\section{25 Points}

\begin{quotation}
\small
Suppose we have a large, complicated program which stores information in
a doubly-linked list whose nodes have the following form:

\begin{flushleft}
\verb% struct node {% \\*
\verb%   int province_code;% \\*
\verb%   int beheadment_count;% \\*
\verb%   int rioting_index;% \\*
\verb%   struct node *next; /* Pointer to next node in list */% \\*
\verb%   struct node *prev; /* Pointer to previous node in list */% \\*
\verb% } ;% 
\end{flushleft}
\label{struct}

The first node in the list has a {\tt NULL} value stored in its {\tt
prev} member; the last node in the list has a {\tt NULL} value stored in
its {\tt next} member.  There are no bogus nodes.

Write a single function, whose argument is a pointer to a node in the
list, and which deletes the node from the list and frees the memory that
was used to hold that node.

Don't worry about communicating back to the caller if you happen to
delete the head or the tail node.
\end{quotation}

Good code for this function appears below:

\begin{flushleft}
\verb% void delete_node(struct node *delete_me)% \\*
\verb% {% \\*
\verb%   if (delete_me->next)% \\*
\verb%     delete_me->next->prev = delete_me->prev;% \\*
\verb% % \\*
\verb%   if (delete_me->prev)% \\*
\verb%     delete_me->prev->next = delete_me->next;% \\*
\verb% % \\*
\verb%   free(delete_me);% \\*
\verb% }% 
\end{flushleft}

If you had written this, you would have gotten a full 25 points.  One
student pointed out that the function is only a few lines of code, and
is hardly worth 25 points, and yes, that's true---I expected it to be
cheap pork.  (Not quite free pork.)

In fact, only two students got the full 25 points.  Most people forgot
to check to see that {\tt delete\_me->next} was not {\tt NULL} before
they set the {\tt prev} member of the structure it pointed to with {\tt
delete\_me->next->prev = ...}.  {\tt delete\_me->next} is {\tt NULL}
when the function is asked to delete the last node in the list;
therefore most people turned in programs that failed when asked to
delete the last node in a list, and in the similar case of being asked
to delete the first node in a list.  I wish I had stressed this more in
class: Dereferencing the {\tt NULL} pointer is an incredibly common
mistake, which yields undefined behavior.  On a good computer, a program
which dereferences a {\tt NULL} pointer fails immediately.  On a crummy
computer like an IBM PC, all that happens is that you clobber some of
the operating system's data structures and cause a mysterious machine
crash ten minutes down the line.  Because I did not push it hard in
class, people who forgot to check to see if the node was at the
beginning or the end of the list lost only 5 points, although in real
life they would have paid a big time penalty as they wondered why their
program crashed the machine all the time.  I am sure that some people
did wonder this while they worked on their calculator programs, but it
is not their fault that I could not grade the assignment before the
exam was due.

Some people included a handler in case the argument was {\tt NULL}:

\begin{flushleft}
\verb% if (delete_me == NULL) % \\*
\verb%   return;% 
\end{flushleft}

\noindent which is a good idea but was not required.

Some people forgot to free the memory that {\tt delete\_me} pointed to;
this cost 5 points.  

Some people made the function's return value an {\tt int} instead of
{\tt void}; this would have been okay if they had actually returned a
useful value to indicate success or failure,\footnote{For example, they
might have passed back the return value of {\tt free}.} but none of them
did do this; this mistake cost 3 points.

\end{document}
