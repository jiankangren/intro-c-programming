
\documentstyle{article}


% \def\brac#1{{\tt <}#1{\tt >}}
\def\brac#1{$<$#1$>$}
\def\Int{{\tt int}}
\def\int{\brac{\Int}}
\def\int{\brac{\Int}}
\def\Shortint{{\tt short~int}}
\def\shortint{\brac{\Shortint}}
\def\Longint{{\tt long~int}}
\def\longint{\brac{\Longint}}
\def\Float{{\tt float}}
\def\float{\brac{\Float}}
\def\Double{{\tt double}}
\def\double{\brac{\Double}}
\def\Char{{\tt char}}
\def\char{\brac{\Char}}
\def\Void{{\tt void}}
\def\void{\brac{\Void}}

\def\p2#1{\brac{pointer~to #1}}

\parskip 8pt

\title{Lecture 2}
\author{CSE 110}
\date{1 July 1992}

\begin{document}

\maketitle

\section{A Warning}

    I think that in order to program effectively, you have to understand
how a computer really works, and you also have to understand what the
compiler is really doing.  Mainstream computer science pedagogues
disagree with me about this.  Nevertheless, because I am an employee of
the university and not a graduate student, I am accountable to
practically nobody and so we can learn this the right way instead of the
usual way.

    However, that means that before we dive into the syntax of the
language, we have to spend some time discussing some of the details of
how a computer works, and we don't get to dive into the syntax of the
language itself for a while.  I think that is all right---C is a
language for talking about what goes on in computers, and if you don't
know what goes on in computers you won't have anything intelligent to
say in C.

    So if these first lectures are not what you expected, please be a
little patient.

\section{The Blackboard}

\subsection{Bytes}

    Computer memory is like a blackboard divided into squares.  The
squares are called {\em bytes}\/. Each byte can hold one piece of
information.  Each square has a name, which is called its {\em
address}\/.  Like street addresses, the names are numbers.

    Bytes vary in size from computer to computer, but on most modern
computers each byte contains eight {\em bits}.  The {\em bit}\/ is the
smallest possible amount of information.  It represents the amount of
information that one can store in a physical object which can be in one
of only two states.  For example, one bit of information suffices to
describe the state of a light switch: the switch is either on or
off.\footnote{Objects that can be in only one state clearly can't be
used to store any information.  It turns out that the amount of
information you can store in an object with $n$ possible states is no
more than $log_2 n$ bits.  These notions were first clarified by Claude
Shannon of Bell Laboratories in the middle 1950's.} Since there are 8
bits in a byte, each byte is in one of $2^8$ states.  $2^8$ is 256, so
we usually imagine that the information in a byte is a number between 0
and 255.

    So each byte has exactly two properties: It has a value, which we
can think of as a number between 0 and 255, and it has an address which
tells the computer where it is on the microchip.  The address is the
only thing that distinguishes two bytes, so two bytes never have the
same address.  Of course two different bytes can have the same value.


\subsection{Bytes are Enough to Represent All Kinds of Data}

    Bytes only hold numbers between 0 and 255, which is too small a
range to be useful.  So when we want to talk about a bigger number, we
use an old trick.  In our number system, we have only ten symbols: 0, 1,
2, 3, 4, 5, 6, 7, 8, and 9.  When we want to write a number bigger than
9, we write two or more symbols together and say that the symbols mean
different things depending on where they are relative to one another.
For example, the number 674 means 6 hundreds, 7 tens, and 4 units.
Because the 6 contributes most to the value of the numeral 674, we say
it is the {\em most significant digit}\/.  Similarly, the 4 is the {\em
least significant digit}\/.  This is called a {\em place value} system
because the value of each symbol depends on the place it is on the
paper.  We say it is a {\em base 10} or {\em decimal}\footnote{{\em
Decimal}\/ is from {\em Decem}\/, the Latin word for `ten'} system
because each place is worth 10 times as much as the one to its right.

    We do the same thing with the bytes on the blackboard, only we use a
base 256 system.  If we want to represent a number bigger than 255, we
use two or more bytes and let some of them be more significant than
others: each byte is worth 256 times as much as the next.  For example,
to represent the number 674, we put 2 in the most significant byte and
162 in the least significant byte.  The 162 in the least significant
byte is worth 162.  The 2 in the most significant byte is worth 256
times as much as a 2 in the least significant byte would be, so it is
worth $256\cdot 2$ or 512.  Thus the total value of this two-byte object
is $162+512 = 674$.

    Clearly if we're going to have multi-byte objects then it's no
longer sufficient to only keep track of where a certain number is in the
computer's memory: we also need to keep track of how many bytes are
being used to store that number.  If someone writes a two-byte number
into memory and then tells us where it is so that we can look at it or
change it, they had better tell us how long it is, or else we might not
look at all of it (which would be like someone writing the number 674 on
the blackboard and us thinking that it was three numbers, a 6, a 7, and
a 4), or else we might look at too much (which would be like someone
writing the numbers 674 and 523 on the blackboard very close together
and us thinking that they were the single number 674523).  Either would
be a big disaster.

    Mostly the compiler keeps track of the lengths of things and handles
them for you.  That's one of the big reasons for having a computer
language and a compiler in the first place.

    Also we want to store other information than just numbers.  Say we
want to store someone's name---how do we do that with just numbers
between 0 and 255?  Of course the answer is that we assign a number to
each letter of the alphabet, and we store the numbers that correspond to
each letter in the name.  It would be a big disaster if we forgot that
this sequence of numbers was actually supposed to represent a sequence
of letters.  Another reason we have computer languages is to keep track
of which numbers on the blackboard really represent letters and which
ones really represent numbers.

\subsection{Variables and Types}

    So suppose we are going to write a program that will need to
remember a number--say it's a program that thinks of a secret number
between 1 and 10000, and then the user tries to guess what that number
is.  We need to find a blank spot on the blackboard big enough to store
the number.  We need at least two bytes to store the number in, because
it might be bigger than 255.  (Two bytes are enough to store numbers
between 0 and 65,535.)  We also want to remember that this is a two-byte
number and not the first half of a four-byte number, and not the first
two letters in someone's name, or anything like that.

    Here's what we say in C to accomplish all that:

\begin{flushleft}
\verb%    int my_number;%
\end{flushleft}

    {\tt int} is short for `integer', which is a mathematical word for a
whole number.  The actual size of an \int\ depends on what computer
and what compiler you are using, but it is never less than two bytes.
When the compiler sees this instruction, it finds enough free space on
the blackboard to hold an \int, reserves it, and makes a note to
itself that from now on, this space is going to hold an \int\ and
that the name {\tt my\_number} refers to the space.\footnote{Actually the
compiler doesn't find the space right away; rather, it writes out
machine instructions into the executable version of your program that
find the space when they are executed.} You don't need to know the
details about how it finds this space or where it is; all you need to
know is that the space is there and that you can call it {\tt
my\_number}.  The semicolon ({\tt;}) is just punctuation; it tells the
compiler where the end of the statement is.

    This whole process is called {\em declaring}\/ a {\em variable}\/.
The {\em name}\/ of the variable is {\tt my\_number} and the {\em type}\/
of the variable is \int.  A variable has two other properties:  Its
value (which depends only on the values of the bytes that make it up and
on its type), and its address, which is the same as the address of the
first byte that makes it up.  Unless you say otherwise, \int\
variables always start with the value 0.

    In some languages you can use variables without declaring them; what
happens then is that the compiler assumes that they have a certain type.
You can't do that in C.  If you name a variable you haven't declared,
the compiler will signal an error.

\subsection{Other Types}

    C has a few other primitive types:

\begin{itemize}

\item \shortint\  and \longint\  are like \int, but might
be shorter or longer.  \shortint\ is always at least two bytes long, and
it's never longer than {\tt int}.  \longint\ is always at least four
bytes, and it's never shorter than {\tt int}.  How long \shortint\ and
\longint\ actually
are depends on your compiler.  \int\ is supposed to be a size that is
convenient for your computer and you should use \int\ rather than
\longint\ or \shortint\ unless there's a good
reason.  You use \shortint\ to save blackboard space and
\longint\ when you need to handle bigger numbers.

\item \char\ is short for {\em character}\/; you use it when you
want to store a character, which could be a letter, a digit, or any
other symbol on the keyboard.  It's always exactly one byte long.

\item \float\ is short for {\em floating-point number}\/.  A
floating point number is one which is represented internally in
scientific notation, with a mantissa and an exponent.\footnote{This
means, for example, that a number like 1234567890 is stored as
$1.234567890\times 10^9$.  1.234567890 is the {\em mantissa}\/ and 9 is
the {\em exponent}\/.  If you don't understand this, don't worry; we
won't need it for a long time.} This means that a \float\ can represent
a fractional number like $3\over 4$ or $\pi$.  A \float\ variable has a
wide range, and can typically hold a number between about $10^{-38}$ and
about $10^{38}$, but it loses precision towards the ends of its range.
{\tt float}s on typical machines are four bytes long.  There's a more
precise version of a \float\ called a \double, which is twice as long
and twice as accurate.  Similarly, there are
\verb+<+{\tt long double}\verb+>+s which are even longer than ordinary
\double s.

\end{itemize}

That's all the simple types there are.  There are other types, but
nearly everything else is built up from these few.

\subsection{A Digression About {\em Pointers}}

    David Phillips said that he doesn't often wax philosophical, and I
should warn you now that I do often wax philosophical.  This is the
philosophical section of these notes.

    The computer is unlike every other machine in one important way: The
computer can operate upon itself.  A drill press can't drill itself (or
shouldn't), and a machine tool factory can't change its own
configuration to make more machine tool factories.  But a computer can
contain a program that modifies itself; a computer can examine its own
internal configuration and completely change that configuration
depending on the configuration.

    The single thing that makes a computer more interesting than a
typewriter or an adding machine is that it can talk about itself.

    Therefore, a computer language must have a way to talk about the
computer itself in order to be useful.  In particular, there must be a
language for discussing the contents and the configuration of the
blackboard.

    In C, for each type $X$, there corresponds another type, called {\em
pointer to } $X$, which represents the {\em address} of a variable of
type X.\footnote{Please note that this implies that there are an
infinite number of different types.} For example, if you want to talk
about where on the blackboard a certain variable of type \int\ is, you
can store that information in a variable of type \p2\Int.  It's called
{\tt pointer to int} because it `points at' an int---it says where a
certain \int\ is.

    One of the biggest reasons for C's pre-eminence in the world of
computer languages is the facility and conciseness of its pointers.  The
syllabus doesn't say anything about pointers, but we'll discuss them at
least a little because they're so important.

\section{Instructions}

    In the bad old days before high-level languages like C, people wrote
in assembly language.  Every part of every program could see all the
bytes at once.\footnote{BASIC and FORTRAN are like this also.} This
might sound like a good idea, but what happened was that people wrote
all sorts of horrible programs.  You might write a program which stored
some payroll information in the memory of the computer, then printed a
message on the screen, and then did something else with the payroll
information.  You would expect that printing a message would not change
the payroll information.  However, it might, depending on what the
person who wrote the print function thought was the best thing to do.
You might find that your program had an error because the print function
had twiddles the payroll information when you were not looking.  Worse
still, if one day someone changed the way the payroll information was
stored internally, they might not rewrite the print function to inform
it of that fact, and then the print function would scramble the payroll
information completely.  People's programs had all kinds of horrible
errors because all the sub-parts of their programs were mucking around
with each others' data in unpredictable ways.

    In C, and most other high-level languages, you can be sure that
something like this will not happen. You can know that the print
function will not try to change the payroll information.  The way that
you know this is that you do not tell the print function where the
payroll information is on the blackboard.  That is, the print function
can't mess with the payroll information because it simply doesn't know
where that information is.

    Languages these days mostly differ in the methods they provide for
letting one part of a program hide its information from another.  How to
divide up the blackboard is a big deal.  One of C's major advantages
over other languages is that it is a particularly good language for
talking about where things are on the blackboard.

    In C, we break up programs into smaller, self-contained parts called
{\em functions}.  Each function has some instructions associated with
it, and it has some variables that it gets to keep all to itself---no
other functions can look at or alter those variables or even know where
they are unless the function that owns them tells the other function.
One fine point of programming is deciding the best way to carve up a
problem into sub-parts so that the functions are as nearly
self-contained as possible.

    Every instruction in a C program is in one function or another.
When you run a C program, the operating system takes care of starting
the first function for you.  This first function can start other
functions, which in turn can start other functions.  When a function
starts another, we say that the first function has {\em invoked}\/ or
{\em called}\/ the second function.  When a function invokes another,
the operating system stops executing instructions from the first
function and starts executing instructions from the second function.
When the second function is finally done, control returns to the first
function and it goes on with what it was doing.  Thus only one function
is ever executing at any time.

    Of course the functions do have to communicate with each other a
little, and they do that in a specific and controlled way.

\subsection{Arguments}

   Suppose I decide that I will have a function which computes the
square of an integer.  (The square of a number is what you get when you
multiply the number by itself.)  We'll call this imaginary function {\tt
square}.  Now whatever other function calls {\tt square} must tell it at
least one thing: The number that {\tt square} is supposed to square.
The calling function can communicate its number to {\tt square} by {\em
passing} the number as an {\em argument}.

    The way we call a function in C is by writing its name, an open
parenthesis ({\tt (}), the arguments, if there are any, and then a close
parenthesis ({\tt )}).  For example, here's how we might call {\tt
square}:

\begin{flushleft}
\verb%    square(57);%
\end{flushleft}

When the program that is executing reaches this statement, it pauses.n
It doesn't go on to the next statement until it has computed the square
of 57.  To do that, it saves somewhere some information about what it
was doing and transfers control to the {\tt square} function.  When {\tt
square} is finished, the program goes back to where it was before and
continues in the function that invoked {\tt square}.  The details of how
the saving beforehand and getting back afterwards work are the
compiler's problem.

\subsection{A Real Function}

Here's how we might write {\tt square}:

\begin{flushleft}
    \verb%    int square(int n)%
\\* \verb%    {%
\\* \verb%      return n*n;%
\\* \verb%    }%
\end{flushleft}

This is our first real C code, and so we'll discuss it at length.

The first line is most complicated.  The word {\tt square} tells the
compiler that the name of the function that is coming up is {\tt
square}, and that it should remember that.  The {\tt (int n)} tells the
compiler what kind of arguments {\tt square} will accept.  It means that
functions that call {\tt square} are supposed to pass in an \int, and
that inside of {\tt square} the name for that \int\ will be {\tt n}.
This name {\tt n} is called a {\em formal parameter}\/ of the function
{\tt square}, but it behaves just like any other variable of type \int,
and the sequence {\tt (int n)} counts as a declaration of {\tt n}.  The
compiler is responsible for finding an \int's worth of space on the
blackboard in which to store the argument, and is responsible for making
sure that if you ask for {\tt n}, you get the value stored in that
space.

The open curly brace ({\tt\{}) and close curly brace ({\tt\}}) {\em
delimit}\/ the function---they say where the function begins and ends.
If there's more than one function in the program, the braces tell the
compiler where one function leaves off and another begins.

In between the braces is the code for the function itself.  The star
{\tt *} means multiplication.  {\tt return} means to compute the value
of the mathematical expression that follows, return control to the
calling function, and pass it back the value of the mathematical
expression.  So this line computes $(n\times n)$ and immediately returns
the result to the calling function, which was waiting around for that.
When a function executes a {\tt return} statement, that means that it is
done.

I never explained the word {\tt int} on the first line: That says that
the value that {\tt square} is going to return is going to be an \int.
The returned value will represent an integer of a certain size, and not
something else like a fraction or a letter of the alphabet.

\subsection{How the Program Starts}

The operating system arranges that if your program has a function named
{\tt main}, that function will be called first.  If your program has no
{\tt main}, the linker will complain and you won't be able to run your
program.\footnote{We'll find out why it is the linker sometime later.}

\subsection{A Real Program}

I fibbed a little about the right way to call the function {\tt square}.
Here I'll present a real program for computing the square of the number
57 and printing it out.

\input square.tex

There are two functions here:  {\tt main} and {\tt square}.  Notice that
{\tt main} is a function which returns a value of type \int\ and which
takes a value of type \void. \void\ is a special type which means ``no
value at all''.  So when the operating system first invokes {\tt main},
it shouldn't give it any arguments.  

The first line in {\tt main} declares a variable, {\tt the\_square},
in which we will store the result of squaring the number 57.  The
compiler finds space for an int and arranges that if we name {\tt
the\_square} again that the value in this space is used.

The second line is something we haven't seen before:  An {\em
assignment} statement.  We'll discuss it at length tomorrow, but what it
says to do is to compute the value of the mathematical expression on the
right of the {\tt =} sign, and that value into the part of the
blackboard represented by the object on the left of the {\tt =} sign.
The value of a function is whatever is returned by that function with a
{\tt return} statement.

When control reaches this point in {\tt main}, {\tt main} stops and
waits, and control passes to {\tt square}.  The compiler has arranged
that the number 57 gets copied into the part of the blackboard referred
to by {\tt n}, so that when you ask for the value of {\tt n}, you get
the number 57.  {\tt square} fetches the value of {\tt n}, multiplies it
by the value of {\tt n}, and returns that result.  The compiler arranges
that the result (it happens to be 3249) gets stored into the variable
{\tt the\_square}.  Then control returns to {\tt main}.

The next thing that happens is that control passes to the next line in
{\tt main}.  We haven't seen the {\tt printf} function yet, but the call
we've made to it here makes it print out

\verb% The square of 57 is 3249. %

We don't have to supply code for {\tt printf}; it's already written and
stored in a file called a {\em library}\/. It's the linker's job to see
if we used any functions, such as {\tt printf}, for which we didn't
supply any code, and, if so, to search for them in the libraries and to
include the code for them if necessary.  We'll discuss this more later.

Notice that we passed two arguments to {\tt printf}, separated by a
comma ({\tt ,}):  A message to print, and the value of {\tt the\_square}
to fill into the message.  We'll talk about the details of what is going
on here soon.

Finally, {\tt main} itself returns a value, 0, to the operating system.
Conventionally, programs return 0 to the operating system if they were
successful, and some other value if something went wrong. 

\subsection{A Note About Local Variables}

The variables {\tt n} and {\tt the\_square} are called {\em local}\/
variables.  That means they are usable only in the functions in which
they're declared. If you tried to use the variable {\tt n} in the
function {\tt main}, or if you tried to use the variable {\tt
the\_square} in the function {\tt square}, the compiler would complain
and would refuse to compile your program.  That way you can be sure that
the functions aer communicating only in the way you expect them to, by
passing arguments and returning return values, and that they are not
mucking around with each others' private data.

\subsection{Exercise}

Your homework for tomorrow is to type in this program, compile it, and
run it.  Change the {\tt 57} to a different value and see what happens.
In particular, try changing it to {\tt 20000}.  If your program won't
compile, save it and I'll take a look at it and see if I can see what's
wrong.

{\em White space}\/ characters are blanks (like the ones you get from
pressing the space bar), tabs (from pressing the tab key) and newlines
(from pressing the enter key).  The compiler ignores extra white space,
so you can insert blanks, tabs, and newlines anywhere you want in this
program, as long as there is some space there already.  You can use
white space to make the program easier to read.  But beware: Every other
character has some meaning, so don't include any extra nonspace
characters and don't leave any characters out.  Be especially careful to
include the semicolons ({\tt ;}) that are at the ends of most lines.

\end{document}
