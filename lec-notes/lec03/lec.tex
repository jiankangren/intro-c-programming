%
% CSE 110 Lecture Notes
%
% Entire contents are copyright 1992 by Mark-Jason Dominus.
% All rights reserved.  Unauthorized reproduction prohobited.
%

\documentstyle[fancyheadings]{article}

% \def\brac#1{{\tt <}#1{\tt >}}
\def\brac#1{$<$#1$>$}
\def\Int{{\tt int}}
\def\int{\brac{\Int}}
\def\int{\brac{\Int}}
\def\Shortint{{\tt short~int}}
\def\shortint{\brac{\Shortint}}
\def\Longint{{\tt long~int}}
\def\longint{\brac{\Longint}}
\def\Float{{\tt float}}
\def\float{\brac{\Float}}
\def\Double{{\tt double}}
\def\double{\brac{\Double}}
\def\Char{{\tt char}}
\def\char{\brac{\Char}}
\def\Void{{\tt void}}
\def\void{\brac{\Void}}

\def\p2#1{\brac{pointer~to #1}}

\parskip 8pt

\title{Lecture 3}
\author{CSE 110}
\date{2 July 1992}

\pagestyle{fancy}
\lhead{CSE 110 Lecture Notes}
\chead{Mark--Jason Dominus}
\rhead{\thepage}
\lfoot{Copyright \copyright 1992 Mark-Jason Dominus}
\cfoot{}
\rfoot{All rights reserved.}



\begin{document}

\maketitle


\section{Functions}

    In the bad old days before high-level languages like C, people wrote
in assembly language.  Every part of every program could see all the
bytes at once.\footnote{BASIC and FORTRAN are like this also.} This
might sound like a good idea, but what happened was that people wrote
all sorts of horrible programs.  You might write a program which stored
some payroll information in the memory of the computer, then printed a
message on the screen, and then did something else with the payroll
information.  You would expect that printing a message would not change
the payroll information.  However, it might, depending on what the
person who wrote the print function thought was the best thing to do.
You might find that your program had an error because the print function
had twiddled the payroll information when you were not looking.  Worse
still, if one day someone changed the way the payroll information was
stored internally, they might not rewrite the print function to inform
it of that fact, and then the print function would scramble the payroll
information completely.  People's programs had all kinds of horrible
errors because all the sub-parts of their programs were mucking around
with each others' data in unpredictable ways.

    In C, and most other high-level languages, you can be sure that
something like this will not happen. You can know that the print
function will not try to change the payroll information.  The way that
you know this is that you do not tell the print function where the
payroll information is on the blackboard.  That is, the print function
can't mess with the payroll information because it simply doesn't know
where that information is.

    Languages these days mostly differ in the methods they provide for
letting one part of a program hide its information from another.  How to
divide up the blackboard is a big deal.  One of C's major advantages
over other languages is that it is a particularly good language for
talking about where things are on the blackboard.

    In C, we break up programs into smaller, self-contained parts called
{\em functions}.  Each function has some instructions associated with
it, and it has some variables that it gets to keep all to itself---no
other functions can look at or alter those variables or even know where
they are unless the function that owns them tells the other function.
One fine point of programming is deciding the best way to carve up a
problem into sub-parts so that the functions are as nearly
self-contained as possible.

    Every instruction in a C program is in one function or another.
When you run a C program, the operating system takes care of starting
the first function for you.  This first function can start other
functions, which in turn can start other functions.  When a function
starts another, we say that the first function has {\em invoked}\/ or
{\em called}\/ the second function.  When a function invokes another,
the operating system stops executing instructions from the first
function and starts executing instructions from the second function.
When the second function is finally done, control returns to the first
function and it goes on with what it was doing.  Thus only one function
is ever executing at any time.

    Of course the functions do have to communicate with each other a
little, and they do that in a specific and controlled way.  The way we
do this in C is by {\em passing arguments in}\/ to a function when it
starts, and by {\em receiving a return value} from a function when it is
finished.  Starting up a function is called {\em calling}\/ the
function or {\em invoking}\/ the function.

    Only one function is ever active at any time.  When a function calls
another function, it pauses, and waits for the called function to
finish, before it continues with what it was doing.

    When a function calls another function, it decides on the {\em
arguments}, which are some data that it will pass to the function it is
calling.  The called function gets to see these arguments, and the
things it does might depend on them.  When the called function is
finished, it gets to return some value to the calling function.  So the
arguments are the way a function communicates to a function that it
calls, and the return value is the way a called function communicates
information back to its caller.

\subsection{Arguments}

   Suppose I decide that I will have a function which computes the
square of an integer.  (The square of a number is what you get when you
multiply the number by itself.)  We'll call this function {\tt square}.
Now whatever other function calls {\tt square} must tell it at least one
thing: The number that {\tt square} is supposed to square.  The calling
function will therefore communicate its number to {\tt square} by
passing in that number as an argument.

    The way we call a function in C is by writing its name, an open
parenthesis ({\tt (}), the arguments, if there are any, and then a close
parenthesis ({\tt )}).  For example, here's how we might call {\tt
square}:

\begin{flushleft}
\verb%    square(57);%
\end{flushleft}

When the function that is executing reaches this instruction, it pauses.
It doesn't go on to the next instruction until it has computed the square
of 57.  To do that, it saves somewhere some information about what it
was doing and transfers control to the {\tt square} function.  When {\tt
square} is finished, control returns to the calling function, which
picks up where it left off.  The details of how the function remembers
where it left off, and how it gets back to what it was doing after the
called function is finished, are the compiler's problem.

\subsection{What a Function Looks Like}

A function has two parts.  The first part, called the {\em header}, says
the name of the function, how many arguments it has and what they are
called and what their types are, and what type of result it returns to
its callers.  The second part is the {\em body} for the function.  It is
enclosed between curly braces ({\verb+{+} and {\verb+}+}) and it contains the
instructions to the compiler about what to do when the function is
called.

Here's how we might write {\tt square}:

\begin{flushleft}
    \verb%    int square(int n)%
\\* \verb%    {%
\\* \verb%      return n*n;%
\\* \verb%    }%
\end{flushleft}

\noindent This is our first real C code, and so we'll discuss it at length.

The first line is the header.  A header has three parts: First, the name
of the type of the return value that the function returns to its callers,
in this case {\tt int}.  Then comes the name of the function, in this
case {\tt square}.  Then, in parentheses ({\tt (} and {\tt )}) is a list
of {\em parameter declarations}, which describe what types of values a
calling function is allowed to pass in, and what to names to give to
these values.  In this case there's only one argument, and {\tt int n}
means that that argument will have type \int\ and that within the body
of the {\tt square} function, the name {\tt n} will refer to that
argument.

The open curly brace ({\tt\{}) and close curly brace ({\tt\}}) {\em
delimit}\/ the function's body---they say where the body begins and
ends.  When there's more than one function in the program, the braces help
the compiler understand where one function leaves off and another
begins.

In between the braces is the code for the function itself.  The star
{\tt *} means multiplication.  {\tt return} means to compute the value
of the mathematical expression that follows, and return control to the
calling function, passing it back the value of the mathematical
expression.  So this line computes $(n\times n)$ and immediately returns
the result to the calling function, which was waiting around for that.
When a function executes a {\tt return} instruction, that means that it
is done.

Normally, the instructions in a function are executed in order, from top
to bottom.  After an instruction is executed, control passes to the
instruction on the next line.  The {\tt return} instruction is an
exception to this.  When a function executes a {\tt return} instruction,
it does {\em not}\/ go on to the next instruction.  Instead, control
returns to the function that called the one that executed the {\tt
return} instruction.  The next instruction that is executed is in the
{\em calling} function, which picks up right where it left off.

\subsection{How the Program Starts}

The operating system arranges that if your program has a function named
{\tt main}, that function will be called first.  If your program has no
{\tt main}, the linker will complain and you won't be able to run your
program.\footnote{We'll find out why it is the linker sometime later.}

When {\tt main} executes a {\tt return} instruction, control is returned
to the operating system, and your program is over.  The value you return
with the {\tt return } statement from {\tt main} gets passed back to the
operating system as a status code about whether the program succeeded or
failed. Conventionally, {\tt main} returns zero for success and nonzero
for failure.

\section{Statements}

    This is just terminology: Each instruction in a C program is called
a {\em statement}\/.  Statements are easy to recognize: Simple
statements always end with semicolons.

    You can group statements together into {\em blocks}, which are also
called {\em compound statements}.  A compound statement is just a bunch
of other statements (which might be simple or compound), with an open
brace at the beginning and a close brace at the end.  To execute a
compound statement, the computer just executes the statements that make
it up, in order.

The body of a function is always a single compound statement.

\section{The Assignment Statement}

The {\em assignment statement} performs a fundamental operation:  It
copies information from one part of the blackboard to another.

Its syntax is:

\begin{quotation}
	{\it lvalue} {\tt =} {\it expression} {\tt ;}
\end{quotation}

Like all simple statements, it ends with a semicolon to tell the
compiler where its end is.

It has three parts: There's an {\em object} on the left, an equals sign
in the middle, and an {\em expression} on the right.  {\em expression}
means the same as it does in mathematics.  An {\em lvalue} is an
expression that refers to part of the blackboard---for example, the name
of a variable is an lvalue; it refers to the part of the blackboard
where the variable is stored.\footnote{`lvalue' is pronounced
`ell-value'.}

When the computer executes an assignment statement, it first {\em
evaluates} the expression on the right; that means that it reads it and
performs whatever calculations are necessary to determine its value.
For example:
\begin{itemize} 

\item If the expression is just a variable name, then the value
of the expression is just the value stored in the variable.  

\item If the
expression is a function call like {\tt square(57)}, the computer
pauses, calls the function, and the value of the expression is whatever
the called function returned with its {\tt return} statement.  

\item If the expression is just a numeral, like 57, then the value of the
expression is just the value of that numeral.

\end{itemize}

After the computer has calculated the value of the expression, it stores
that value into the part of its memory referred to by the lvalue.

\subsection{Examples of Assignment Statements}

We'll suppose that someone has already declared \int\ variables called
{\tt x} and {\tt y}.

\verb%		x = 10 ; %

The computer evaluates the expression on the right of the equals sign;
the expression is just {\tt 10}, and so the value of the expression is
the integer 10.  Then the computer stores the integer 10 into the
variable {\tt x}.

\verb%		y = x ; %

The computer evaluates the expression on the right of the equals sign;
the expression is just {\tt x}, and so the value of the expression is
the the value of the variable {\tt x}, which is now 10.  Then the
computer stores the integer 10 into the variable {\tt y}.

\verb%		x = x + 1 ; %

The computer evaluates the expression on the right of the equals sign.
It evaluates the {\tt x}, whose value is 10, and it evaluates the {\tt
1}, whose value is 1, and then it adds those two numbers together,
because {\tt +} means addition.  The result, 11, is the value of the
expression on the right of the equals sign.  Then the computer stores
the integer 11 into the variable {\tt x}.

The following is {\em not}\/ a legal assignment statement; the compiler
will refuse to compile it:

\verb%		4 = x ; %

This fails because {\tt 4} is not an lvalue.  That is, it does not refer
to a part of the computer's memory, that way a variable name does.  It
is just a number.  In short, you can't do this because {\tt 4} is not
the name of a place where you can store a value.

\subsection{Value Contexts and Object Contexts}

Notice that in the statement

\verb%		y = x ; %

there is an asymmetry:  {\tt x} is evaluated, but {\tt y} is not.  In
fact, {\tt y}'s value is completely irrelevant---the compiler evaluates
{\tt x} because it is going to store that value in the part of memory
referred to by {\tt y}, but it never evaluates {\tt y} at all.

{\tt x} and {\tt y} in this statement are canonical representatives of
the two contexts that an expression can be in in C.  One context is
called the {\em value context}\/, and it means that the expression will
be evaluated to produce a value.  {\tt x} here is in a value context.
The other context is called the {\em object context}\/.  It means that
the expression will not be evaluated; instead, it is being used to refer
to part of the computer's memory.  {\tt y} here is in an object context.

{\em lvalue}\/ and {\em object context} are really two sides of the same
coin.\footnote{In fact, many people say {\em lvalue context}\/ instead of
{\em object context}\/.}  Only an lvalue may appear in an object
context; conversely, to decide if something is an lvalue, just see if it
makes sense in an object context, such as on the left-hand side of an
assignment statement.

For example:  Is {\tt x + y} an lvalue?
No, because

\verb%		x + y = 4 ; %

doesn't make any sense---it says to store the value 4 into the part of
the computer's memory represented by {\tt x + y}, and that doesn't mean
anything.  

\subsection{A Real Program}

I fibbed a little about the right way to call the function {\tt square}.
Here I'll present a real program for computing the square of the number
57 and printing it out.

\input square.tex

There are two functions here:  {\tt main} and {\tt square}.  Notice that
{\tt main} is a function which returns a value of type \int\ and which
takes a value of type \void. \void\ is a special type which means ``no
value at all''.  So when the operating system first invokes {\tt main},
it shouldn't give it any arguments.  When {\tt main} finishes, it'll
return an \int\ to the operating system to report whether it succeeded
of failed. 

The first line in {\tt main} declares a variable, {\tt the\_square},
in which we will store the result of squaring the number 57.  The
compiler finds space for an int and arranges that if we name {\tt
the\_square} again that the value in this space is used.

The second line is something we haven't seen before: An {\em assignment}
instruction.  We'll discuss it at length tomorrow, but what it says to
do is to compute the value of the mathematical expression on the right
of the {\tt =} sign, and that value into the part of the blackboard
represented by the object on the left of the {\tt =} sign.  The value of
a function is whatever is returned by that function with a {\tt return}
instruction.

When control reaches this point in {\tt main}, {\tt main} stops and
waits, and control passes to {\tt square}.  The compiler has arranged
that the number 57 gets copied into the part of the blackboard referred
to by {\tt n}, so that when you ask for the value of {\tt n}, you get
the number 57.  {\tt square} fetches the value of {\tt n}, multiplies it
by the value of {\tt n}, and returns that result.  The compiler arranges
that the result (it happens to be 3249) gets stored into the variable
{\tt the\_square}.  Then control returns to {\tt main}.

The next thing that happens is that control passes to the next line in
{\tt main}.  We haven't seen the {\tt printf} function yet, but the call
we've made to it here makes it print out

\verb% The square of 57 is 3249. %

We don't have to supply code for {\tt printf}; it's already written and
stored in a file called a {\em library}\/. It's the linker's job to see
if we used any functions, such as {\tt printf}, for which we didn't
supply any code, and, if so, to search for them in the libraries and to
include the code for them if necessary.  We'll discuss this more later.

Notice that we passed two arguments to {\tt printf}, separated by a
comma ({\tt ,}):  A message to print, and the value of {\tt the\_square}
to fill into the message.  We'll talk about the details of what is going
on here soon.

Finally, {\tt main} itself returns a value, 0, to the operating system,
to indicate that it completed successfully.

\subsection{A Note About Local Variables}

The variables {\tt n} and {\tt the\_square} are called {\em local}\/
variables.  That means they are usable only in the functions in which
they're declared. If you tried to use the variable {\tt n} in the
function {\tt main}, or if you tried to use the variable {\tt
the\_square} in the function {\tt square}, the compiler would complain
and would refuse to compile your program.  That way you can be sure that
the functions are communicating only in the way you expect them to, by
passing arguments and returning return values, and that they are not
mucking around with each others' private data.

\end{document}
