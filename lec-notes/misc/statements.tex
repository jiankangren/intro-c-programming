

\documentstyle{article}

% \def\baselinestretch{2}

% \def\brac#1{{\tt <}#1{\tt >}}
\def\brac#1{$<$#1$>$}
\def\Int{{\tt int}}
\def\int{\brac{\Int}}
\def\int{\brac{\Int}}
\def\Shortint{{\tt short~int}}
\def\shortint{\brac{\Shortint}}
\def\Longint{{\tt long~int}}
\def\longint{\brac{\Longint}}
\def\Float{{\tt float}}
\def\float{\brac{\Float}}
\def\Double{{\tt double}}
\def\double{\brac{\Double}}
\def\Char{{\tt char}}
\def\char{\brac{\Char}}
\def\Void{{\tt void}}
\def\void{\brac{\Void}}

\def\p2#1{\brac{pointer~to #1}}

\parskip 8pt

\title{About Statements}
\author{CSE 110}
\date{13 July 1992}

\begin{document}
\maketitle

Some people have asked just what a statement is, and how you can
identify one, and how to tell if a certain thing should be followed by a
semicolon or not.  These notes should sum all that up.

A statement is one of the following forms:

\begin{itemize}

\item The {\em null statement}\/, which is written

\begin{flushleft}
{\tt ;}
\end{flushleft}

To execute the null statement, the computer simply does nothing.

\item An {\em expression statement}\/, which has this form:

\begin{flushleft}
{\em expression}\/{\tt ;}
\end{flushleft}

To execute an expression statement, the computer evaluates the
expression, takes the resulting value and throws it away.  Then it
resolves any side effects, such as increments, decrements, and
assignments, that may have resulted from the evaluation.


\item A {\em compound statement}\/, which has this form:

\begin{flushleft}
{\tt \{} \\*
{\hspace{8pt}\em declarations} \\*
{\hspace{8pt}}\\*
{\hspace{8pt}\em statement 1} \\*
{\hspace{8pt}\em statement 2} \\*
{\hspace{8pt} $\cdots$} \\*
{\hspace{8pt}\em statement n} \\*
{\tt \}}
\end{flushleft}

To execute a compound statement, the computer executes {\em
statement~1}\/.  Then it executes {\em statement~2}\/, and soforth,
until finally it executes {\em statement~n}\/.  The sub-statements are
always executed in the order in which they appear in the file.

Variable declarations may appear at the beginning of any compound
statement.  The variables so declared exist only while control is inside
the compound statement.

\item A {\em {\tt return} statement}\/, which has this form:

\begin{flushleft}
{\tt return {\em expression} ; }
\end{flushleft}

To execute a {\tt return} statement, the computer evaluates the {\em
expression}\/.   It then imediately terminates all processing in the
current function, and control passes back to the place from which the
function was called; the value of the {\em expression} is the function's
return value.  A {\tt return} from {\tt main} returns control to the
operating system and so terminates the entire program.

If the function containing the {\tt return} statement has return type
{\tt void}, the {\em expression}\/ should simply be omitted, and no
value will be returned.

\item An {\em {\tt if} statement}\/, which has this form:

\begin{flushleft}
{\tt if ( {\em condition} )} \\*
{\hspace{8pt}\em statement} 
\end{flushleft}

To execute an {\tt if} statement, the computer evaluates the {\em
condition}\/, which is just an expression.  If the value of the {\em
condition} is {\bf true} (nonzero), the computer then executes the {\em
statement}.  If the value of the {\em condition} is {\bf false} (zero),
the computer does not execute the {\em statement}\/.

\item An {\em {\tt if{\rm--}else} statement}\/, which has this form:

\begin{flushleft}
{\tt if ( {\em condition} )} \\*
{\hspace{8pt}\em statement 1} \\*
{\tt else } \\*
{\hspace{8pt}\em statement 2} 
\end{flushleft}

To execute an {\tt if{\rm--}else} statement, the computer evaluates the
{\em condition}\/, which is just an expression.  If the value of the
{\em condition} is {\bf true} (nonzero), the computer then executes {\em
statement~1}.  If the value of the {\em condition} is {\bf false}
(zero), the computer executes {\em statement~2}\/.  In no case does it
execute both {\em statement~1} and {\em statement~2}.

\item A {\em {\tt while} statement}\/, which has this form:

\begin{flushleft}
{\tt while ( {\em condition} )} \\*
{\hspace{8pt}\em statement} \\*
\end{flushleft}

To execute a {\tt while} statement, the computer evaluates the {\em
condition}\/, which is just an expression.  If the value of the {\em
condition} is {\bf true} (nonzero), the computer then executes the {\em
statement}, loops back, and executes the entire {\tt while} statement
again. If the value of the {\em condition}\/ is {\bf false} (zero), the
computer does not execute the {\em statement}\/ and does not loop back.

\item A {\em {\tt do{\rm--}while} statement}\/, which has this form:

\begin{flushleft}
{\tt do } \\*
{\hspace{8pt}\em statement} \\*
{\tt while ( {\em condition} ) ;}
\end{flushleft}

To execute a {\tt do{\rm-}while} statement, the computer executes the
{\em statement}\/. Then it evaluates the {\em condition}\/, which is
just an expression.  If the value of the {\em condition} is {\bf true}
(nonzero), the computer loops back, and executes the entire {\tt
do{\rm--}while} statement again. If the value of the {\em condition}\/
is {\bf false} (zero), the computer does not loop back.

\item A {\em {\tt for} statement}\/, which has this form:

\begin{flushleft}
{\tt for ( {\em initialization}\/ ; {\em condition} ; {\em update} ) } \\*
{\hspace{8pt}\em statement} 
\end{flushleft}

To execute a {\tt for} statement, the computer executes the following
code:

\begin{flushleft}
{ \em initialization \tt ;} \\*
{\tt while ( {\em condition} ) \{ } \\*
{\hspace{8pt}\em statement} \\*
{\hspace{8pt}\em update \tt ;} \\*
{\tt\}} \\*
\end{flushleft}

Any or all of the {\em initializiation}\/, {\em condition}\/, or {\em
update} expressions may be omitted.  If the {\em initialization}\/ or
{\em update} is omitted, the corresponding statement in the
corresponding {\tt while} loop simply becomes null.  If the {\em
condition} is omitted, it is taken to be permanently {\bf true}.

\item A {\em {\tt break} statement}\/, which has this form:

\begin{flushleft}
{\tt break ;}
\end{flushleft}

To execute a {\tt break} statement, the computer jumps out of the
smallest {\tt for}, {\tt switch}, {\tt while}, or {\tt do{\rm--}while}
statement that the {\tt break} was inside.  Control continues with the
statement immediately following this {\tt for}, {\tt switch}, {\tt
while}, or {\tt do{\rm--}while} statement.

If the {\tt break} statement is not inside any {\tt for}, {\tt switch},
{\tt while}, or {\tt do{\rm--}while} statements, the compiler will
signal an error.

\item A {\em {\tt continue} statement}\/, which has this form:

\begin{flushleft}
{\tt continue ;}
\end{flushleft}

To execute a {\tt continue} statement, the computer jumps to the end of
the smallest {\tt for}, {\tt while}, or {\tt do{\rm--}while} statement
that the {\tt continue} was inside.  Control continues with the {\em
condition}\/ test of the {\tt while} or {\tt do{\rm--}while}, or, if the
{\tt continue} statement was inside a {\tt for} statement, with the {\em
update} expression of the {\tt for}.

If the {\tt continue} statement is not inside any {\tt for}, {\tt
while}, or {\tt do{\rm--}while} statements, the compiler will signal an
error.

\item A {\em {\tt switch} statement}\/, which has this form:

\begin{flushleft}
{\tt switch ( {\em expression} )  } \\*
{\hspace{8pt}\em statement} \\*
\end{flushleft}

To execute a {\tt switch} statement, the computer evaluates the {\em
expression}\/.  Control then passes to the statement immediately after
one of the three following things, which are listed in order of
preference:
\begin{enumerate}
\item A {\em case label}\/ in the {\em statement}, which has the form 
\begin{flushleft}
{\tt case {\em constant} : } \\*
\end{flushleft}

\noindent where the value of the {\em constant}  is the same as the
value of the {\em expression}\/.  If there is no case label whose value
matches that of the {\em expression}, control passes to the statement
immediately after:

\item A {\em {\tt  default} label}\/ in the {\em statement}\/, which has
the form 
\begin{flushleft}
{\tt default : } \\*
\end{flushleft}

\noindent .  If there is no {\tt default \em label}\/, control passes to
the statement immediately after:

\item The end of the {\tt case} statement.  

\end{enumerate}



\end{itemize}

\end{document}

