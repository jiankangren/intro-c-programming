
\documentstyle{article}

% \def\brac#1{{\tt <}#1{\tt >}}
\def\brac#1{$<${#1}$>$}
\def\Int{{\tt int}}
\def\int{\brac{\Int}}
\def\int{\brac{\Int}}
\def\Shortint{{\tt short~int}}
\def\shortint{\brac{\Shortint}}
\def\Longint{{\tt long~int}}
\def\longint{\brac{\Longint}}
\def\Float{{\tt float}}
\def\float{\brac{\Float}}
\def\Double{{\tt double}}
\def\double{\brac{\Double}}
\def\Char{{\tt char}}
\def\chr{\brac{\Char}}
\def\Void{{\tt void}}
\def\void{\brac{\Void}}

\def\ptr#1{pointer~to {#1}}
\def\p2#1{\brac{\ptr{#1}}}
\def\Ano#1#2{array~of {#1}~{#2}s}
\def\ano#1#2{\brac{\Ano{#1}{#2}}}
\def\Ao#1{array~of {#1}}
\def\ao#1{\brac{\Ao#1}}

\def\np{{\tt NULL} pointer}

\def\breakhere{\mbox{$\otimes$}}


\title{Lecture 18}
\author{CSE 110}
\date{4 August 1992}

\parskip 8pt

\begin{document}
\maketitle

\section{The Tower of Hanoi}

As promised, here's the code for the Tower of Hanoi program.

In the program, the three pegs are represented internally by the
numbers 1, 2, and 3.  As written, the program reads a number of rings
from the user, and print out instructions to tell the user how to
transfer a tower of this height from ring 1 to ring 3.  

The workhorse function, {\tt hanoi}, moves a tower of {\tt num\_rings}
rings from peg {\tt start\_peg} to peg {\tt end\_peg}.  The first thing
it does is to figure out where the spare peg is; to do this it uses a
simple trick: If pegs $s$ and $e$ are the start and end pegs, then peg
$6-s-e$ is the spare peg.  (For example, if $s$ is 2 and $e$ is 1, the
spare peg is $6-2-1$ or $3$.)

If the height of the tower that {\tt hanoi} is called upon to move has
size 0, it returns immediately, because it doesn't need to do anything.
Otherwise, it calls itself recursively to move the subtower of ${\tt
num\_rings} - 1$ rings from the start peg to the spare beg, calls {\tt
move} to move the largest ring from the start peg to the end peg, and
then calls itself recursively again to move the subtower from the spare
peg to the end peg.

\begin{flushleft}
\verb%void%
\\* \verb%  hanoi(int num_rings,%
\\* \verb%        int start_peg, %
\\* \verb%        int end_peg)%
\\* \verb%{%
\\* \verb%  int spare_peg = 6 - start_peg - end_peg;%

\verb%  if (num_rings > 0) {%
\\* \verb%    hanoi(num_rings - 1, start_peg, spare_peg);%
\\* \verb%    move(num_rings, start_peg, end_peg);%
\\* \verb%    hanoi(num_rings - 1, spare_peg, end_peg);%
\\* \verb%  }%

\verb%  return;%
\\* \verb%}%
\end{flushleft}

{\tt move} is the function which is called each time we want to move a
particular single ring.  If we were writing our Tower of Hanoi program
to do a fancy screen display which showed the rings flying around from
peg to peg, we would put the code for drawing the rings on the screen in
{\tt move}.  In this simple program, we'll be content to just print out
an instruction about what peg should be moved where:

\begin{flushleft}
\verb%void move(int ring_num, int start, int end)%
\\* \verb%{%
\\* \verb%  printf("Move disk %\verb-%-\verb%d from peg %\verb-%-\verb%d onto peg %\verb-%-\verb%d.\n", %
\\* \verb%         ring_num, start, end);%

\verb%  return;%
\\* \verb%}%
\end{flushleft}

We'll add a {\tt main} which examines its command-line arguments
to decide how many rings the user wants, and then just calls {\tt hanoi}
to print the instructions for moving a tower of that many rings from peg
1 to peg 3:

\begin{flushleft}
\verb%void hanoi(int num_rings, int start_peg, int end_peg);%
\\* \verb%void move(int ring_num, int start, int end);%
\end{flushleft}

\begin{flushleft}
\verb%int main(int argc, char **argv)%
\\* \verb%{%
\\* \verb%  int num_rings;%

\verb%  if (argc != 2) {%
\\* \verb%    fprintf(stderr, "Usage: %\verb-%-\verb%s number_of_rings\n", argv[0]);%
\\* \verb%    return 1;%
\\* \verb%  }%

\verb%  num_rings = atoi(argv[1]);%

\verb%  hanoi(num_rings, 1, 3);%

\verb%  return 0;%
\\* \verb%}%
\end{flushleft}

    The {\tt atoi} function is something new: It accepts a string which
is supposed to contain the string representation of an integer, and it
returns the integer that the string represents.  For example {\tt
atoi("119")} returns the integer 119.  {\tt atoi} returns 0 if there is
an error, for example, because the string passed in was not composed of
digits.

The program is almost trivial with recursion, but it would be very
difficult to do without recursion, because the solution we have, to
reduce the problem to a simpler case, is already organized along
recursive lines.

\section{Divide-and-Conquer Sorting}

For a more serious application of recursion, consider the sorting
problem again.

First observe that simple sorting algorithms, like insertion sort, run much
faster on short lists than on long lists.

\subsection{How Long Does insertion Sort Take?}

    Suppose we have a list of $n$ things, and we want to find the
smallest member.  We have to scan over all the elements in the list,
looking for the smallest one.  The three fundamental operations we have
to perform are looking at an element, comparing it to the current
smallest, and copying the element into our `current smallest' variable
if it is smaller.

    The number of times we have to perform the first two operations is
clearly $n$.  The number of times we have to perform the third operation
is clearly no more than $n$.  If all three operations take 1 unit of
time,\footnote{This is usually true on comventional computers; however,
the main conclusion of this section, that inserion sort takes tmie
proportional to the square of the number of elements in the list, is
still valid even if these three operations do not all take the same
amount of time.} then the running time of our algorithm to find the
smallest element is no more than $3n$.  It can be demonstrated that the
number of times we can expect to have to perform the third step is about
$n+1\over 2$, so the running time of the find-smallest-element operation
averages about $5n+1\over 2$.

Thus if we double the length of the list, the time it takes to find the
smallest element also doubles.

Now consider the insertion sort: We repeatedly find the smallest
element, and copy it to an auxiliary array.  To do this once on a list
of $n$ things takes $5n + 3\over 2$ units of time ($5n+1\over 2$ for the
search and $1$ for the copy).  We have to do it once for each element in
the original list, so the total time to do an insertion sort on $n$
things is about $n\cdot {5n+3\over 2}$ or $5n^2 + 3n \over 2$.  

\begin{tabular}{r|l}
$n$ & $5n^2+3n\over 2$ \\ \hline
1   & 4 \\
2   & 13 \\
3   & 27 \\
4   & 46 \\
5   & 70 \\
10  & 265 \\
20  & 1030 \\
40  & 4060 \\
80  & 16120 
\end{tabular}

It seems that if we double to length of the list, the time it takes to
sort it approximately quadruples.

\subsection{An Improvement to Insertion Sorting}

Suppose we have a list of 20 things we want to sort.  We can see from
the table above that sorting them with insertion sort will take about
1,030 units of time.

Now suppose we broke the list of 20 things into two lists of 10 things.
It would take 265 units of time to sort each of the two lists of 10
things, for a total of 530 units of time.  If we would merge the two
sorted lists together in less than 500 units of time, we'd have
done the sort faster by splitting it into two pieces.

In fact, it's quick and easy to merge two sorted lists.  You look at the
first element in each list, find the smaller of the two, remove it from
the list it's in, and copy it to an auxiliary list.  This takes 4 units
of time.  Then you repeat the process until both lists are empty.  If
the two lists have a total of $n$ things in them, you have to repeat
this operation $n$ times, so it takes $4n$ units of time.  In the case
above, $n$ was 20, so it takes only 80 units of time to merge the two
sorted lists of length 10.  

That means the split-sort-and-merge technique has reduced the running
time of our sort from 1030 units to 610 units.  

If we were sorting 80 things instead, we would reduce the running time
from 16120 units to 8440 ($4060+4060+320$), a savings of almost 50\%.
Clearly this is a substantial improvement.

\subsection{Recursion}

Now suppose we're sorting the 80 things.  We break the list into two
lists of 40 things each.  We need to sort each list.

We've already seen that it's a substantial improvement to break a list
in half, sort the parts separately, and merge them, rather than sorting
the entire list.  So let's take this sub-list of 40 elements and apply
our imporvement, by breaking it into two sub-sub-lists of 20 things
each.  To sort each list of 40 things by this method takes
$1030+1030+160 = 2220$ units of time, for a total of $2220+2220+320 =
4760$ units of time, down from 16120.  

But we already saw that by using the improved method to sort a list of
20 things we could get a speed up, so let's apply the improvement yet
again, breaking the lists of 20 things into two lists of 10 things each.
This reduces the running time of our sort to 3080 units.  The
improvement is less this time, because the savings we got by sorting
lists of 10 things instead of 20 things is not so big, and because the
cost of merging the lists back together an extra time is larger in
proportion.  

Nevertheless if we sort the lists of 10 things by breaking them each
into two lists of 5 things and doing insertion sort on those lists and
merging the results, we can sort our original 80 objects in 2400 time
units, instead of the 16120 that our original straight insertion sort
yielded. 

So here's our improved sorting algorithm:

\begin{enumerate}
\item If the list to sort is very short, use insertion sort.
\item Otherwise, break the list into two lists of approximately equal
size, sort each list with {\em this}\/ algorithm, and merge the two
sorted lists back together.
\end{enumerate}

This algorithm is called {\em mergesort}\/.  We won't see a C
implementation, because to do it proprly you get bogged down in a lot of
little details about how to store the objects, and allocating auxiliary
space, and handling odd-length lists which can't be broken evenly, and
soforth.  But it's clear that the algorithm will be recursive.
Somewhere in our program we will have a function something like this:

\begin{flushleft}
\verb% void mergesort(struct list *data, int length) % \\*
\verb% {% \\*
\verb%   ...% \\*
\verb%   if (length < SMALL_SIZE )  % \\*
\verb%     insertion_sort(data, length);% \\*
\verb%   else {% \\*
\verb%     split_list(data, top, bottom);% \\*
\verb%     mergesort(top,    length/2);% \\*
\verb%     mergesort(bottom, length/2);% \\*
\verb%     merge_lists(top, bottom, data);% \\*
\verb%     return;% \\*
\verb%   }% \\*
\verb% }% \\*
\end{flushleft}

{\tt mergesort} does a little preparatory work, calls itself do to the
bulk of the work on simpler cases, and does a little cleanup work.  

\end{document}
