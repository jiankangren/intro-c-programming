%
% CSE 110 Lecture Notes
%
% Entire contents are copyright 1992 by Mark-Jason Dominus.
% All rights reserved.  Unauthorized reproduction prohobited.
%

\documentstyle[fancyheadings]{article}

% \def\brac#1{{\tt <}#1{\tt >}}
\def\brac#1{$<$#1$>$}
\def\Int{{\tt int}}
\def\int{\brac{\Int}}
\def\int{\brac{\Int}}
\def\Shortint{{\tt short~int}}
\def\shortint{\brac{\Shortint}}
\def\Longint{{\tt long~int}}
\def\longint{\brac{\Longint}}
\def\Float{{\tt float}}
\def\float{\brac{\Float}}
\def\Double{{\tt double}}
\def\double{\brac{\Double}}
\def\Char{{\tt char}}
\def\chr{\brac{\Char}}
\def\Void{{\tt void}}
\def\void{\brac{\Void}}

\def\p2#1{\brac{pointer~to #1}}

\def\breakhere{\mbox{$\otimes$}}
\parskip 8pt

% \def\baselinestretch{2}

\title{Lecture 8}
\author{CSE 110}
\date{13 July 1992}

\pagestyle{fancy}
\lhead{CSE 110 Lecture Notes}
\chead{Mark--Jason Dominus}
\rhead{\thepage}
\lfoot{Copyright \copyright 1992 Mark-Jason Dominus}
\cfoot{}
\rfoot{All rights reserved.}



\begin{document}

\maketitle

\section{A Function to Swap the Values of Two Variables}

    Consider a program which takes a long list of numbers and arranges
them in numerical order.  This operation is called {\em sorting}\/.  By
some estimates, 50\% of all computer use in the world is devoted to
sorting things.  Many typical sorting algorithms work by finding two
elements in the list that are out of order, and then swapping their
positions.  In C that would mean we would have a bunch of variables, one
for each element in the list, and we would swap the values in the two
variables that contained out-of-order elements.

    If we were writing such a program, we would do a lot of swapping.
We can swap the values of two variables, say {\tt x} and {\tt y}, as
follows:

\begin{flushleft}
\verb% temp = x; %
\verb% x = y; %
\verb% y = temp; %
\end{flushleft}

\noindent where {\tt temp} is a variable of the same type as that of
{\tt x} and {\tt y}.  Of course, that'll get ugly and cluttered if we
have to write it a lot; it would be better if we could put it into a
function and simply write something like

\begin{flushleft}
\verb% swap(x,y); %
\end{flushleft}

\noindent to swap the values of variables {\tt x} and {\tt y}.  

\subsection{A First Try}

Here's our first cut at a program which sets up two variables and the
calls a {\tt swap} function to swap their values:

\input swap1.tex

\subsection{Why the First Try Doesn't Work}

If you key this in and compile it, you'll discover it doesn't work.  It
compiles okay, but the values of {\tt x} and {\tt y} never get swapped.
What's wrong?

The problem is this: {\tt swap} only gets the {\em values}\/ of its
arguments {\tt x} and {\tt y}; it never finds out where {\tt x} and {\tt
y} actually are so that it can change them.  Put another way, the values
{\tt swap} receives are {\em copies}\/ of the values stored in {\tt x}
and {\tt y}.  {\tt swap}'s variables {\tt n1} and {\tt n2} are private
variables; they get created when {\tt swap} is called and destroyed
again when {\tt swap} returns.  All the shuffling around of values
that {\tt swap} does is in its own private variables which get
destroyed when it returns.

This is actually a feature and not a bug---normally, of course, we don't
want functions that we call from {\tt main} to be able to mess around
with {\tt main}'s variables.  It would be very bad, for example, if
{\tt printf} surreptitiously changed the computer's secret number in
the middle of your guessing game program.

The way out is the same as with {\tt scanf}:  instead of passing the
values of the variables whose values we want to swap, we'll pass their
addresses.  Then {\tt swap} will know where {\tt x} and {\tt y} are
actually written on the blackboard, and it will be able to change their
values all it wants.  Note that even if we tell {\tt swap} where {\tt
x} and {\tt y} are on the blackboard, it still doesn't know where any of
{\tt main}'s other variables are, so it can only change the ones we
wanted it to.  

This would be a good time to go back and review the notes for
Lecture 5, section 2.3: ``Pointers and the {\tt\&} operator''.

\subsection{This One Works}
\label{swap2}

\input swap2.tex

\subsection{The {\tt\&} Operator Again}

First, note that the call to {\tt swap} in {\tt main} has changed from

\begin{flushleft}
\verb% swap(x, y); %
\end{flushleft}
\noindent to 
\begin{flushleft}
\verb% swap(&x, &y); %
\end{flushleft}

\noindent .   Now, recall what the {\tt\&} operator does:  if {\tt x} is
some variable, then {\tt\&x} describes where that variable can be found
on the blackboard.  The value of {\tt\&x} is said to {\em point to}\/
{\tt x}; since the type of {\tt x} is \int, the type of {\tt\&x} is
\p2\Int.

\subsection{The {\tt *} Operator}

Now we know how to make a pointer: we use the {\tt\&} operator.  Once we
have a pointer, how to we find out or change what it is pointing to?
The complement of the {\tt\&} operator is the {\tt *} operator,
sometimes called the {\em dereferencing operator}\/.  If {\em p}\/ is a
pointer value, then {\tt *}{\em p}\/ is the object to which {\em p}\/
points.  So, for example, the value of {\tt *\&x} is the same as the
value of {\tt x}, because {\tt *\&x} means to get whatever {\tt\&x}
points to, and {\tt\&x} points to {\tt x}.

Let's suppose that {\tt swap} was called from {\tt main} as in the
example above, and let's look at the individual statements in {\tt swap}
and see what they do.  First, when {\tt swap} gets called, {\tt n1} gets
assigned the value of the first argument, which is a pointer to the
variable {\tt x}.  (That is, {\tt\&x}.)  {\tt n2} gets assigned the
value of the second argument, which is a pointer to the variable {\tt
y}.  (That is, {\tt\&y}.)  Then, the line

\begin{flushleft}
\verb% temp = *n1; %
\end{flushleft}

\noindent in {\tt swap} says to find the variable that {\tt n1} points to
({\tt x} in this case), get its value (5), and assign that value to the
variable on the left, {\tt temp}.

The line 

\begin{flushleft}
\verb% *n1 = *n2; %
\end{flushleft}

\noindent says to find the variable that {\tt n2} points to
({\tt y}), get its value (119), and assign that value to the variable
that {\tt n1} points to ({\tt x}).

The line 

\begin{flushleft}
\verb% *n2 = temp; %
\end{flushleft}

\noindent says to get the value of {\tt temp} (5), and
assign that value to the variable that {\tt n2} points to.  ({\tt y} in
this case.)

\subsection{{\tt swap's} Header}

As usual, in {\tt swap}'s header we have to tell the compiler what types
of arguments {\tt swap} will take and what type of return value it will
return.  We say that {\tt swap} returns type {\tt void}.  That means
that {\tt swap} really doesn't return any value.

The parameter declaration {\tt int *n1} declares a variable, {\tt n1},
of type \p2\Int.  Here is how to remember what this means:  If the
declaration had said {\tt int n1}, it would mean that {\tt n1} is an
\int.  But instead it says {\tt int *n1}, so instead it means that {\tt
*n1} is an \int.  Since {\tt *n1}, the thing {\tt n1} points to, is an
\int, {\tt n1} itself must be a \p2\Int.

\subsection{A Note About Functions that Return {\tt void}}

If a function's return type is {\tt void}, that means it doesn't return
a value at all.  To return from such a function, we just write 

\begin{flushleft}
\verb% return ; %
\end{flushleft}

\noindent , omitting the expression that usually follows the word {\tt
return}.

The other way to return from a {\tt void} function is to just let
control flow off the bottom of the function; this is the same as doing
{\tt return~;}.  

If you do {\tt return~;} in a non-{\tt void} function, or let control
flow off the bottom of a non-{\tt void} function, the function may
return a random garbage value to its caller, or your program may fail
completely.

\subsection{More About Prototypes}

In section 2.2 of the notes for Lecture 5, there was a brief note about
{\em prototypes}\/.  You should go and reread that now if you've
forgotten it; it's on page 5.

Here's the problem we must solve: At the time the compiler compiles the
line {\tt swap(\&x, \&y); }, it hasn't seen the definition of {\tt
swap} yet.\footnote{In fact, it's quite possible that {\tt swap}'s
source code resided in a different file entirely, which was compiled
eight months ago and then thrown away.  I am not being facetious.} But
the compiler needs to write the machine instructions for the call to and
return from {\tt swap}, and to write those correctly it needs to know
the types and number of the arguments and the type of the return
value.\footnote{I'm afraid you won't be able to appreciate why this is
until you've studied the inner workings of the compiler in detail.  We
won't do that in this course.} We need a way to give the compiler this
information even in the absence, temporary or otherwise, of the {\tt
swap} function itself.

The way we do that is with a {\em prototype}\/.  To write a prototype
for a certain function, we write an ordinary function header for that
function, but we follow it with a semicolon instead of a function body.
The function header contains all the information the compiler needs to
translate the call and return properly.

That's what the line {\tt void swap(int *n1, int *n2); } is doing at the
top of the program of section \ref{swap2}.  It provides argument and
return value type information for {\tt swap} to the compiler in
advance of the actual definition of {\tt swap}.

If the compiler hasn't seen a prototype for a certain function at the
time it compiles a call to that function, it guesses and does the best
that it can.  It assumes that the function's return value is \int, which
can lead to disaster if it isn't,\footnote{Try writing a program that
uses {\tt sqrt} without including {\tt <math.h>} and you'll see what
kind of horrors can occur---you get completely bogus answers back from
{\tt sqrt} because the compiler thinks that {\tt sqrt} is returning an
\int\ when it's really returning a \double.} and it does the best it can
to handle the arguments, which sometimes works out and sometimes
doesn't.  

You should have a prototype in your program for every function except
{\tt main}.  ({\tt main} always has the same return type and
argument types anyway.)

Library functions like {\tt printf} and {\tt sqrt} must have prototypes
too, so that the compiler can compile the calls to them
correctly,\footnote{In this case the source code for the called function
{\em really}\/ isn't available---it's locked in a vault somewhere at
Borland.} but the prototypes almost always appear in some header file,
and so you include the prototype into your program by including the
appropriate header file.  For example, if you hunt up the {\tt math.h}
header file and look in it, you'll see, along with a lot of other
nonsense, something like

\begin{flushleft}
\verb% double sqrt(double arg); %
\end{flushleft}
\noindent .

The actual definition of a function, the one that supplies the
instructions about how to execute the function, includes a header for
the function, and so it counts as a prototype---after the compiler has
seen an entire function, it certainly has enough information to compile
calls to that function.  If we moved the {\tt swap} function so that it
appeared before {\tt main} in the file, we could omit the prototype,
because by the time the compiler had to compile the call to {\tt swap},
it would have seen the entire definition of the {\tt swap} function and
would have known all about it.

\section{{\tt break}}
\label{break}

The {\tt break} statement interrupts a loop prematurely.  It's easy to
use: You just write {\tt break;}, and if the computer reaches the {\tt
break} statement, control immediately passes the the statement following
the end of the smallest enclosing {\tt while}, {\tt do{\rm\ldots}while},
{\tt for}, or {\tt switch} statement.\footnote{We haven't seen {\tt
switch} yet, but it's coming up soon.}

\subsection{Examples of {\tt break}}

In each of these ghostly examples, the \breakhere\ symbol shows the
place in the program to which the computer skips if it happens to
execute the indicated {\tt break} statement.
 
\begin{flushleft}
\verb% while ( . . . ) { % \\*
\verb%     . . . % \\*
\verb%     if ( . . . ) % \\*
\verb%       break ; % \\*
\verb%     . . . % \\*
\verb% } % \\*
\breakhere
\end{flushleft}


\begin{flushleft}
\verb% while ( . . . ) { % \\*
\verb%     for ( . . . ) { % \\*
\verb%         . . . % \\*
\verb%         if ( . . . ) % \\*
\verb%             break ; % \\*
\verb%         . . . % \\*
\verb%     } % \\*
\verb%    % \breakhere \\*
\verb%     . . .  % \\*
\verb% } % \\*
\end{flushleft}


\begin{flushleft}
\verb% do { % \\*
\verb%     . . . % \\*
\verb%     if ( . . . ) % \\*
\verb%         if ( . . . ) % \\*
\verb%             break ; % \\*
\verb%     . . . % \\*
\verb% } while ( . . . ) ; % \\*
\breakhere
\end{flushleft}

Note that {\tt break} does {\em not}\/ care about {\tt if} (or {\tt
else}) when it breaks; if it did, it would be useless.  (Why?)

If you write a {\tt break} statement that isn't enclosed by a {\tt
while}, {\tt do{\rm\ldots}while}, {\tt for}, or {\tt switch} statement,
the compiler will grouse and refuse to compile your program.

\subsection{{\tt break} Statement Considered Harmful?}

Many early programming languages had only two control structures: They
had an {\tt if{\rm--}else}, and they had the infamous {\tt goto}
statement, which unconditionally transferred control to a certain other
statement.\footnote{This is an oversimplification.  Many early
programming languages, notably LISP, had an utterly different way of
managing control flow in the first place, and this whole debate is moot
for them.  On the other hand, FORTRAN (which is much more like C than
LISP is) had a whole mob of conditional and computed test statements,
all of which ended by transferring control to the statement with a
particular line number.  The point stands anyway.}

Around 1968, imperative languages such as ALGOL (a distant ancestor of
C), were just beginning to have what are known as {\em logical control
structures}, which let you express your algorithms in terms of blocks of
code which were executed when certain conditions held, rather than in
terms of a flow of control which jumped around the program from
numbered statement to numbered statement in response to certain
conditions.

In 1968 a gentleman named Edsger Dijkstra\footnote{In class I said it
was Nicklaus Wirth, but I was mistaken.} wrote a note in {\em
Communications of the ACM}\/, a well-known computer science journal,
called {\em Goto Statement Considered Harmful}.  He had discovered
that when programmers in his organization were forbidden from using {\tt
goto}, and required to use only logical control structures, the programs
they wrote had fewer errors and the errors the programs did have were
easier to fix.

This is a reasonable thing to notice, and, because the logical control
structure proponents were mostly right, most modern imperative
languages, including C, stress logical control structures such as {\tt
while}, and have {\tt goto} only as an afterthought, if at all.

Logical control structures are a little closer to the way we think than
{\tt goto} is.  Rather than giving someone a laundry list of
instructions like ``14. If you're not at the store, {\tt goto} step
11.'' we're more likely to say, ``Keep walking north {\tt until} you get
to the store.''  {\tt goto} is considered bad form in most cases, and
although it occasionally has its uses, it really is better to avoid it
whenever possible.

That warning extends to {\tt break} (and to {\tt continue}, although we
haven't seen that yet): {\tt break} interrupts the logical flow of
control and causes an unconditional jump to somewhere else, and so it
can be confusing; overuse of {\tt break} can obscure what your program
is really doing.  When you see a {\tt while} loop, you normally know
what's going on; you can say, ``Oh, this part of the program tries to do
such-and-so {\tt while} there is data left in the file.''  But if
there's a {\tt break} in the loop, you have to add on a qualifier:
``\ldots unless we hit the {\tt break}.''

I'm tolerant of {\tt break}.  Sometimes it's much easier to express
something with a {\tt break} than it would be without, and the code is
shorter or clearer with the {\tt break} than without.  Nevertheless,
there are a lot of logical-control fanatics in the world, and they'll
tell you never to use {\tt break}, {\tt continue}, or {\tt goto}, and to
{\tt return} only from the bottom of a function, and never from the
middle.  

In the CSE110 handbook, it says never to use {\tt break} or {\tt
continue} in this class, and that you'll lose at least one point on any
assignment you turn in that has a {\tt break} or a {\tt continue}.
That's the logical-control fanatics talking.  Since I am not a
logical-control fanatic, that rule goes out the window.  I am, however,
a simplicity and clarity fanatic, because I spend a substantial part of
my life reading other people's C code.  So the rule I'd rather have you
follow is this: When in doubt, write the code both ways, and pick the
one that seems clearest and simplest.

\end{document}
