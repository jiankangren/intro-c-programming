%
% CSE 110 Lecture Notes
%
% Entire contents are copyright 1992 by Mark-Jason Dominus.
% All rights reserved.  Unauthorized reproduction prohobited.
%

\documentstyle[fancyheadings]{article}

% \def\brac#1{{\tt <}#1{\tt >}}
\def\brac#1{$<${#1}$>$}
\def\Int{{\tt int}}
\def\int{\brac{\Int}}
\def\int{\brac{\Int}}
\def\Shortint{{\tt short~int}}
\def\shortint{\brac{\Shortint}}
\def\Longint{{\tt long~int}}
\def\longint{\brac{\Longint}}
\def\Float{{\tt float}}
\def\float{\brac{\Float}}
\def\Double{{\tt double}}
\def\double{\brac{\Double}}
\def\Char{{\tt char}}
\def\chr{\brac{\Char}}
\def\Void{{\tt void}}
\def\void{\brac{\Void}}

\def\ptr#1{pointer~to {#1}}
\def\p2#1{\brac{\ptr{#1}}}
\def\Ano#1#2{array~of {#1}~{#2}s}
\def\ano#1#2{\brac{\Ano{#1}{#2}}}
\def\Ao#1{array~of {#1}}
\def\ao#1{\brac{\Ao#1}}

\def\np{{\tt NULL} pointer}

\def\breakhere{\mbox{$\otimes$}}


\title{Lecture 19}
\author{CSE 110}
\date{5 August 1992}

\parskip 8pt

\pagestyle{fancy}
\lhead{CSE 110 Lecture Notes}
\chead{Mark--Jason Dominus}
\rhead{\thepage}
\lfoot{Copyright \copyright 1992 Mark-Jason Dominus}
\cfoot{}
\rfoot{All rights reserved.}



\begin{document}
\maketitle

\begin{center}
\begin{quote}
{You taught me Language, and my profit on't \\
is I know how to curse.} \\

{\small --- William Shakspere, {\em The Tempest}, I, ii.}
\end{quote}
\end{center}
\section{Left-Over Language Features}

Today was the day we saw the last of the language features, things that
are rarely or never important.  Each of these I expected to bring up
when the appropriate time came, at some moment when it would be useful,
and the moment never came.  

We'll see these features in increasing order of usefulness.

\subsection{The {\tt ,} Operator}

The {\tt ,} operator takes two operands, which must be expressions.  To
evaluate an expression of the form {\tt {\em expr1} , {\em expr2}}, the
computer evaluates {\em expr1}\/, and then evaluates {\em expr2}\/.  The
value of the entire expression is the value of {\em expr2}\/.  

The {\tt ,} operator has two interesting properties: It guarantees that
{\em expr1}\/ will be evaluated first\footnote{Most operators, such as
{\tt +}, do not guarantee which of their operands will be evaluated
first.}, and it guarantees a sequence point between the evaluation of
the two expressions.  This means that any side effects such as
increments or assignments that are scheduled as a result of evaluating
{\em expr1} will be completely resolved before {\em expr2}\/ is
evaluated.

Nevertheless, the {\tt ,} operator is mostly used when you want to stick
two expressions where only one normally fits, such as in the condition
of a {\tt while} statement.  The example that Kernighan and Ritchie give
for the {\tt ,} operator is a program to reverse a string in place,
which we've already seen:  (They called it {\tt reverse}.)

\begin{flushleft}
\verb% void strrev(char *s)% \\*
\verb% {% \\*
\verb%   int c, i, j;% \\*
\verb% % \\*
\verb%   for (i=0, j=strlen(s)-1;  i<j;  i++, j++) {% \\*
\verb%     c = s[i];% \\*
\verb%     s[i] = s[j];% \\*
\verb%     s[j] = c;% \\*
\verb%   }% \\*
\verb% }% \\*
\end{flushleft}

The commas that separate a function's arguments, the variables in a
declaration, and soforth, are {\em not}\/ {\tt ,} operators.

\subsection{The {\tt ?:} Operator}

The {\tt ?:} is unusual in that it has {\em three}\/ operands.  To
evaluate the expression {\em expr1 {\tt ?} expr2 {\tt :} expr3}\/, the
computer first evaluates {\em expr1}\/.  If {\em expr1} is true, the
computer evaluates {\em expr2}\/ and that value is the value of the
entire expression.  Otherwise, the computer evaluates {\em expr3}\/ and
that value is the value of the entire expression.

A {\tt ?:} expression is like a miniature {\tt if{\rm --}then}.  Here's
an example:  This function prints out the elements of an array,
separated by spaces, with ten elements per line.

\begin{flushleft}
\verb! void print_array(int *a, int nelems) ! \\*
\verb! { ! \\*
\verb!   int i; ! \\*
\verb!  ! \\*
\verb!   for (i=0; i<nelems; i++)  ! \\*
\verb!     printf("%d%c", a[i], (i+1)%10 ? ' ' : '\n'); ! \\*
\verb! } ! \\*
\end{flushleft}

Here's a simpler example:  It sets the value of {\tt a} to the value of
either {\tt x} or {\tt y}, whichever is smaller:

\begin{flushleft}
\verb%  a = (x > y) ? y : x ; %
\end{flushleft}

I swiped this example from K\&R also.

\subsection{The {\tt switch} Statement}

A {\em {\tt switch} statement}\/ has this form:

\begin{flushleft}
\tt
switch ({\em expression}\/) \{ \\*
{\hspace{8pt}}case {\em value1}\/: \\*
{\hspace{8pt}}{\hspace{8pt}}{\em statement11}\/ \\*
{\hspace{8pt}}{\hspace{8pt}}{\em statement12}\/ \\*
{\hspace{8pt}}{\hspace{8pt}}... \\*
{\hspace{8pt}}case {\em value2}\/: \\*
{\hspace{8pt}}{\hspace{8pt}}{\em statement21}\/ \\*
{\hspace{8pt}}{\hspace{8pt}}{\em statement22}\/ \\*
{\hspace{8pt}}{\hspace{8pt}}... \\*
{\hspace{8pt}}... \\*
{\hspace{8pt}}default: \\*
{\hspace{8pt}}{\hspace{8pt}}{\em statementd1}\/ \\*
{\hspace{8pt}}{\hspace{8pt}}{\em statementd2}\/ \\*
{\hspace{8pt}}{\hspace{8pt}}... \\*
\} 
\end{flushleft}


To execute a {\tt switch} statement, the computer evaluates the {\em
expression}\/.  Control then passes to the statement immediately after
one of the three following things, which are listed in order of
preference:
\begin{enumerate}
\item A {\em case label}\/ in the {\em statement}, which has the form 
\begin{flushleft}
{\tt case {\em constant} : } \\*
\end{flushleft}

\noindent where the value of the {\em constant}  is the same as the
value of the {\em expression}\/.  If there is no case label whose value
matches that of the {\em expression}, control passes to the statement
immediately after:

\item A {\em {\tt  default} label}\/ in the {\em statement}\/, which has
the form 
\begin{flushleft}
{\tt default : } \\*
\end{flushleft}

\noindent .  If there is no {\tt default \em label}\/, control passes to
the statement immediately after:

\item The end of the {\tt case} statement.  

\end{enumerate}

Once into the compound statement part of the {\tt switch}, control flows
from top to bottom as usual.  Control does {\em not}\/ leave the
compound statement except by flowing off the bottom of the statement or
by encountering a {\tt break} or {\tt return} as usual.  In particular,
after the computer executes the last statement in one of the {\tt case}
sections, it continues on to the statements under the next {\tt case}
section.  This usually isn't what we want, and so we almost always end
each {\tt case} section with a {\tt break} statement.  

It's so rare to actually allow control to fall through from one {\tt
case} section to the next that you should always put in a comment when
you do let control fall through.  

{\tt case} can always be replaced by {\tt if{\rm --}else if{\rm
--}else}, but the restrictions on it allow the compiler to generate more
efficient code, particularly if there are a lot of {\tt case} labels.  

\subsection{Block-Scoped variables}

We've only seen variables declared at the head of a function.  These
variables had names that were known only inside the function in which
they were declared, and the variables were created each time the
function was called and destroyed again when the function returned.

In fact, you can declare new variables at the head of {\em any}\/
compound statement.  The variables are created when control enters the
statement, and destroyed again when control leaves the statement.  The
names of the variables are known only within the statement.  

It's good to declare variables as belonging to as small a compound
statement as possible, because it keeps the definition and use of the
variable close together, because it restricts communication, and because
it's easier to understand when you're reading the code---you know you
can forget about the variable's value unless you're reading the code for
the compound statement in which it's declared.  

We've actually seen this before:  I used it in the sample solution to
the guessing game project.

\subsection{{\bf true} and {\bf false}}

We've been saying all semester that {\bf false} was a zero value and
that {\bf true} was a nonzero value.  That's only part of the story.

In any condition context, such as the condition part of a {\tt while} or
{\tt for} loop, or the first operand of a {\tt ?:} expression, the
following values mean {\bf true} and {\bf false}:

\begin{tabular}{l|cc}
Type & {\bf false } & {\bf true} \\ \hline
number & 0 & non-zero \\
pointer & {\tt NULL} & not {\tt NULL} \\
character & {\tt '\verb+\+0'} & anything else \\
\end{tabular}

This means, in particular, that all those times we wrote {\tt if (c !=
'\verb+\+0')} we could have just written {\tt if (c)}, and whenever we
wrote {\tt if (newnode != NULL)} we could just have written {\tt if
(newnode)} instead.  

Opinion seems to be divided on whether or not this is bad style.  Since
opinion is divided, and about half the world does in fact use the short
forms, it is therefore good style.

\section{The Compiler and the Linker}

The compiler is really two programs in one.  The first part, the {\em
compiler}\/, translates the C into whatever machine language is
appropriate for the machine you're using.  The translated version of the
program is called an {\em object file}\/. 

The second part of the compiler, the {\em linker}\/, does a simpler job.
It looks through the object file for function calls, and it arranges
that the right functions are called.  When it's done, it might discover
that you called some functions for which you didn't provide code, such
as {\tt printf}; these are called {\em unresolved references}\/.  The
linker then looks through {\em library files}\/, which contain
pre-compiled functions, for functions whose names match the unresolved
references.  If it finds any, it includes the object code for those
functions into your program.  If there are still unresolved references,
the linker complains that it couldn't finish compiling your program;
otherwise it has resolved all the references and creates an executable
program.

Having a linker means that you can write your program in several
different files, compile them separately, months apart, have functions
in one file call functions in another, and the linker will sort out all
the calls at the end. If you change one function, you only need to
recompile the file it's in and then re-link the program; you don't need
to recompile the unchanged source files.  

We didn't see how to do this because we never wrote a program long
enough to deserve it.

\section{Idle Questions}

We had a lot of idle questions about what `object-oriented' means;
how long it takes to write a big program and how big one is; how one
organizes a big programming project, features of other languages, how
widespread various languages are, and soforth.  I can't possibly
remember all the nonsense we discussed, but I do remember that there was
one point I wanted to make that I didn't get a chance to bring up:

\end{document}
