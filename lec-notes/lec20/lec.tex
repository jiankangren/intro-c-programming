%
% CSE 110 Lecture Notes
%
% Entire contents are copyright 1992 by Mark-Jason Dominus.
% All rights reserved.  Unauthorized reproduction prohobited.
%

\documentstyle[fancyheadings]{article}

% \def\brac#1{{\tt <}#1{\tt >}}
\def\brac#1{$<${#1}$>$}
\def\Int{{\tt int}}
\def\int{\brac{\Int}}
\def\int{\brac{\Int}}
\def\Shortint{{\tt short~int}}
\def\shortint{\brac{\Shortint}}
\def\Longint{{\tt long~int}}
\def\longint{\brac{\Longint}}
\def\Float{{\tt float}}
\def\float{\brac{\Float}}
\def\Double{{\tt double}}
\def\double{\brac{\Double}}
\def\Char{{\tt char}}
\def\chr{\brac{\Char}}
\def\Void{{\tt void}}
\def\void{\brac{\Void}}

\def\ptr#1{pointer~to {#1}}
\def\p2#1{\brac{\ptr{#1}}}
\def\Ano#1#2{array~of {#1}~{#2}s}
\def\ano#1#2{\brac{\Ano{#1}{#2}}}
\def\Ao#1{array~of {#1}}
\def\ao#1{\brac{\Ao#1}}

\def\np{{\tt NULL} pointer}

\def\breakhere{\mbox{$\otimes$}}


\title{Lecture 20}
\author{CSE 110}
\date{6 August 1992}

\parskip 8pt

\pagestyle{fancy}
\lhead{CSE 110 Lecture Notes}
\chead{Mark--Jason Dominus}
\rhead{\thepage}
\lfoot{Copyright \copyright 1992 Mark-Jason Dominus}
\cfoot{}
\rfoot{All rights reserved.}

\begin{document}
\maketitle

\section{Divide-and-Conquer Sorting}

For a more serious application of recursion, consider the sorting
problem again.

First observe that simple sorting algorithms, like insertion sort, run much
faster on short lists than on long lists.

\subsection{How Long Does insertion Sort Take?}

    Suppose we have a list of $n$ things, and we want to find the
smallest member.  We have to scan over all the elements in the list,
looking for the smallest one.  The three fundamental operations we have
to perform are looking at an element, comparing it to the current
smallest, and copying the element into our `current smallest' variable
if it is smaller.

    The number of times we have to perform the first two operations is
clearly $n$.  The number of times we have to perform the third operation
is clearly no more than $n$.  If all three operations take 1 unit of
time,\footnote{This is usually true on conventional computers; however,
the main conclusion of this section, that insertion sort takes time
proportional to the square of the number of elements in the list, is
still valid even if these three operations do not all take the same
amount of time.} then the running time of our algorithm to find the
smallest element is no more than $3n$.  It can be demonstrated that the
number of times we can expect to have to perform the third step is about
$(n+1) / 2$, so the running time of the find-smallest-element operation
averages about $(5n+1) / 2$.

Thus if we double the length of the list, the time it takes to find the
smallest element also doubles.

Now consider the insertion sort: We repeatedly find the smallest
element, and copy it to an auxiliary array.  To do this once on a list
of $n$ things takes $(5n + 3) / 2$ units of time ($(5n+1) / 2$ for the
search and $1$ for the copy).  We have to do it once for each element in
the original list, so the total time to do an insertion sort on $n$
things is about $n\cdot (5n+3)/2$ or $(5n^2 + 3n) / 2$.  

\begin{tabular}{r|l}
$n$ & $(5n^2+3n)/ 2$ \\ \hline
1   & 4 \\
2   & 13 \\
3   & 27 \\
4   & 46 \\
5   & 70 \\
10  & 265 \\
20  & 1030 \\
40  & 4060 \\
80  & 16120 
\end{tabular}

It seems that if we double to length of the list, the time it takes to
sort it approximately quadruples.

\subsection{An Improvement to Insertion Sorting}

Suppose we have a list of 20 things we want to sort.  We can see from
the table above that sorting them with insertion sort will take about
1,030 units of time.

Now suppose we broke the list of 20 things into two lists of 10 things.
It would take 265 units of time to sort each of the two lists of 10
things, for a total of 530 units of time.  If we would merge the two
sorted lists together in less than 500 units of time, we'd have
done the sort faster by splitting it into two pieces.

In fact, it's quick and easy to merge two sorted lists.  You look at the
first element in each list, find the smaller of the two, remove it from
the list it's in, and copy it to an auxiliary list.  This takes 4 units
of time.  Then you repeat the process until both lists are empty.  If
the two lists have a total of $n$ things in them, you have to repeat
this operation $n$ times, so it takes $4n$ units of time.  In the case
above, $n$ was 20, so it takes only 80 units of time to merge the two
sorted lists of length 10.  

That means the split-sort-and-merge technique has reduced the running
time of our sort from 1030 units to 610 units.  

If we were sorting 80 things instead, we would reduce the running time
from 16120 units to 8440 ($4060+4060+320$), a savings of almost 50\%.
Clearly this is a substantial improvement.

\subsection{Recursion}

Now suppose we're sorting the 80 things.  We break the list into two
lists of 40 things each.  We need to sort each list.

We've already seen that it's a substantial improvement to break a list
in half, sort the parts separately, and merge them, rather than sorting
the entire list.  So let's take this sub-list of 40 elements and apply
our improvement, by breaking it into two sub-sub-lists of 20 things
each.  To sort each list of 40 things by this method takes
$1030+1030+160 = 2220$ units of time, for a total of $2220+2220+320 =
4760$ units of time, down from 16120.  

But we already saw that by using the improved method to sort a list of
20 things we could get a speed up, so let's apply the improvement yet
again, breaking the lists of 20 things into two lists of 10 things each.
This reduces the running time of our sort to 3080 units.  The
improvement is less this time, because the savings we got by sorting
lists of 10 things instead of 20 things is not so big, and because the
cost of merging the lists back together an extra time is larger in
proportion.  

Nevertheless if we sort the lists of 10 things by breaking them each
into two lists of 5 things and doing insertion sort on those lists and
merging the results, we can sort our original 80 objects in 2400 time
units, instead of the 16120 that our original straight insertion sort
yielded. 

So here's our improved sorting algorithm:

\begin{enumerate}
\item If the list to sort is very short, use insertion sort.
\item Otherwise, break the list into two lists of approximately equal
size, sort each list with {\em this}\/ algorithm, and merge the two
sorted lists back together.
\end{enumerate}

This algorithm is called {\em mergesort}\/.  We won't see a C
implementation, because to do it properly you get bogged down in a lot of
little details about how to store the objects, and allocating auxiliary
space, and handling odd-length lists which can't be broken evenly, and
soforth.  But it's clear that the algorithm will be recursive.
Somewhere in our program we will have a function something like this:

\begin{flushleft}
\verb% void mergesort(struct list *data, int length) % \\*
\verb% {% \\*
\verb%   ...% \\*
\verb%   if (length < SMALL_SIZE )  % \\*
\verb%     insertion_sort(data, length);% \\*
\verb%   else {% \\*
\verb%     split_list(data, top, bottom);% \\*
\verb%     mergesort(top,    length/2);% \\*
\verb%     mergesort(bottom, length/2);% \\*
\verb%     merge_lists(top, bottom, data);% \\*
\verb%     return;% \\*
\verb%   }% \\*
\verb% }% \\*
\end{flushleft}

{\tt mergesort} does a little preparatory work, calls itself do to the
bulk of the work on simpler cases, and does a little cleanup work.  

\section{Optimization and Performance}

By doing a careful analysis and by using a better algorithm, we were
able to improve the speed of a sorting program enormously.  The gain for
a short list, of 80 things, was to cut the running time of the sort by
85\%.  The improvement would be even greater for longer lists; for a
500-element list our mergesort would take about $1\over 30$ the time
that the insertion sort would, instead of $ 1\over 7$.

This demonstration suggests a number of things:  First, that program
performance is largely determined by the overall efficiency of the
algorithm the program uses, and much less so by micro details of the
code.  

Therefore, concern about whether you are doing one extra test or not
each time through your add-a-node-to-a-list function is misplaced.  Such
considerations are usually dominated by more important matters.

Mergesort uses a clever algorithm and sorts $n$ object is time that is
approximately proportional to $n \cdot log n$; straight insertion sort
sorts $n$ objects in time that is approximately proportional to $n^2$.
Changing a sort program to use a fast sort like mergesort rather than a
slow sort like straight insertion sort will improve performance more
than fussing around with extra tests.

There's a saying in the business that there are two rules for when to
optimize:\footnote{The Berkeley {\sc unix} fortune file attributes this
to `Michael Jackson'.}

\begin{enumerate}
\item Don't do it.
\item (For experts only)  Don't do it yet.
\end{enumerate}

This is good advice.  Write the code itself to be a straightforward
implementation of a good algorithm, and write it to be clear to humans
and easy to maintain, rather than to avoid a couple of unnecessary
tests.

If, {\em after}\/ the program is written, it actually turns out to be
`too slow' for some practical use, then optimize first by considering
obvious waste in the program and by researching better algorithms.

If, after due consideration and research, you decide that no faster
suitable algorithm is available, then micro-optimization may be
appropriate. (Then again it may not.)  There are tools available on most
platforms that will analyze a run of your program and tell you where the
program is spending most of its time.  Then you can micro-optimize these
parts.  There's a rule of thumb called the `90-10' rule, which is that
the program spends about 90\% of its running time executing about 10\%
of the program, and the other 90\% of the program is initializations and
special cases that get executed rarely or only once.  

Computer time is cheap these days, and if your program runs a little
slow, that is probably all right.  On the other hand programmer time is
expensive, and if someone has to spend a lot of time figuring out your
code because you made it obscure in an effort to save a few
microseconds, then you've made a bad trade.

The summary is: Write for style, not for efficiency, because style is
probably more important than you think it is. and efficiency is probably
less important than you think it is.  When you direct your attentions
toward efficiency considerations, you should do so in a way that is
likely to yield the most results: Improve the algorithm first, then, if
you must, do real research to find out what parts of your code are
actually slowing down the program, and fix those and nothing else.

\end{document}


