%
% CSE 110 Lecture Notes
%
% Entire contents are copyright 1992 by Mark-Jason Dominus.
% All rights reserved.  Unauthorized reproduction prohobited.
%

\documentstyle[fancyheadings]{article}

% \def\brac#1{{\tt <}#1{\tt >}}
\def\brac#1{$<$#1$>$}
\def\Int{{\tt int}}
\def\int{\brac{\Int}}
\def\int{\brac{\Int}}
\def\Shortint{{\tt short~int}}
\def\shortint{\brac{\Shortint}}
\def\Longint{{\tt long~int}}
\def\longint{\brac{\Longint}}
\def\Float{{\tt float}}
\def\float{\brac{\Float}}
\def\Double{{\tt double}}
\def\double{\brac{\Double}}
\def\Char{{\tt char}}
\def\chr{\brac{\Char}}
\def\Void{{\tt void}}
\def\void{\brac{\Void}}

\def\ptr#1{pointer~to #1}
\def\p2#1{\brac{\ptr#1}}
\def\Ano#1#2{array~of {#1}~#2s}
\def\ano#1#2{\brac{\Ano#1#2}}
\def\Ao#1{array~of #1}
\def\ao#1{\brac{\Ao#1}}

\def\breakhere{\mbox{$\otimes$}}
\parskip 8pt

% \def\baselinestretch{2}

\title{Lecture 11}
\author{CSE 110}
\date{16 July 1992}

\pagestyle{fancy}
\lhead{CSE 110 Lecture Notes}
\chead{Mark--Jason Dominus}
\rhead{\thepage}
\lfoot{Copyright \copyright 1992 Mark-Jason Dominus}
\cfoot{}
\rfoot{All rights reserved.}



\begin{document}

\maketitle

\section{Examples of Pointer Arithmetic}

In each of the following examples, suppose that {\tt array} is an \ano
3\Int, whose elements contain the values 5, 23, and 119, respectively.
Suppose {\tt p} and {\tt q} are pointer variables of type \p2\Int, which
point initially to the first element of the array {\tt array}, and that
{\tt d} is an \int\ variable.  Then:

\begin{itemize}
\item This statement
\begin{flushleft}
\verb% q = p + 1; % 
\end{flushleft}

computes the value of {\tt p + 1}, which is a pointer to {\tt array}'s
second element (by the pointer arithmetic rule), and assigns that
pointer value to the variable {\tt q}.  {\tt q} now points to {\tt
array}'s second element.  {\tt p} has not changed.

\item The statement
\begin{flushleft}
\verb% q = p++; % 
\end{flushleft}

gets the value of {\tt p}, assigns that value to {\tt q}, and makes a
note to bump up the value of {\tt p} by one before the statement is
over.  So {\tt q} ends up pointing to the first element of {\tt array},
and {\tt p}, which got bumped up, points to the the second element of
{\tt array}.

\item This statement is illegal:

\begin{flushleft}
\verb% q = *(p + 1);%
\end{flushleft}

\noindent because {\tt p} is a \p2\Int, and so {\tt p + 1} is also a
\p2\Int---it points to the element after the one that {\tt p} points to.
Thus \mbox{\tt *(p + 1)} is the thing that \mbox{\tt p + 1} points to, and is
therefore an \int.  {\tt q}, on the other hand, is a pointer variable.
You can't store an \int\ in a pointer variable, because pointers are not
\int s. The compiler will refuse to compile this statement.

\item Perhaps what we meant to write in the last example was this:

\begin{flushleft}
\verb% *q = *(p + 1);% \\*
\end{flushleft}

This gets the value of the element after the one {\tt p} is pointing to,
and assigns that value to the element that {\tt q} is pointing to.  {\tt
q} is pointing to the first element of {\tt array}, so that first element
gets assigned the value 23. Neither {\tt p} nor {\tt q} changes; both
are still pointing to the first element of {\tt array}.  But {\tt array}
itself now contains the values {\tt \{23, 23, 119\}}.  

\item Or, perhaps, what we meant to say in the incorrect example was
this: 

\begin{flushleft}
\verb% d = *(p + 1);%
\end{flushleft}

{\tt d} is an \int, \mbox{\tt *(p + 1)} is an \int, so things work out.  \mbox{\tt
p + 1} points to the second element of {\tt array}, and \mbox{\tt *(p + 1)} is
this second element itself.  Since the expression \mbox{\tt *(p + 1)} is in a
value context, it's evaluated to produce a value, and the result is the
contents of the second element of {\tt array}, which is 23.  This value is
assigned to the \int\ variable {\tt d}.  The value of {\tt p} does not
change. 

\item On the other hand, this is something else again:

\begin{flushleft}
\verb% d = *p + 1;%
\end{flushleft}

{\tt p} points to the first element of {\tt array}, so {\tt *p} gets the
value of that element, 5.  Then the computer adds 5 and 1 and assigns
the result, 6, to the variable {\tt d}.  Neither {\tt p} nor {\tt *p}
change.  

\item Unary operators like {\tt ++} and {\tt *} take precedence from
right to left.  That means that if you see {\tt *p++}, it means {\tt
*(p++)} and not {\tt (*p)++}.  What's the difference between these two
expressions?

\begin{flushleft}
\verb% d = *(p++);% 
\end{flushleft}

\noindent says to get the value of {\tt p}, which is a pointer to the first
element of {\tt array}, and to make a note to bump up {\tt p} by the end
of the statement.  Then, the computer finds the thing that {\tt p} is
pointing to, which has value 5, and assigns that value to {\tt d}.
After the statement, {\tt d} got the value 5 and {\tt p} was bumped up
to point to the second element of {\tt array}.  None of the values in {\tt
array} changed.  On the other hand,

\begin{flushleft}
\verb% d = (*p)++;% 
\end{flushleft}

\noindent says to get the thing that {\tt p} points to, namely the first
element of {\tt array}, get its value, and make a note to bump that value
up by 1 before the end of the statement.  The value of the first element
of {\tt array} gets assigned to {\tt d}, and then the value gets bumped up
by 1.  So {\tt d} gets the value 5, and {\tt p} hasn't changed---it
still points to the first element of {\tt array}, whose value is now 6
instead of 5.

In short:  {\tt p++} means to bump up {\tt p} so that it points to the
next element in the array.  {\tt (*p)++} means to bump up the value of
the thing {\tt p} points to, but to leave {\tt p} itself unchanged so
that it points to the same place. 

\end{itemize}

\section{Too Much Work}

C has a philosophy that it won't perform any operation unless that
operation is completely trivial.  For example, there is no way to say
``Add the value 3 to every element of the array {\tt a}.''  You can do
it, but you have to code the loop yourself.  That's because there is
usually more than one way to perform any nontrivial operation, and C
does not want to be in the position of having to figure out what way is
best for your particular application.  C lets the programmer decide what
method is best.\footnote{This philosophy is why there is no automatic
run-time array bounds checking in C: If it were there and your program
needed to run quickly, you would be out of luck, because there would be
no way to take the bounds checking out.  On the other hand, if you
really did want it there, and you were willing to pay the speed penalty,
you could write the code for the array bounds checking yourself.  There
is a trade-off between speed and safety, and C does not presume to know
which one you value more highly.}  

Accordingly, there is no operation in C that operates on all the
elements of an array at once.  You always have to use a loop.  Operating
on all the elements of an array at once is ``Too Much Work.''  You can
get C to do a little bit of work at a time, working on one array
element, and then the next, and then the next, but that is the way you
have to say it.

In fact, C goes even farther than that.  Even manipulating an entire
array at once is ``Too Much Work''.   In C, there are no array values.  

\subsection{No Array Values}

This is a little surprising.  If {\tt i} is an \int\ variable, {\tt i}'s
value is an \int\ value.  If {\tt pc} is a \p2\Char\ variable, {\tt
pc}'s value is a \p2\Char\ value.  

But: If {\tt af} is an \ao\Float\ variable, {\tt af}'s value is {\em
not}\/ an \ao\Float\ value, because there are no array values.

So suppose {\tt af} is an \ao\Float s; what is the value of the
expression {\tt a}?

\subsection{The Array Value Rule}

No expression has a value which is an array.  If {\tt a} is an
\ao{thing}, then the value of {\tt a} is a pointer to the first element
of the array {\tt a}, and has type \p2{thing}.

\subsection{Implications for Pointer Arithmetic}

The array value rule says that the value of an array {\tt a} is a
pointer to {\tt a}'s first element.  That means that in a value context,
where {\tt a} is being evaluated to produce a value, the expressions
{\tt a} and {\tt\&a[0]} are interchangeable.  The first denotes the
array itself, and the second is a pointer ({\tt \&}) to {\tt a}'s first
element ({\tt a[0]}), but the values of the two expressions are the
same.  The two expressions therefore behave differently from one another
only in an object context.  

Now, the pointer arithmetic rule tells us that if {\tt p} points to the
first element of an array, say to {\tt a[0]}, then the expression {\tt p
+ 1}, by definition, is a pointer which points to the second element of
that array, namely {\tt a[1]}, and that {\tt p + n}, in general, points
to {\tt a[n]}.

The expression {\tt a} is just such a pointer---the array value rule
says that it points to the first element of the array {\tt a}.
Therefore, {\tt a + n} points to the element {\tt a[n]}.  

Since {\tt a + n} is a pointer to the element {\tt a[n]}, the expression
{\tt *(a + n)} denotes the element {\tt a[n]} itself.  The expression
{\tt *(a + n)} is an lvalue expression, because it represents an object:
the element of array {\tt a} with index {\tt n}.

Here's the punch line:  The expression {\tt a[n]} is in all ways
completely identical  with the expression {\tt *(a + n)}.  When the
compiler needs to compile an array reference such as {\tt a[n]}, it
compiles it as though you had written {\tt *(a + n)}.  Array
subscripting is only a notation convenience, and everything we want to
do with array can be done with pointers instead, because of C's powerful
pointer arithmetic convention.

\section{Functions that Operate on Strings}

We'll apply our powerful theory of arrays and pointers, and write some
functions that operate on strings.  First we'll write {\tt strlen}, a
function which takes a string as an argument and returns the number of
characters in the string.  

We'll be calling {\tt strlen} this way:
\begin{flushleft}
\verb% char a[] = "Jean Ogrinz";% \\*
\verb% int length; % \\*
\verb% % \\*
\verb% length = strlen(a);% \\*
\verb% % \\*
\end{flushleft}

Now, if we want to write {\tt strlen}, we need to decide what its header
looks like.  Clearly it'll return an \int.  What's its argument?  A
string, which is an \ao\Char, but that's not quite right.  The argument
to a function is in a value context, and so the {\tt a} in {\tt
strlen(a)} is in a value context and is being evaluated to produce a
value.  The array value rule says that the value of an array in a value
context is a pointer to the array's first element.  That means that when
{\tt strlen} is really being passed is a pointer to {\tt a}'s first
element, which is a \p2\Char.  So {\tt strlen} will begin like this:

\begin{flushleft}
\verb% int strlen(char *s)% \\*
\end{flushleft}

here's the rest of {\tt strlen}:

\begin{flushleft}
\verb% int strlen(char *s)% \\*
\verb% { % \\*
\verb%   int i = 0; % \\*
\verb%  % \\*
\verb%   while (*s != '\0') { % \\*
\verb%     i++; % \\*
\verb%     s++;      /*% Make $s$ point to the next character \verb%*/ % \\*
\verb%   } % \\*
\verb%  % \\*
\verb%   return i; % \\*
\verb% } % 
\end{flushleft}

\subsection{How {\tt strlen} Works}

When {\tt strlen} starts up, {\tt s} is initialized to point to the
first character of the string we want to examine, as we've said.
Since {\tt s} is a \p2\Char, {\tt *s} is the \chr that {\tt s} is
pointing to.  

We run through the {\tt while} loop as long as {\tt *s} is not the NUL
character.  Each time through, we increment {\tt i}, which is a counter
of how many characters we've seen so far, and we increment {\tt s},
which makes it point to the next character in the string.  When that
character is the NUL character, we quit the loop and return the count.


\end{document}
