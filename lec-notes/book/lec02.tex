%
% CSE 110 Lecture Notes
%
% Entire contents are copyright 1992 by Mark-Jason Dominus.
% All rights reserved.  Unauthorized reproduction prohobited.
%

\section{A Warning}

    I think that in order to program effectively, you have to understand
how a computer really works, and you also have to understand what the
compiler is really doing.  Mainstream computer science pedagogues
disagree with me about this.  Nevertheless, because I am an employee of
the university and not a graduate student, I am accountable to
practically nobody and so we can learn this the right way instead of the
usual way.

    However, that means that before we dive into the syntax of the
language, we have to spend some time discussing some of the details of
how a computer works, and we don't get to dive into the syntax of the
language itself for a while.  I think that is all right---C is a
language for talking about what goes on in computers, and if you don't
know what goes on in computers you won't have anything intelligent to
say in C.

    So if these first sections are not what you expected, please be a
little patient.

\section{The Blackboard}

\subsection{Bytes}

    Computer memory is like a blackboard divided into squares.  The
squares are called {\em bytes}\/. Each byte can hold one piece of
information.  Each square has a name, which is called its {\em
address}\/.  Like street addresses, the names are numbers.

    Bytes vary in size from computer to computer, but on most modern
computers each byte contains eight {\em bits}.  The {\em bit}\/ is the
smallest possible amount of information.  It represents the amount of
information that one can store in a physical object which can be in one
of only two states.  For example, one bit of information suffices to
describe the state of a light switch: the switch is either on or
off.\footnote{Objects that can be in only one state clearly can't be
used to store any information.  It turns out that the amount of
information you can store in an object with $n$ possible states is no
more than $log_2 n$ bits.  These notions were first clarified by Claude
Shannon of Bell Laboratories in the middle 1950's.} Since there are 8
bits in a byte, each byte is in one of $2^8$ states.  $2^8$ is 256, so
we usually imagine that the information in a byte is a number between 0
and 255.

    So each byte has exactly two properties: It has a value, which we
can think of as a number between 0 and 255, and it has an address which
tells the computer where it is on the microchip.  The address is the
only thing that distinguishes two bytes, so two bytes never have the
same address.  Of course two different bytes can have the same value.


\subsection{Bytes are Enough to Represent All Kinds of Data}

    Bytes only hold numbers between 0 and 255, which is too small a
range to be useful.  So when we want to talk about a bigger number, we
use an old trick.  In our number system, we have only ten symbols: 0, 1,
2, 3, 4, 5, 6, 7, 8, and 9.  When we want to write a number bigger than
9, we write two or more symbols together and say that the symbols mean
different things depending on where they are relative to one another.
For example, the number 674 means 6 hundreds, 7 tens, and 4 units.
Because the 6 contributes most to the value of the numeral 674, we say
it is the {\em most significant digit}\/.  Similarly, the 4 is the {\em
least significant digit}\/.  This is called a {\em place value} system
because the value of each symbol depends on the place it is on the
paper.  We say it is a {\em base 10} or {\em decimal}\footnote{{\em
Decimal}\/ is from {\em Decem}\/, the Latin word for `ten'} system
because each place is worth 10 times as much as the one to its right.

    We do the same thing with the bytes on the blackboard, only we use a
base 256 system.  If we want to represent a number bigger than 255, we
use two or more bytes and let some of them be more significant than
others: each byte is worth 256 times as much as the next.  For example,
to represent the number 674, we put 2 in the most significant byte and
162 in the least significant byte.  The 162 in the least significant
byte is worth 162.  The 2 in the most significant byte is worth 256
times as much as a 2 in the least significant byte would be, so it is
worth $256\cdot 2$ or 512.  Thus the total value of this two-byte object
is $162+512 = 674$.

    Clearly if we're going to have multi-byte objects then it's no
longer sufficient to only keep track of where a certain number is in the
computer's memory: we also need to keep track of how many bytes are
being used to store that number.  If someone writes a two-byte number
into memory and then tells us where it is so that we can look at it or
change it, they had better tell us how long it is, or else we might not
look at all of it (which would be like someone writing the number 674 on
the blackboard and us thinking that it was three numbers, a 6, a 7, and
a 4), or else we might look at too much (which would be like someone
writing the numbers 674 and 523 on the blackboard very close together
and us thinking that they were the single number 674523).  Either would
be a big disaster.

    Mostly the compiler keeps track of the lengths of things and handles
them for you.  That's one of the big reasons for having a computer
language and a compiler in the first place.

    Also we want to store other information than just numbers.  Say we
want to store someone's name---how do we do that with just numbers
between 0 and 255?  Of course the answer is that we assign a number to
each letter of the alphabet, and we store the numbers that correspond to
each letter in the name.  It would be a big disaster if we forgot that
this sequence of numbers was actually supposed to represent a sequence
of letters.  Another reason we have computer languages is to keep track
of which numbers on the blackboard really represent letters and which
ones really represent numbers.

\subsection{Variables and Types}

    So suppose we are going to write a program that will need to
remember a number--say it's a program that thinks of a secret number
between 1 and 10000, and then the user tries to guess what that number
is.  We need to find a blank spot on the blackboard big enough to store
the number.  We need at least two bytes to store the number in, because
it might be bigger than 255.  (Two bytes are enough to store numbers
between 0 and 65,535.)  We also want to remember that this is a two-byte
number and not the first half of a four-byte number, and not the first
two letters in someone's name, or anything like that.

    Here's what we say in C to accomplish all that:

\begin{flushleft}
\verb%    int my_number;%
\end{flushleft}

    {\tt int} is short for `integer', which is a mathematical word for a
whole number.  The actual size of an \int\ depends on what computer
and what compiler you are using, but it is never less than two bytes.
When the compiler sees this instruction, it finds enough free space on
the blackboard to hold an \int, reserves it, and makes a note to
itself that from now on, this space is going to hold an \int\ and
that the name {\tt my\_number} refers to the space.\footnote{Actually the
compiler doesn't find the space right away; rather, it writes out
machine instructions into the executable version of your program that
find the space when they are executed.} You don't need to know the
details about how it finds this space or where it is; all you need to
know is that the space is there and that you can call it {\tt
my\_number}.  The semicolon ({\tt;}) is just punctuation; it tells the
compiler where the end of the statement is.

    This whole process is called {\em declaring}\/ a {\em variable}\/.
The {\em name}\/ of the variable is {\tt my\_number} and the {\em type}\/
of the variable is \int.  A variable has two other properties:  Its
value (which depends only on the values of the bytes that make it up and
on its type), and its address, which is the same as the address of the
first byte that makes it up.  Unless you say otherwise, \int\
variables always start with the value 0.

    In some languages you can use variables without declaring them; what
happens then is that the compiler assumes that they have a certain type.
You can't do that in C.  If you name a variable you haven't declared,
the compiler will signal an error.

\subsection{Other Types}

    C has a few other primitive types:

\begin{itemize}

\item \shortint\  and \longint\  are like \int, but might
be shorter or longer.  \shortint\ is always at least two bytes long, and
it's never longer than {\tt int}.  \longint\ is always at least four
bytes, and it's never shorter than {\tt int}.  How long \shortint\ and
\longint\ actually
are depends on your compiler.  \int\ is supposed to be a size that is
convenient for your computer and you should use \int\ rather than
\longint\ or \shortint\ unless there's a good
reason.  You use \shortint\ to save blackboard space and
\longint\ when you need to handle bigger numbers.

\item \chr\ is short for {\em character}\/; you use it when you
want to store a character, which could be a letter, a digit, or any
other symbol on the keyboard.  It's always exactly one byte long.

\item \float\ is short for {\em floating-point number}\/.  A
floating point number is one which is represented internally in
scientific notation, with a mantissa and an exponent.\footnote{This
means, for example, that a number like 1234567890 is stored as
$1.234567890\times 10^9$.  1.234567890 is the {\em mantissa}\/ and 9 is
the {\em exponent}\/.  If you don't understand this, don't worry; we
won't need it for a long time.} This means that a \float\ can represent
a fractional number like $3\over 4$ or $\pi$.  A \float\ variable has a
wide range, and can typically hold a number between about $10^{-38}$ and
about $10^{38}$, but it loses precision towards the ends of its range.
{\tt float}s on typical machines are four bytes long.  There's a more
precise version of a \float\ called a \double, which is twice as long
and twice as accurate.  Similarly, there are
\verb+<+{\tt long double}\verb+>+s which are even longer than ordinary
\double s.

\end{itemize}

That's all the simple types there are.  There are other types, but
nearly everything else is built up from these few.

