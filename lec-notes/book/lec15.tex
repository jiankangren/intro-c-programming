%
% CSE 110 Lecture Notes
%
% Entire contents are copyright 1992 by Mark-Jason Dominus.
% All rights reserved.  Unauthorized reproduction prohobited.
%

\section{{\tt fgets}}

{\tt fgets} is an awfully useful function.  It takes three arguments.
The third argument is a stream.  {\tt fgets} reads characters from the
stream.  The first argument is a pointer to a space in which {\tt fgets}
can store the characters it reads.  The second argument is an \int; it
says how big that space is.  If the second argument is $n$, then {\tt
fgets} will read no more than $n-1$ characters from the stream and store
them into the buffer.  It might read fewer, because {\tt fgets} stops
after it reads a newline character.  

After storing the characters it read into the buffer, it sticks a {\tt
NUL} character on the end.  (That's why it reads at most $n-1$
characters; it needs to save space for the {\tt NUL}.)  

If it succeeds, {\tt fgets} returns its first argument, which is not
usually very useful since we already know what the first argument was.
But if it fails, {\tt fgets} returns the {\tt NULL} pointer.

In short, {\tt fgets} reads a line of input from a certain stream,
stopping when the buffer it's storing the input into is full.

\subsection{{\tt gets}}

{\tt fgets} has a dysfunctional little brother, {\tt gets}.  {\tt gets}
only reads from the standard input, its only argument is a pointer to
the space to in which you want the input stored.  

The question:  How does {\tt gets} know how much space you've arranged
for it?

The answer:  It doesn't.

If the line that {\tt gets} is reading is too big to fit in the array
you've provided, {\tt gets} happily writes past the end of the array and
destroys who-knows-what.

For this reason, serious programmers never use {\tt gets}.

\section{Debugging Facilities}

Turbo-C++ has reasonably good debugging facilities.  You can step
through part of your program one statement at a time, watching how
certain variables change, and you can run your program normally and have
it pause when control reaches a certain line.

This is all covered in detail in your texbook, pages 723--733.

\subsection{Stepping}

The {\tt F7} and {\tt F8} keys will run your program one step at a time;
each time you press one, your program will run one more statement.  When
you press {\tt F7} or {\tt F8}, the program runs, and stops again when
control reaches the next line. 

The difference between {\tt F7} and {\tt F8} is that if the current line
contains a function call, the pressing {\tt F7} will go stop at the next
executed line, which is inside the called function, but {\tt F8} will
run the function calls and stop only when control reaches the next line
in the calling function.  {\tt F8} skips the details of the function
calls; they're run, but you don't have to step into them and go through
the one step at a time like you do with {\tt F7}.

\subsection{Inspecting Variables}

Under the {\tt Debug} menu is a selection called {\tt Inspect}.  If you
select this menu item, you'll get a dialog box that lest you enter an
expression.  A window will pop up on the screen that displays the value
of the expression.  For example, if you wanted to see the current value
of the expression {\tt *ptr}, type {\tt *ptr} and a window will pop up
with the value of {\tt *ptr}.  The value in this window will change if
the value of {\tt *ptr} changes as you step through the program.

\subsection{Setting Breakpoints}

Normally when you pick {\tt Run} from the {\tt Run} menu, the program
runs without stopping.  You can set a {\em breakpoint} at a line of
code, and execution will pause when control reaches that line.  Then you
can examine the values of variables with the inspector, step through
critical sections of code one step at a time, or set or clear more
breakpoints.  

To set a breakpoint on a line, move the cursor to that line and choose
{\tt Toggle Breakpoint} from the {\tt Debug} menu.  When you {\tt Run}
the program, control will stop when control reaches the line with the
breakpoint.  To continue execution normally, choose {\tt Run} again; the
program will pick up where it left off and run until it hits another
breakpoint.  

To remove a breakpoint, move the cursor to the line with the breakpoint
and choose {\tt Toggle Breakpoint} again.

\subsection{Restarting the Program}

The {\tt Run} command only starts your program from the beginning if
you've changed it or if it's completely finished.  If you've stopped at
a breakpoint or stepped partway through the program, the {\tt Run}
command tells Turbo-C++ to continue running the program where it left
off.   

If you really to want to start all over again from the beginning, choose
{\tt Program Reset} from the {\tt Run} menu.

\subsection{Setting Command-Line Arguments}

You don't have to break into a DOS shell to run your program with
command-line arguments.  If you choose {\tt Arguments} under the {\tt
Run} menu, you can enter the arguments you want to run your program with.
