%
% CSE 110 Lecture Notes
%
% Entire contents are copyright 1992 by Mark-Jason Dominus.
% All rights reserved.  Unauthorized reproduction prohobited.
%

\section{More Loops}

Looping is so important that there are three ways to do it.

\subsection{{\tt do{\rm \ldots}while}}

Consider this example:
\begin{flushleft}
\verb% % \\*
\verb% int n = 0;% \\*
\verb% % \\*
\verb% while ( n<1 || n>10 ) { % \\*
\verb%   printf("Please enter an integer between 1 and 10.\n");% \\*
\verb!   scanf("%d", &n);! \\*
\verb% } % 
\end{flushleft}
Note that {\tt n} is initialized to 0; if it were initialized to 7
instead for some reason, we'd never prompt at all, because the condition
would be false the first time we got to the {\tt while} statement.  We
had to be careful to fix up {\tt n} in advance to make sure that the
loop gets executed at least once.

There's another loop construct like {\tt while}, which isn't as often
used, but which might have been better for the example above.  It looks
like this:

\begin{quote}
{\tt do} {\em statement}\/ {\tt while ({\em condition}\/) ;}
\end{quote}

The difference is that the condition is tested {\em after}\/ the
statement is executed, instead of before.  That means that the statement
is always executed at least once.  For example:

\begin{flushleft}
\verb% % \\*
\verb% int n = 7;% \\*
\verb% % \\*
\verb% do { % \\*
\verb%   printf("Please enter an integer between 1 and 10.\n");% \\*
\verb!   scanf("%d", &n);! \\*
\verb% } while ( n<1 || n>10 ) ; % \\*
\end{flushleft}

\noindent Now it doesn't matter that {\tt n} is initialized to 7,
because we prompt and ask for a number from the user at least once
anyway, and the 7 is obliterated by the call to {\tt scanf()} before we
ever look at the value of {\tt n}.

Use {\tt while} when it might be appropriate for the actions in the body
to never happen at all; use {\tt do{\rm\ldots}while} when you always
want the actions to be executed at least once.

\subsection{{\tt for}}

This code exemplifies the notion that every loop has four logical
parts:

\begin{flushleft}
\verb% int x = 1;% \\*
\verb% % \\*
\verb% while (x <= 50) {% \\*
\verb!   printf("%d ", x);!	 \\*
\verb%   x++; % \\*
\verb% } %
\end{flushleft}

    There's an {\em initialization}\/ to set up the variable or
variables that form the main control of the loop; that's the {\tt x =
1}.  There's a {\em control expression}\/ that says how long to run the
loop and when to stop; that's {\tt x <= 50} here.  There's a {\em
control update}\/, which updates the values the main control variables;
that's {\tt x++} here.  And there's a {\em loop body}\/, which contains
the statements that the control expression actually controls; that's the
{\tt printf()}.

    There's a loop construct that makes all four parts explicit.  It
looks like this:

\begin{quote}
{\tt for ( {\em initialization}\/ ; {\em control}\/ ; {\em update}\/ ) } \\*
{\em \mbox{\hspace{.25in}} loop body}\/
\end{quote}

This is functionally identical\footnote{Not quite identical, but the
only difference is in the behavior of {\tt continue} statements, which
we'll see later.} to this code:

\begin{quote}
{\em initialization}\/ \\*
{\tt while ( {\em control} ) \{ } \\*
{\em \mbox{\hspace{.25in}} loop body}\/ \\*
{\em \mbox{\hspace{.25in}} update}\/ \\*
{\tt \} }
\end{quote}

For example, here's our example that prints the numbers from 1 to 50
again:

\begin{flushleft}
\verb% int x; % \\*
\verb% % \\*
\verb% for ( x=1; x<=50; x++ ) % \\*
\verb!     printf("%d ", x); ! 
\end{flushleft}

When to use {\tt for} and when to use {\tt while} is a matter of style.
Typically, we like to use {\tt for} when the initialization, control,
and update expressions are all closely related, because it keeps them
together, and {\tt while} in other cases, because it doesn't emphasize a
connection that isn't there.

Sometimes for some reason we don't want to have an initialization in a
{\tt for} statement.  In that case we just omit it and leave a blank
space between the open parenthesis and the first semicolon after the
{\tt for}.  Similarly, we can omit the update expression.  We can even
omit the control condition, in which case the computer will take it as
being always true.  So as a special case, 

\begin{flushleft}
\verb% for ( ; ; ) { ... } %
\end{flushleft}

\noindent performs the statement delimited by the braces over and over,
forever.\footnote{There are ways to get out of this statement:  You
could use {\tt return}, for example.  Other ways we haven't seen yet
include {\tt goto} and {\tt break}.}  This is functionally identical with

\begin{flushleft}
\verb% while ( 1 ) { ... } %
\end{flushleft}

\section{Example: A Program to Compute Prime Numbers}

A {\em prime number} is a positive integer, $n$, which is evenly
divisible only by $n$ and by 1.  For example, 2, 3, 5, 7, 11, and 13 are
prime numbers.  4 is not prime, because it is divisible by 2; 6 and 9
are not prime because they are divisible by 3.

This program reads number inputs from the user and says whether they are
prime or not, until the user enters 0, when it exits.  It checks a
number $n$ to see if it is prime by dividing it by each number $j$ between
$2$ and $n\over 2$; if the remainder, computed with the {\tt\%}
operator, is zero, then $n$ is evenly divisible by $j$ and so is not
prime.  

\input primes.tex
