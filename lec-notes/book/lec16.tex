%
% CSE 110 Lecture Notes
%
% Entire contents are copyright 1992 by Mark-Jason Dominus.
% All rights reserved.  Unauthorized reproduction prohobited.
%

\section{Dynamic Memory Allocation}

We are finally going to learn how to allocate memory on the fly, getting
more when we need it, a subject not usually covered in introductory C
courses.

\subsection{{\tt malloc}}

We've already seen {\tt strdup}, which finds memory somewhere and
reserves it.  How did {\tt strdup} find that memory?

Chances are it called {\tt malloc}.  {\tt malloc} finds memory by asking
the operating system for a big bunch of memory and then parceling it out
a little at a time as you ask for it.\footnote{{\tt malloc} does it this
way, instead of going to the operating system every time, because
operating system requests are very slow.} It's {\tt malloc} that takes
care of recording how big each parcel is so that {\tt free} can free it
properly.

To call {\tt malloc}, pass as an argument the number of bytes of space
you want to allocate.  {\tt malloc} will reserve that many bytes and
return a pointer to the memory it found.  If it fails for any reason,
such as because all the memory is already reserved, {\tt malloc} returns
(all together now) the {\tt NULL} pointer.  

\subsection{{\tt sizeof}}

Frequently you want to allocate enough memory to store an object of a
certain type; you need to be able to tell {\tt malloc} how many bytes of
space the object takes up so it knows how much to allocate.  There's an
operator in C which tells you how big an object is: {\tt sizeof}. 

If {\em type} is the name of a type, the value of the expression {\tt
sizeof({\em type}\/)} is the number of bytes needed to hold that type;
for example, on our machines, {\tt sizeof(int)} is 2 and {\tt
sizeof(double)} is 8.  

If we want to get enough space for an array of 13 \int s, we can do {\tt
malloc(13 * sizeof(int))} or even {\tt
malloc(sizeof(int[13]))}.\footnote{{\tt int[13]} is the name of the type
\ano{13}{\Int}.  To construct the name of a certain type, just write a
declaration for a variable of that type, and then discard the variable
name and the semicolon from the declaration.}

There's another way to use {\tt sizeof}: If $e$ is an expression with
type $t$, then the value of {\tt sizeof($e$)} is the same as the value
of the expression {\tt sizeof($t$)}.

\section{{\tt struct}s}

We've seen how to use an array to store many values of the same type,
and even to treat the collection of values as a unit.  Now we'll look at
the complementary type, the {\em structure}\/, which allows you to group
several different object together as a unit.

\subsection{Creating a New Structure Type}

The typical example of a structure is this:  You want to keep track of
the employees in your organization.  Each employee has a name (a string
of no more than 50 characters), a social security number (a \longint),
and a salary (an \int).  You want to handle several such records.

You can create a new type, a {\tt struct employee}, which keeps all of
this information in one place.  First, you define the new type:

\begin{flushleft}
\verb% struct employee {% \\*
\verb%   char name[51];% \\*
\verb%   long int ssn;% \\*
\verb%   int salary;% \\*
\verb%} ; % 
\end{flushleft}

When you write this in your program, it defines a new type, the type
{\tt struct employee}.  A variable of type {\tt struct employee} has
three {\em members}\/:  {\tt name}, an \ano{50}{\Char}, {\tt ssn}, a
\longint, and {\tt salary}, an \int.  

{\tt employee} is called the {\em tag}\/ of the structure.

\subsection{Creating {\tt struct} Variables}

To declare a variables of type {\tt struct employee}, just write:

\begin{flushleft}
\verb% struct employee smithers, marketing[12], *the_employee;% 
\end{flushleft}

{\tt smithers} is a {\tt struct employee}; {\tt marketing} is an array
of twelve of these structures, and {\tt the\_employee} is a pointer to
one of these structures.

\subsection{The {\tt .} Operator}

Suppose {\tt smithers} is a {\tt struct employee} variable, as above.
Suppose we want to store the information that Smithers' social security
number is {\tt 314-15-9265}.  Here's how we would do that:

\begin{flushleft}
\verb% smithers.ssn = 314159265; %
\end{flushleft}

The {\tt .} operator is the {\em structure member}\/ operator.  Its left
operand must be a structure of some kind; its right operand must be the
name of one of the members of that structure.  The entire expression
refers to the specified member of the specified structure.  So
similarly,

\begin{flushleft}
\verb% if ( smithers.salary > 10000 ) %  \\*
\verb%   ... % 
\end{flushleft}

\noindent asks if Smithers' salary exceeds ten thousand dollars. 

Similarly, let's suppose the array {\tt marketing}, which is an array of
12 {\tt struct employee}s, has been initialized with the names,
salaries, and social security numbers of the members of the marketing
department.  Here's how we'd compute the average salary in the marketing
department:  

\begin{flushleft}
\verb% long int totalsalary = 0;% \\*
\verb% % \\*
\verb% for (i=0; i<12; i++) % \\*
\verb%   totalsalary += marketing[i].salary;% \\*
\verb! printf("Average salary is %f.\n", totalsalary/12.0);! \\*
\end{flushleft}

{\tt marketing} is an array of {\tt struct employee}s, so {\tt
marketing[i]} is a {\tt struct employee}, and {\tt marketing[i].salary}
is the salary member of that {\tt struct employee}. 

As a final example, this is some code which prints out the name and
social security number of the highest-paid member of the marketing
department: 

\begin{flushleft}
\verb! int highest_salary = 0;! \\*
\verb! int i, highest_salary_index; ! \\*
\verb! ! \\*
\verb! for (i=0; i<12; i++) ! \\*
\verb!   if (marketing[i].salary > highest_salary) {! \\*
\verb!     highest_salary = marketing[i].salary;! \\*
\verb!     highest_salary_index = i;! \\*
\verb!   }! \\*
\verb! ! \\*
\verb! printf("The highest-paid employee is %s (%d).\n", ! \\*
\verb!        marketing[highest_salary_index].name, ! \\*
\verb!        marketing[highest_salary_index].ssn); ! \\*
\end{flushleft}

\section{The Linked List}

Suppose we want to write a program which reads an input file, counts the
number of occurrences of each different word in the input, and then
prints out a report of the form ``{\tt There were 27 occurrences of the
word `I', 53 occurrences of the word `the'}\ldots''. You might want to
have an array that will hold words and another that will hold counts,
but that doesn't work, because the arrays have a fixed size, say 500
elements each, and if the input contains 501 different words, you're
stumped.  We need to have a data structure that can grow arbitrarily
large if we need it to. 

The simplest example of such a data structure is the {\em linked
list}\/.  A linked list is a collection of {\em nodes} in some order;
each node contains some information about how to find the `next' node,
and perhaps some auxiliary information as well.  The last node has some
kind of marker that says it's last.

To keep track of a linked list, all we need to do is remember where the
first node is.  Then that first node contains information (called a {\em
link}\/) that tells us where the second node is; the second node has a
link to tell us where the third node is, and soforth. Clearly the list
can be any length.

\subsection{The Nodes are Structs}

Here's a simple way to implement a linked list: it keeps track of as
many numbers as we like:

\begin{flushleft}
\verb% struct listnode {% \\*
\verb%   int data; % \\*
\verb%   struct listnode *next; % \\*
\verb%} ; % 
\end{flushleft}

This {\tt struct} has two members:  {\tt data}, which actually holds a
number, and {\tt next}, which points to the next node in the list.  

\subsection{Adding a New Node to a List}

Suppose the address of the first node in the list is stored in the
variable {\tt firstnode}, whose type is {\tt struct listnode *}.
Suppose also we discover that we need to remember one more number, {\tt
newnumber}.  How can we add that number onto the list of numbers in a
list of these nodes?

\begin{flushleft}
\verb% struct listnode *temp;% \\* 
\verb% % \\* 
\verb% temp = malloc(sizeof(struct listnode));% \\* 
\verb% % \\* 
\verb% (*temp).next = firstnode;% \\* 
\verb% (*temp).data = newnumber;% \\* 
\verb% firstnode = temp;% \\* 
\end{flushleft}

First we create enough space for a new list node, with {\tt malloc};
then we link the node to the rest of the list by setting its {\tt next}
pointer to point to the old first node.  Then we store the number we
want to remember into the data member of the new node.  Then we remember
that the new node is now first, with {\tt firstnode = temp}.  We can do
this as many times as we want, until the memory runs out.

\subsection{Getting the Data Back Out of the List}

Now suppose we want to average the numbers stored in each node.  It's
easy:

\begin{flushleft}
\verb% struct listnode *current;%  \\*
\verb% int average;%  \\*
\verb% long int sum = 0L;%  \\*
\verb% int nodecount = 0;%  \\*
\verb% %  \\*
\verb% for (current = firstnode; % \\*
\verb%      current != NULL; %  \\*
\verb%      current = (*current).next) {%  \\*
\verb%   sum += (*current).data; %  \\*
\verb%   nodecount += 1;%  \\*
\verb% }%  \\*
\verb% %  \\*
\verb% average = sum / nodecount; /* round down */ %  \\*
\end{flushleft}

Here we've assumed that the {\tt next} pointer of the last node in the
list is {\tt NULL}.  This makes perfect sense, and we can be sure that
no other node but the last has {\tt NULL} for its {\tt next} pointer,
because all the other {\tt next} pointers point to other nodes, whose
addresses are not {\tt NULL}. 

I hope it's obvious how this is all relevant to the problem of managing
an arbitrarily large stack for your calculator.  

\subsection{The {\tt ->} Operator}

The operation of accessing the member of a structure via a pointer to
the structure is common.  We've used the construction {\tt (*foo).bar}
several times already.  C lets you use a short cut.

In all circumstances, the expression {\tt foo->bar} is identical with
{\tt (*foo).bar}.  {\tt foo} must be a pointer to a structure, and {\tt
bar} must be the name of one of the members of that structure. 

\subsection{Other Operations on Structures}

The things you can do with a structure are limited.  You can find its
address with {\tt\&}, and you can access its members with {\tt .} .  

Since the ANSI standard, some things which were Too Much Work are no
longer considered to be too Much Work:

You can pass a structure as an argument to a function (in which case the
function gets a copy of the structure, the same way it does when you
pass an \int) and you can return a structure value as a return value
from a function. 

You can compare two structures of the same type for equality or
inequality with {\tt ==} and {\tt !=}; two structures are `equal' if
each pair of corresponding members are equal.  

You can assign a structure value to a variable of the right structure
type with {\tt =}; the members are copied one at a time.  

You can't do anything else with a structure.

\section{Global Variables and Type Declarations}

Normally, you declare a variable inside a function.  Then the variable
is created when the function is called, is destroyed when the function
returns, and the name of the variable is known only within the function.

This holds for structure type definitions also; if you define a
structure type inside a function, the type and its name are only known
within that function.  That's not too useful; we would like the
structure type definition to be visible everywhere.

If you write a declaration or type definition outside of all the
functions, it is called a {\em global}\/ variable or definition.  It
becomes visible to every function that follows the declaration or
definition in the entire file.  We have been doing this with function
prototypes all along; in fact you can put a function prototype inside
another function, and the information that the prototype communicates is
only known locally, and now outside the function in which the prototype
appears.  Doing this is rarely useful.

If you write {\tt int foo;} at the top of your program, outside all the
functions, then every function in your entire file will have access to
the variable {\tt foo}.  If one function changes the value of {\tt foo},
the others can see the change; the name {\tt foo} refers to the same
variable, no matter which function is using it.  Furthermore, unlike a
local variable, which is created when the function that owns it is
called and destroyed again when the function returns, a global variable
is created when your program is run and is not destroyed until the
program completes.

A global variable is an abnormal communication mechanism; functions can
use it to communicate values back and forth even though it may not be
obvious that they're doing so; for this reason, you should avoid using
global variables in your program.  There are occasions when it is
appropriate: when some large piece of information is really global, and
most of the functions make frequent references to it.\footnote{For
example, a program might read a {\em configuration file} that specifies
certain details about how the program should work; the information in
the configuration file is stored in a large structure.  Nearly every
function has to consult one or another member of this structure, so it
might be appropriate to make the structure variable global.}

Function prototypes, structure type definitions, manifest constants, and
other information that doesn't change, on the other hand, are
appropriate things to make global.

\section{A Program to Count Words}

At the end of these notes you'll find the complete code for the example
we chipped away at in class: It reads `words' from the standard input,
counts the number of occurrences of each word, and prints a report when
it's done.  It uses a linked list to keep track of what words it seen
and how many times it's seen each word.  There is no arbitrary upper
bound on the number of different words it can handle.

{ \input count.tex }

\subsection{Notes on the Code}

What follows here are notes on the code; the numbers in the margins are
source code line numbers.  I hope you will read the code carefully and
try to understand what each part is doing.  The notes only explain the
fine points; most of the details of how the program works are not
explained in the notes.  

\begin{itemize}
\item[16] {\tt <ctype.h>} is here to provide the definition of {\tt
isspace}, which we use later, in function  {\tt getword}.  
\item[18] {\tt <malloc.h>} declares {\tt malloc}, {\tt free}, and other
related functions. 
\item[27] This is the defintion of the ystructure type we'll use for a
node in our linked list.  Each node has three parts:  A pointer to the
word it represent, a count of the number of occurrences of that word
seen so far, and a pointer to the next node.   
\item[41]  We could describe a list of $n$ nodes this way:  It has a
head node, which has some data associated with it, and which has a
pointer to a list of $n-1$ elements.  So a list of 2 nodes has a head
node, which has some data and which has a pointer to a list of 1 node.
A list of 1 node has a head node, which has some data and which has a
pointer to a list of 0 nodes.  But this latter pointer is actually {\tt
NULL}, because the head node is the last node in the list, so we've just
suggested that it might be wise to consider the {\tt NULL} pointer as a
`pointer to a list with 0 nodes'.

When we initialize {\tt list}, we initialize it with a `pointer to the
empty list', {\tt NULL}, and it turns out that the function {\tt count},
which manipulates lists, does the right thing if we pass it this
pointer.  

\item[45] {\tt count} might append a new node onto the head of the list,
and we need a way to apprise {\tt main} of that fact, so that it can
remember where its list starts.  We do this by returning a pointer to
the new head node from {\tt count} after we've attached it, and main
just stores this pointer in {\tt list}, the variable it was using to
hold a pointer to the first node in the list.

We wouldn't have to bother with this if {\tt count} just attached the
new node to the tail of the list instead of to the head, but that's
uaully harder to do, because the head of a list is usually easier to
find than the tail.  
\item[46] {\tt getword} allocated space for {\tt newword}; we passed
{\tt newword} to {\tt count}, which in turn made a copy of it to store
in the list, and so the copy we got back from {\tt getword} can be freed
now.  

Why have {\tt count} copy {\tt newword} at all?  If you're writing a
function like {\tt count} and you need to save some data that was passed
in, it's a good idea to make a copy because you never know when your
calling function might decide to destroy the original.

\item[51] Note that this code works even if the input contained no words
at all:  In that case, the body of the {\tt while} loop on lines 44--49
was never executed at all, and {\tt list} is still {\tt NULL}; we exit
the program without printing anything.  
\item[66] {\tt count} accepts two arguments:  a word and a pointer to
the first node in a list.  It searches the list for a node whose {\tt
word} member is {\tt word}; if it finds it, it performs its primary
function and increments the count in that node.  Otherwise, it
manufactures, initializes, and links in a new node.  In either case it
returns a pointer to the new head node of the list (which might be the
same as the old head node) to tell its caller whether or not the list
has gotten longer.
\item[71] Note that this code also works if the calling function passes
in {\tt NULL} for {\tt list}:  the condition on the {\tt for} loop fails
immediately and we proceed to like 77, where we begin the process of
adding a new node to the list.
\item[81] We used the {\tt exit} function here.  {\tt exit} completely
terminates the program when it is called; control returns to the
operating system.  Accordingly, {\tt exit} does not have a return value.
Contrast {\tt exit}, which terminates `normally', with {\tt abort},
which terminates `abnormally'.  {\tt exit} accepts one argument, which
is treated the same way a return value from {\tt main} is; in fact, the
effect of {\tt exit(n)} is identical to that of a {\tt return n;} from
{\tt main}.\footnote{It might be more accurate to say that a {\tt
return} from main is identical to a call to {\tt exit}, because
compilers frequently compile it as one.} Here we {\tt exit} with a
return status of {\tt 1}, signalling that something went wrong.
\item[90]  We copy {\tt word} here; see the note for line 46.
\item[92] If we're initializing a new node, we initialize its count to
1, because we've seen exactly one occurrence of the word that the node
represents. 
\item[100] {\tt create\_new\_node} is a separate function for the usual
reasons: 
\begin{enumerate}
\item It's functionally separate from the other functions.
\item It might one day be called from more than one place, since it
performs a generally useful function.
\item It might one day change and include more complicated node-building
apparatus such as initializations. 
\item If we ever again need a function that allocates and returns a list
node, we might be able to come steal this one.
\end{enumerate}
\item[103] My compiler complains about this line, because the return
value from {\tt malloc} is a \p2\Char, which is implicitly converted to
a \p2{{\tt struct wordcount\_node}} when we return it from the function.
Usually, if you're converting pointers in this way, it means you made a
mistake.  On some machines, certain pointer conversions will fail
altogether.\footnote{For example, consider a computer with a main
processor and an auxiliary processor which is very good a floating-point
arithmetic and which has its own auxiliary memory, optimized for storing
floating-point numbers. The compiler might very well decide to store
\int s into the main memory and \float s into 
the auxilliary memory; in that case you wouldn't be able to interconvert
\p2\Int\ and \p2\Float.  This is a somewhat contrived example.}  However,
the value returned from {\tt malloc} is an exception: The memory it
points to is guaranteed to be suitable for storing any variable
whatever, and the pointer is guaranteed to be freely convertible to any
other pointer type.  {\tt malloc} must take special pains to ensure that
this is the case.  The compiler complains only because it doesn't know
enough about {\tt malloc}.
\item[111] I wrote this function two years ago, and I've been re-using
it ever since. 
\item[113] We allocate a buffer of {\tt MAXWORDLEN} characters to hold
the input word, but this function never checks to make sure it isn't
writing past the end of the buffer.  If the input contains an extremely
long word, {\tt getword} will happily write past the end of {\tt buf}.
This is a serious error.
\item[117] {\tt isspace} is a function whose argument is
a character; it returns {\bf true} if the argument is a white space
character and {\bf false} otherwise.  White space characters include the
blank, tab, and newline.  {\tt isspace} was declared in {\tt <ctype.h>}.
(Line 16.)
\item[120] We haven't seen {\tt ungetc} yet; it's like the opposite of
{\tt getc}: it `unreads' a character. After you {\tt ungetc} a character
onto a stream, the next attempt to read the stream will proceed as if
you had never read that character; the first character returned by the
reading function will be the character you `unread'. {\tt ungetc} is
limited: You can't un-get more than one character at a time, and you
can't un-get a second character until you've read the first one back.
{\tt ungetc} is declared in {\tt <stdio.h>}.
\item[128] the loop on lines 123--124 reads characters from the standard
input into the buffer {\tt buf}\footnote{{\em buffer}\/ is a generic
word that describes a part of memory that input is being read into.},
and stops when it reads a space or hits {\tt EOF}.  Now suppose the last
word in the file was not followed by a space; say the last word is {\tt
visible}, and then after the {\tt e}, nothing.  {\tt getword} has 
read {\tt visible} into the buffer, and has just hit {\tt EOF}.  What do
we want it to do?  

We do not want {\tt getword} to return {\tt NULL} the instant it sees
{\tt EOF}, because if it did our caller would never find out about {\tt
visible}.  So instead, we return {\tt NULL} only if there is nothing in
the buffer; that is, when {\tt i == 0}.  In the case of {\tt visible},
{\tt i} is 7, so we duplicate {\tt visible} and return the copy; then
the {\em next}\/ time {\tt getwords} is called, it hits {\tt EOF} right
away, without reading any characters into the buffer {\tt buf}, and so
{\tt i} is 0 and it returns {\tt NULL}.  
\end{itemize} 
