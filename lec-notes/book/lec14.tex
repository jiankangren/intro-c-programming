%
% CSE 110 Lecture Notes
%
% Entire contents are copyright 1992 by Mark-Jason Dominus.
% All rights reserved.  Unauthorized reproduction prohobited.
%

\section{Problem 2.2 ({\tt strrev}) from the Exam}

\begin{quotation}
\small
Write the function {\tt strrev}, whose argument is a string, and which
reverses its argument in place.  That means that if we do:
\begin{flushleft}
\verb% char word[] = "Foo"; % \\*
\verb! printf("%s\n", word); ! \\*
\verb% strrev(word);% \\*
\verb! printf("%s\n", word); ! \\*
\end{flushleft}

\noindent the output should be {\tt Foo}, followed by {\tt ooF}.
\end{quotation}

This was one of the hard problems on the exam.  Nobody actually turned
in a working solution, although several people came close.  First I'll
show a working solution, and then we'll discuss some alternative
solutions.

One thing a lot of people did was to write a program that actually
printed out the string backwards.  People who did this correctly got
half credit, but it wasn't what the function was supposed to be
about---the question asked for a function that would reverse a string
`in place'.  That means that we want to be able to tell the function
where our string is (presumably via a pointer), and the function will
find the data there and reverse it and leave the answer in memory in the
same place that the original word was.  No input or output was required.
This should have been clear from the example:  {\tt word} contains {\tt
Foo} when we print it out the first time; we run {\tt strrev}, and
nothing is printed out until the second {\tt printf}, which demonstrates
that {\tt word} now contains {\tt ooF}. 

When you're asking yourself how to reverse a string in place, you might
think to yourself, ``Well, I could copy the string backwards into some
auxiliary space, and then copy the reversed string back from the
auxiliary space to the original space.''  To do that you need to know
how to ask for extra space on the fly, and we didn't know how to do that
at the time we took the exam.

\subsection{A Solution}

The solution I was hoping for took some ingenuity to find: We have two
pointers, {\tt s} and {\tt e}. {\tt s} starts out pointing to the first
character in the string, and {\tt e} starts out pointing to the last
character in the string.  We swap the characters that {\tt s} and {\tt
e} are pointing to; that's easy.  Then we increment {\tt s} to point to
the second character, and decrement {\tt e} to point to the next-to-last
character.  We repeat, until both pointers are pointing to the middle
letters of the string; then we're done.

Here's the code.

\begin{verbatim}
void strrev(char *s)
{
  char *e;               
  int left=0;                        /* Count of characters left to swap */
  char temp;                         /* For swap */

  /* If length of string is less than 2, don't bother.
   * Note short-circuiting here---it's VERY IMPORTANT.
   * If s has length 0, so that s[0] == '\0', it's ESSENTIAL that the
   * computer does NOT go and try to look at s[1] anyway, because that's
   * off the end of the array. */
  if (s[0] == '\0' || s[1] == '\0')
    return;
  
  /* e starts at string end, s at beginning.  We march s forward
     and e backwards, simultaneously, swapping the characters that
     e and s point to.  Thus the first character swaps with the last, 
     the second with the next-to-last, and soforth. */

  /* First, point e at end of s and compute length of s: */

  for (e=s; *e != '\0'; e++)
    left++;
  /* e now points to NUL character at end of s. left is the number of 
     characters we have to swap. */

  e--;        
  /* e now points at last character in s. */


  while (left > 1) {            /* If there's only one character left, 
                                   it's in the middle of the string anyway,
                                   so stop. */
    

    temp = *s; *s = *e; *e = temp; /* Swap characters at beginning and end */
    s++; e--;                      /* Move towards middle of string  */
    left -= 2;                     /* two fewer characters to swap. */
  }

}
\end{verbatim}


There's nothing new here, so it was fair game for the exam.

\subsection{A Non-Solution}

Some people did think of the auxiliary-space method, and handed in
something like this:

\begin{flushleft}
\verb%void strrev(char *s)%
\\* \verb%{%
\\* \verb%  int i, len = strlen(s);%
\\* \verb%  char aux[len+1];              /*%\mbox{ You can't do this }\verb%*/%
\end{flushleft}

\begin{flushleft}
\verb%  aux[len] = '\0';%
\end{flushleft}

\begin{flushleft}
\verb%  for (i=0; i<len; i++)%
\\* \verb%    aux[i] = s[len-1-i];        /*%\mbox{ Copy string backwards }\verb%*/%
\end{flushleft}

\begin{flushleft}
\verb%  for (i=0; i<len; i++)%
\\* \verb%    s[i] = aux[i];              /*%\mbox{ Copy string forwards }\verb%*/%
\end{flushleft}

\begin{flushleft}
\verb%  return;%
\\* \verb%}%
\end{flushleft}


This is ingenious, but it doesn't work.  The catch is that an array's
size must be determined at the time the program is
compiled.\footnote{Some compilers do allow variable-sized arrays, as a
non-portable extension.} Nevertheless, people who did this correctly
(not counting the illegal array declaration itself) got three-quarters
credit.

\subsection{Another Solution with {\tt strdup}}

Here's another auxiliary-space solution---this one does work.
Unfortunately, nobody could have turned this in because we hadn't
covered {\tt strdup} by the time we had the exam:

%
% CSE 110 Lecture Notes
%
% Entire contents are copyright 1992 by Mark-Jason Dominus.
% All rights reserved.  Unauthorized reproduction prohobited.
%
\begin{flushleft}
\verb%void strrev(char *s)%
\\* \verb%{%
\\* \verb%  int i, len = strlen(s);%
\\* \verb%  char *aux;%
\end{flushleft}

\begin{flushleft}
\verb%  aux = strdup(s);%
\end{flushleft}

\begin{flushleft}
\verb%  ... /*%\mbox{ Same as above }\verb%*/%
\\* \verb%}%
\end{flushleft}


{\tt strdup}'s argument is a string.  {\tt strdup} does this:

\begin{enumerate}
\item It looks at the string,
\item It finds out (probably with {\tt strlen}) how much space is
necessary to store the string, 
(one byte for each character in the string, and an extra byte for the
{\tt NUL} character that terminates it),
\item It  somehow finds and reserves
enough space for a copy of the string,
\item It copies the string into the memory it's reserved, and
\item It returns a pointer to the first character in the new copy of the
string.
\end{enumerate}

If {\tt strdup} fails, for example because it can't reserve enough free
space for a copy of its argument, it returns the {\tt NULL} pointer.

When we say that {\tt strdup} `reserves' space, we mean that the space
won't be used for something else until we tell the computer that it's
all right to do so.  We can be sure that future calls to {\tt strdup}
will find other space, and that variables we declare will be allocated
out of space other than that reserved by {\tt strdup}.

\subsection{{\tt free}}

The space reserved by {\tt strdup} stays reserved until the program
terminates or until we explicitly make the space available for re-use.
This is called {\em freeing}\/ the space; we do it with the {\tt free}
function.  If {\tt p} is a \p2\Char\ which points to space reserved by
{\tt strdup}, the call {\tt free(p)} frees that space; a variable might
get put there later, or a future call to {\tt strdup} might copy new
data there and return a new pointer to it.  

\section{Comparing Characters and Strings}

The comparison operators {\tt <}, {\tt <=}, {\tt >}, and {\tt >=} all
work on characters as well as on numbers, in a reasonable way. 

\subsection{Comparing Single Characters}

It's definitely true that:

\begin{verbatim}
        '0' < '1' < ... < '8' < '9'
        'A' < 'B' < ... < 'Y' < 'Z'
        'a' < 'b' < ... < 'y' < 'z'
\end{verbatim}

The relative ordering of these three classes, however, is
implementation-dependent.  Depending on what kind of machine you are
using, it might be true either that\mbox {\tt 'A' < 'a'} or that
\mbox{\tt 'a' < 'A'}.  On the machines we use, it's the case that {\tt
'9' < 'A' < 'Z' < 'a'}.

The {\tt NUL} character is always less than any printable character.  A
printable character is one that makes a space or a mark on the screen.

We can use this character ordering to write a function that compares
strings alphabetically. 

\subsection{Comparing Strings}

We'll write a function which accepts two strings as arguments and
returns some kind of answer to say which string comes earlier in an
alphabetical ordering.  We'll do something reasonable for nonalphabetic
strings.  

    We'll say that a string is {\em less}\/ than another string if it
comes before that string in the dictionary.  For example: {\tt bar} is
less than {\tt foo}; {\tt food} is less than {\tt fool}; {\tt fool } is
less than {\tt foolish}.  Please note that `less' does not mean
`shorter'.  {\tt foolish} is less than {\tt zebra}, but it is not
shorter than {\tt zebra}.

    Our function will examine the first character of each argument.  If
the characters are different, we know right away which string is least:
The one whose first character is least.  We can return an answer right
away in this case.  We'll return {\tt -1} if the first argument is less
than the second, and {\tt 1} if the first argument is greater than the
second.

    If the first two characters are equal, we'll go look at the second
two characters, and soforth.

    We keep doing this until either we find two characters that don't
match (in which case we return {\tt 1} or {\tt -1} as above) or until we
reach the end of one or both strings.

    If we reach the end of both strings at once, then they were
identical, and we return {\tt 0}. 

    Otherwise, one of the strings is a {\em prefix}\/ of the other,
which means that it is just the same as the other string, only it is
missing some characters off the end; for example, {\tt foo} is a prefix
of {\tt foolish}.  In this case the shorter string is least, and we
should return {\tt 1} or {\tt -1} as above.

That said, here's the code we wrote in class:

\input strcmp1.tex

This code works.  It's functionally equivalent to a standard library
function called {\tt strcmp}, which operates in the same way.  {\tt
strcmp} returns negative, zero, or positive, depending on whether its
first argument is less than, equal to, or greater than its second
argument.  ({\tt strcmp} doesn't always return $-1$, 0, or 1.)  To use
{\tt strcmp}, you must include {\tt <string.h>}.

\subsection{Case-Insensitive Comparison}

Our {\tt strcmp} function does a thing that we might find peculiar: It
says that {\tt Snider} is less than {\tt food}, because, in Turbo-C++,
capital letters are always less than lowercase letters.  We say that
{\tt strcmp} is {\em case-sensitive}\/, because it treats the words {\tt
Snider} and {\tt snider} differently.  We might wish to erase this
distinction and make {\tt Snider} greater than {\tt food}, because {\tt
s}, capital or not, comes after {\tt f} in the dictionary.

To do this, we can use the standard function {\tt tolower}.  {\tt
tolower} accepts one argument, which is a character.  If the character
is not an uppercase letter, {\tt tolower} returns the character it was
passed; if the argument was an uppercase letter, {\tt tolower} returns
the lowercase version of that letter.  For example, {\tt tolower('X')}
is {\tt 'x'}; {\tt tolower('y')} is {\tt 'y'}; {\tt tolower('2')} is
{\tt '2'}.

What we'll do is convert the characters in our strings to lowercase
before we compare them; then the {\tt S} in {\tt Snider} will behave as
though it were a lowercase S.  

Replace the lines

\begin{flushleft}
\verb%    else if ( *s1 < *s2 )% \\*
\verb%      return -1;% \\*
\verb%    else if ( *s1 > *s2 )% \\*
\verb%      return 1;	% \\*
\end{flushleft}

with

\begin{flushleft}
\verb%    else if ( tolower(*s1) < tolower(*s2) )% \\*
\verb%      return -1;% \\*
\verb%    else if ( tolower(*s1) > tolower(*s2) )% \\*
\verb%      return 1;	% \\*
\end{flushleft}

\noindent.  The function we get is called {\tt strcasecmp} on {\sc unix}
systems, but it seems to be called {\tt stricmp} or {\tt strcmpi} under
Turbo-C++.  

{\tt tolower}, and a collection of similar functions, such as {\tt
toupper}, are declared in {\tt <ctype.h>}.

