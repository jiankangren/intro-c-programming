%
% CSE 110 Lecture Notes
%
% Entire contents are copyright 1992 by Mark-Jason Dominus.
% All rights reserved.  Unauthorized reproduction prohobited.
%

\section{More about Object Contexts and Value Contexts}


If you look back at the notes for section \ref{contexts}, you'll recall
the following facts:

\begin{itemize}
\item The word {\em object}\/ is short for the phrase ``a particular
part of the computer's memory.''
\item Some expressions, such as {\tt x},  refer to objects.  (For
example, {\tt x} refers to the variable {\tt x}, which is a particular
part of the computer's memory and is therefore an object.)  Most
expressions, such as {\tt 4} or {\tt 3*x + y}, do not refer to objects.
\item An expression which refers to  an object is
called an {\em lvalue expression}\/.\footnote{The `l' in `lvalue' stands
for `left', because the thing on the {\bf left} side of an assignment
must be an lvalue expression.}
\item A {\em context}\/ is a place in your program where you are allowed
to put an expression.
\item When the computer encounters a certain expression in one context,
it may behave differently than when it encounters the same expression in
another context.
\item There are only two kinds of contexts in C: {\em object contexts}\/
and {\em value contexts}\/.  What the computer does with an expression
depends only on the expression itself and on whether that expression is
in an object context or a value context.
\item Most contexts are value contexts.  
\item The only contexts that are not value contexts are:
\begin{enumerate}
\item The left-hand-side of an assignment expression, and
\item The operand of the {\tt\&} operator.
\end{enumerate}
These two contexts are object contexts, not value contexts.
\item Any expression may appear in a value context.
\item An expression in a value context is evaluated to produce a value.
\item Only an lvalue expression may appear in an object context.   
\item An lvalue expression in an object context is not evaluated to
produce a value.  Instead, the computer figures out what object it
refers to, and then
\begin{enumerate}
\item If the lvalue expression was on the left-hand-side of an assignment, the
computer uses the object it represents as a place to store the value it
computed on the right-hand-side of the assignment, or
\item If the lvalue expression was the operand of the {\tt\&} operator, then the
computer gets the address of the object that the lvalue expression represents.
\end{enumerate}
\end{itemize}

\subsection{ A Pointer Value Points to an Object}

If {\em p}\/ is a valid expression whose type is \p2{something}, then
{\em p}\/ points to an object of type \brac{something}. Then {\tt *}{\em
p}\/ is an expression which represents the object that {\em p}\/ is
pointing to.  Since {\tt *}{\em p}\/ therefore refers to an object, {\tt
*}{\em p}\/ is an lvalue expression, and may appear on the left-hand
side of an assignment statement, as in

\begin{flushleft}
\verb% *n1 = temp ;  %
\end{flushleft}

\noindent .  {\tt temp} is in a value context, so it's evaluated, and
the computer gets the value stored in the variable {\tt temp}.  On the
other hand, {\tt *n1} is in an object context, and so once the computer
finds out what part of the blackboard {\tt *n1} represents, it does {\em
not} go get the value stored there.  Rather, it uses it as a place to
store the value of {\tt temp}.  On the other hand, in

\begin{flushleft}
\verb% temp = *n1 ; %
\end{flushleft}

\noindent the {\tt *n1} is in a value context;  once the computer
figures out what region of the blackboard {\tt *n1} represents, it goes
and gets the value stored there, and then it stores that value into the
region of the blackboard represented by {\tt temp}.  {\tt temp} is in an
object context, not a value context, so it is not evaluated and the
computer does never retrieves the value stored in {\tt temp}.

\subsection{ The operand of {\tt\&} is in an Object Context}

We have seen exactly two examples of object contexts so far.  One is the
left-hand side of an assignment.

The other is that in the expression {\tt \&{\em foo}\/}, {\tt {\em
foo}\/} is in an object context: It is not evaluated to produce a value,
and the value of the entire {\tt \&{\em foo}\/} expression has nothing
whatever to do with the value actually stored in {\tt {\em foo}\/}
itself.  Also, {\tt {\em foo}\/} must be an lvalue expression, because
{\tt\&} asks where in memory its operand is stored, so therefore its
operand must be an object, and only lvalue expression, by definition,
represent objects.

\subsection{So What?}

This is going to become frightfully important in section
\ref{array-in-value-context}.  In the meantime, we'll do something else.

\section{Arrays}

Suppose we want to write a program which reads in a list of 100 numbers
from the user, and then prints them out in reverse order. To do that, we
must remember all 100 numbers until the end; there's no way around it.
We could write a program with a very long declaration:

\begin{flushleft}
\verb% int main(void) % \\*
\verb% { %  \\*
\verb%   int n0, n1, n2, n3, n4, n5, n6, n7, n8, n9; % \\*
\verb%   . . . % \\*
\verb%   int n90, n91, n92, n93, n94, n95, n96, n97, n98, n99; % \\*
\verb% % \\*
\verb%   . . . % \\*
\verb% } %
\end{flushleft}

\noindent and then use 100 {\tt scanf}'s to read them in and 100 {\tt
printf}'s to write them out backwards, but of course there has to be a
better way.  

\subsection{A Better Way}

If we write

\begin{flushleft}
\verb% int n[100]; %
\end{flushleft}

\noindent instead, the compiler creates an {\em array}\/ of 100 \int s for
us.  That means that it finds enough space for 100 \int\ variables,
reserves the space, and arranges that {\tt n} refers to that space.  The
space is all contiguous, meaning that if an \int\ is 2 bytes long, the
compiler will find 200 bytes of space all in the same place---it won't
find us 37 bytes here and 22 bytes there and 41 bytes somewhere else.  

We can ask for these 100 \int s individually: The expression {\tt n[0]}
refers to the first one, and {\tt n[1]} to the second one, and soforth,
up to the last one which is {\tt n[99]}.  These 100 \int\ variables are
called the {\em elements}\/ of the array; each element has an {\em
index} which says where it is in the array.  The index of the first
element is 0, the index of the second is 1, and the index of the last
one is 99.  So here's our program:

\subsection{Program to Read 100 Integers from the User and Write them out in Reverse Order}

\input rev.tex

\subsection{Notes on the Program}


In the first loop, {\tt i} starts at 0 and goes up by ones until it gets
to 99, the index number of the last element in the array {\tt n}.  Each
time through the loop, the computer figures out what {\tt n[i]} is (it's
the {\tt i}'th element of the array {\tt n}), but, because the {\tt
n[i]} is in an object context, not a value context (it's the operand of
the {\tt\&} operator), the computer does not go and get the value of
{\tt n[i]}.  Instead, it computes the address of {\tt n[i]}, and passes
that address to {\tt scanf}, which reads an \int\ value from the user
and stores the value in the variable {\tt n[i]}.  {\tt scanf} could
change the value of {\tt n[i]} because it knew where on the blackboard
{\tt n[i]} was.  {\tt scanf} knew where on the blackboard {\tt n[i]} was
because we passed it the address of {\tt n[i]}.

The first number the user enters gets stored in {\tt n[0]}, and the
second gets stored in {\tt n[1]}, and soforth, until finally the user
enters the hundredth number, which is stored in {\tt n[99]}.  

Then the computer executes the second loop.  {\tt i} starts at 99, the
index number of the last element of the array {\tt n}, and counts
backwards until it gets to 0.  After 0, the {\tt for} loop is done. Each
time through the loop, the computer figures out what {\tt n[i]} is (it's
the {\tt i}'th element of the array {\tt n}), and then, because the {\tt
n[i]} is in a value context, the computer gets the value of {\tt n[i]}
and passes that value to {\tt printf} as an argument.  {\tt printf} then
prints out the value of {\tt n[i]}.

Since {\tt i} starts at 99, the first value the computer prints out is
that of {\tt n[99]}, which was the last number the user entered.  Then
the computer decrements {\tt i}, runs through the loop again, and prints
out the value of {\tt n[98]}, whose value is the next-to-last number the
user entered.  The computer keeps running backwards through the array,
until finally {\tt i} is 0; then the computer prints out the value of
{\tt n[0]}, which is the first number the user entered.  Then the
computer decrements {\tt i} to $-1$, exits the loop, and quits the
program.

\subsection{Initializing Arrays}

When you declare an ordinary variable, you can initialize it thus:

\begin{flushleft}
\verb% int n = 57; %
\end{flushleft}

This means that {\tt n} starts out with the value 57; the 57 here is
called an {\em initializer}.  We can do a similar thing with an array,
only we have to specify more than one initializer: If we write

\begin{flushleft}
\verb% int n[5] = { 1, 3, 9, 27, 81} ; %
\end{flushleft}

\noindent then
{\tt n[0]} starts out with the value 1,
{\tt n[1]} starts out with the value 3,
{\tt n[2]} starts out with the value 9,
{\tt n[3]} starts out with the value 27, and
{\tt n[4]} starts out with the value 81.  

If you make the initializer list too short for the array, as in

\begin{flushleft}
\verb% int n[5] = { 1, 3 }; %
\end{flushleft}

\noindent the computer initializes the elements of the array until it
runs out of initializers, and initializes the rest of the elements of
the array with zero values.  So the example above is the same as if we
had written {\tt int n[5] = \{ 1, 3, 0, 0, 0\};}.

If you make the initializer list too long, the compiler will complain.

If you include an initializer, you can omit the array size:

\begin{flushleft}
\verb% int n[] = { 1, 3, 5}; %
\end{flushleft}

The compiler counts the number of initializers and makes an array that
is just big enough to hold all of them.  In this case it makes an array
with 3 elements because you specified 3 initializers.

\subsection{Character Arrays}

Of course you can have an array of variables of any type, including \chr
s.  Here's a program to print someone's name out backwards:

\input jeano.tex

Note that we use {\tt\%c} to print out a \chr\  with {\tt printf}.  Also
note we had to hard-wire the length of the character array into the
loop.  We'll learn how to get around that tomorrow.

It's a pain having to type all those character constants, and C gives us
a shortcut:  We're allowed to initialize a character array this way:
Instead of writing

\begin{flushleft}
\verb%char name[] = { 'J', 'e', 'a', 'n', ' ', 'O', 'g', 'r', 'i', 'n', 'z' };%
\end{flushleft}

\noindent we can do this:

\begin{flushleft}
\verb%char name[] = "Jean Ogrinz";%
\end{flushleft}

\noindent .  The shorthand {\tt "Jean Ogrinz"} denotes a special array of
characters, called a {\em string of characters}\/ or usually just a {\em
string}\/.  Strings are how we handle things like peoples' names in C.
We'll see a lot of them in the rest of the course.  

In case you hadn't made the connection yet, the first argument to {\tt
printf} is always a string.

