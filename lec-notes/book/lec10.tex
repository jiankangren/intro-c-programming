%
% CSE 110 Lecture Notes
%
% Entire contents are copyright 1992 by Mark-Jason Dominus.
% All rights reserved.  Unauthorized reproduction prohobited.
%
\section{Miscellaneous Details About Arrays}

Here are some fine points of array use that we didn't talk about before.

\subsection{Out-of-Bounds Array References}

Question: {\tt arr} has only 23 elements.  In this code we're trying to
store the value 5 into the 120th element.  What happens?

\begin{flushleft}
\verb% . . . % \\*
\verb% int arr[23];% \\*
\verb% % \\*
\verb% arr[119] = 5;% 
\end{flushleft}

Answer: You get undefined behavior, so the compiler might choose to
teleport elephants into the room, or to generate an executable program,
which, when run, will teleport elephants into the room.  Most likely,
however, is that the compiler will say nothing at all, and will generate
an executable program that will assign the value 5 to the place in
memory where {\tt arr[119]} {\em would}\/ have been if {\tt arr} had had
that many elements.  However, {\tt arr} doesn't extend that far, and
chances are that the compiler put some other variable at that spot
instead, so that the {\tt 5} will obliterate some other piece of data.

You might ask why the compiler doesn't catch this for you, and the
answer is that catching out-of-bounds array references at the time the
program is compiled is impossible.\footnote{One can prove mathematically
that there is {\em no}\/ algorithm which, given the source code for a C
program, can determine whether or not that C program will generate an
out-of-bounds array reference when it is run.}

You can catch out-of-bounds array references at the time the program is
run, so that when you run the program, it aborts in the middle rather
than doing a bogus array reference.  Some languages, such as Pascal, do
this automatically.  But to detect the bogus array reference the
computer must test the index value, every time it looks at any element
of any array, to make sure it is not too big or too small, and that will
make your programs slow.  The C compiler takes the point of view that if
you wanted to check to index value, and accept the corresponding
slowness, then you'd write code yourself that tests the index value, and
that if you didn't you wouldn't, and so washes its hands of the problem.

An out-of-bounds array reference can create a bug that is very difficult
to find.  If you store a value into an array element that isn't there,
you'll clobber some other variable's value, and you might not find out
about that until much later when you try to use the variable and
discover it is full of garbage.  So be careful.

\subsection{The Expression {\tt a[i]} is an Lvalue}

The expression {\tt a[i]} represents an object, namely the {\tt i}'th
variable in the array {\tt a}, and is therefore an lvalue, and may
appear on the left-hand-side of an assignment statement, or as the
operand of the {\tt\&} operator.  We've already seen array references in
both contexts.

\subsection{{\tt[$\cdots$]} has High Precedence}

The {\tt[{$\cdots$}]} operator has higher precedence than {\em
everything}\/ else, so that {\tt\&a[i]} means {\tt\&(a[i])} and not
{\tt(\&a)[i]}. 

\subsection{More About Strings}

Suppose we have an array, and we want to perform some operation on every
element of the array.  We'd write a loop, of course; and loop over the
elements in the array, performing the operation on the first one, and
then on the second one, and soforth.

The question:  How do we know when to stop?

Either, we have to remember how long the array was, or, we have to
arrange that the last element of the array contains a special {\em
sentinel}\/ value that we can look for so that when we see the sentinel,
we know we're done.  Both methods are widely used.  For strings,
however, we always use a the sentinel value method, and the C language
provides a little support for this.

When you write something like

\begin{flushleft}
\verb% char name[] = "Jean Ogrinz"; %  
\end{flushleft}

\noindent the compiler creates an array of {\em twelve}\/ characters,
not eleven.  It initializes the first element of the array with the
letter {\tt J}, the second with {\tt e}, and soforth, and it initializes
the last element with a special sentinel value: the {\em NUL character}.

The C library provides many functions that operate on strings, and every
one of them assumes that the strings you pass to it as arguments will
end with the NUL character; that's how these functions know where to
stop.  For example, there's a function {\tt strlen} which takes a string
as an argument and returns the length of the string, and it does this by
counting the characters in the string one at a time until it sees the
NUL character.

So when I said a string was an array of characters, I fibbed a little.
A string is an array of characters that is terminated with a NUL
character.

Every character is represented internally by some integer between 0 and
255; for example, on typical machines, the character {\tt A} is
represented with the number 65.  The details about what number
represents what character are None of Your Business, with one exception:
The NUL character is always, always, always the character which is
represented by the number 0.

If you need to write the NUL character in your C program, you can write
{\tt '\verb+\+0'}.  

\section{Declarations}

The C declaration syntax was designed to be intuitive, but isn't.  

The declaration 


\begin{flushleft}
\verb! int i, *pi, ai[5], ii; !
\end{flushleft}

\noindent declares four things:  {\tt i}, an \int; {\tt pi}, a \p2\Int;
{\tt ai}, an \ano 5\Int; and {\tt ii}, another \int.

The way to remember this is that if you were to take any one of
the expressions from the declaration and actually write it in your
program, you'd have an \int.  So:  
\begin{itemize}
\item {\tt i} is an \int. 
\item  {\tt *pi} is an
\int, and therefore {\tt pi} itself is a \p2\Int. 
\item {\tt ai[5]} is an
\int\ (or would be if it weren't out-of-bounds; in any case, {\tt ai[{\rm
something}]} is an \int.), and therefore {\tt ai} itself is an \ao\Int)
\item {\tt ii} is an \int. 
\end{itemize}

This interpretation may come in handy when you're trying to understand a
more complicated declaration, such as

\begin{flushleft}
\verb! char *argv[]; !
\end{flushleft}

\noindent which says that {\tt *argv[]} is a \chr, so therefore {\tt
argv[{\rm something}]} must be a \p2\Char, and {\tt argv} itself must be
an \ao{\ptr\Char}. 

C's declaration syntax has been widely criticized already, so let's try
to live with it.  We won't be seeing declarations much more complicated
than this in any case.

\section{Pointer Arithmetic}

Having just had a section on C's declaration syntax, surely one of its
worst features, we should try to recoup a little prestige and have a
section on one of C's very best features.  Here it is; it gets big type
because it is so important:

\subsection{The Pointer Arithmetic Rule}

{\large If {\em p}\/ is a pointer which points to a certain element of
an array, then the value of the expression {\em p} {\tt + 1} is, by
definition, a pointer which points to the {\em next}\/ element of the
array.}

\subsection{An Example}

\begin{flushleft}
\verb% float arr[53], *p *q;% \\*
\verb% % \\*
\verb% p = &arr[4];% 
\end{flushleft}

\noindent {\tt p} is now pointing to the fifth element of the array {\tt
arr}.  (Remember that {\tt arr[4]} is the fifth element, because {\tt
arr[0]} is first and {\tt arr[1]} is second.)

\begin{flushleft}
\verb% q = p + 1; % 
\end{flushleft}

\noindent {\t q} is now pointing to the sixth element of the array {\tt
arr}.  (Remember that {\tt arr[5]} is the sixth element.)  Then

\begin{flushleft}
\verb% *q = 119; % \\*
\verb!  printf("%d\n", arr[5]);!
\end{flushleft}

\noindent will print {\tt 119}, regardless of what {\tt arr[5]} was
holding before.

\subsection{Consequences of the Pointer Arithmetic Rule}

Similarly, if {\em p}\/ is a pointer which points to a certain element of
an array, then the value of the expression {\em p} {\tt + 2} is, by
definition, a pointer which points to the element after the next one in
the array.  

Similarly, if {\em p}\/ is a pointer which points to a certain element of
an array, then the value of the expression {\em p} {\tt - 1} is, by
definition, a pointer which points to the previous element in the array.

Similarly, if {\tt p} points to the element {\tt a[0]} of the array {\tt
a}, and if {\tt n} is an \int, then the value of the expression {\tt p +
n} is a pointer to the element {\tt a[n]} of the same array {\tt a}.
Note what happens when the value of {\tt n} is 0:  {\tt p + 0} points to
the same place as {\tt p}, as it should.

Similarly, if {\tt p} points to an element of an array, and you do {\tt
p++;}, then afterwards, {\tt p} points to the next element of the array.

This simple convention is at least partly responsible for C's vast
popularity.  C makes pointer manipulations easy and flexible by using
this simple notation for talking about pointers to different parts of
the same array.

We'll see how to use this in the next few sections and in section
\ref{pointer-punch-line}, everything will finally fall
into place.

