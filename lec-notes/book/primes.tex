%
% CSE 110 Lecture Notes
%
% Entire contents are copyright 1992 by Mark-Jason Dominus.
% All rights reserved.  Unauthorized reproduction prohobited.
%

\begin{flushleft}
\verb%#include <stdio.h>%
\\* \verb%#include <math.h>               /*%\mbox{ For sqrt() }\verb%*/%
\end{flushleft}

\begin{flushleft}
\verb%int main(void)%
\\* \verb%{%
\\* \verb%  int n;%
\end{flushleft}

\begin{flushleft}
\verb%  for (;;) {%
\\* \verb%    printf("Enter a number, 0 to quit.  ");%
\\* \verb%    if (scanf("%\verb-%-\verb%d\n", &n) < 1)  /*%\mbox{ Bogus input }\verb%*/%
\\* \verb%      while (getchar() != '\n') /*%\mbox{ Discard input characters to end of line }\verb%*/%
\\* \verb%        /*%\mbox{ nothing }\verb%*/ ;%
\\* \verb%    else {%
\\* \verb%      if (n == 0)%
\\* \verb%        return 0;%
\\* \verb%      else if (is_prime(n))%
\\* \verb%        printf("That number is prime.\n");%
\\* \verb%      else%
\\* \verb%        printf("That number is not prime.\n");%
\\* \verb%    }%
\\* \verb%  }%
\\* \verb%}%
\end{flushleft}


\begin{flushleft}
\verb%int is_prime(int n)%
\\* \verb%{%
\\* \verb%  int divisor;%
\end{flushleft}

\begin{flushleft}
\verb%  for (divisor=2; divisor < sqrt(n); divisor+=1)%
\\* \verb%    if (n%\verb-%-\verb%divisor == 0)%
\\* \verb%      return 0;%
\end{flushleft}

\begin{flushleft}
\verb%  return 1;%
\\* \verb%}%
\end{flushleft}
