%
% CSE 110 Lecture Notes
%
% Entire contents are copyright 1992 by Mark-Jason Dominus.
% All rights reserved.  Unauthorized reproduction prohobited.
%
 
\documentstyle[fancyheadings]{article}

% \def\brac#1{{\tt <}#1{\tt >}}
\def\brac#1{$<$#1$>$}
\def\Int{{\tt int}}
\def\int{\brac{\Int}}
\def\int{\brac{\Int}}
\def\Shortint{{\tt short~int}}
\def\shortint{\brac{\Shortint}}
\def\Longint{{\tt long~int}}
\def\longint{\brac{\Longint}}
\def\Float{{\tt float}}
\def\float{\brac{\Float}}
\def\Double{{\tt double}}
\def\double{\brac{\Double}}
\def\Char{{\tt char}}
\def\chr{\brac{\Char}}
\def\Void{{\tt void}}
\def\void{\brac{\Void}}

\def\ptr#1{pointer~to #1}
\def\p2#1{\brac{\ptr#1}}
\def\Ano#1#2{array~of {#1}~#2s}
\def\ano#1#2{\brac{\Ano#1#2}}
\def\Ao#1{array~of #1}
\def\ao#1{\brac{\Ao#1}}

\def\np{{\tt NULL} pointer}

\def\breakhere{\mbox{$\otimes$}}
\parskip 8pt

% \def\baselinestretch{2}

\title{Lecture 12}
\author{CSE 110}
\date{20 July 1992}

\pagestyle{fancy}
\lhead{CSE 110 Lecture Notes}
\chead{Mark--Jason Dominus}
\rhead{\thepage}
\lfoot{Copyright \copyright 1992 Mark-Jason Dominus}
\cfoot{}
\rfoot{All rights reserved.}



\begin{document}

\maketitle

Today we'll cover the things that you still don't know that you need to
know in order to do the assignment.

\section{The {\tt NULL} Pointer}

For each pointer type, there is one special value of that type which
represents a pointer which does not point anywhere at all.  This value
is called the {\tt NULL}{\em pointer}\/.  The \np\ has one and only one
interesting property:  If {\em x}\/ is an object of type \brac{foo},
and therefore {\tt\&}{\em x}\/ is pointer to the object {\em x}\/, with
type \p2{foo}, then the value of {\tt\&}{\em x}\/ is {\em not} the \np.

The way you represent the \np\ in your code is by writing {\tt NULL}.
{\tt NULL} is actually a macro, but we won't discuss what it's a macro
for until later, because it's confusing.

Caution:  To not confuse the \np\ with the {\tt NUL} character.  The
\np\ is a pointer, which happens not to point anywhere.  The {\tt NUL}
character is a character, which is conventionally used to mark the end
of a string.  It's unfortunate that the names are so similar, but the
two things have nothing at all to do with one another.

Many functions which return pointer values return the \np\ to indicate
that something went wrong.  For example, you might have a function, {\tt
malloc}, which takes an argument {\tt n}, somehow finds {\tt n}
contiguous bytes of space on the backboard, reserves them, and returns a
pointer to the first byte.  {\tt malloc} could return the \np\ to
indicate that there weren't {\tt n} free bytes of space left on the
blackboard.  You could check the return value from {\tt malloc} to see
if you had run out of space:  
\begin{flushleft}
\verb% char *buf;% \\*
\verb% buf = malloc(1000);% \\*
\verb% if (buf == NULL) {% \\*
\verb%   printf("Out of memory.\n"); % \\*
\verb%   abort();% \\*
\verb% }% \\*
\verb% . . . %
\end{flushleft}

You know that if {\tt malloc} succeeded in finding the memory it was
looking for and returned a pointer to that memory, the value of that
pointer would be different from {\tt NULL}.  If the return value
compares equal to {\tt NULL}, we know that {\tt malloc} didn't succeed.

\section{Input Sources and Output Sinks}

When you call an input function such as {\tt scanf} or {\tt getchar},
the input comes from a place called the {\em standard input}.  Normally,
the standard input it attached to the keyboard so that {\tt scanf} and
{\tt getchar} read from the keyboard.  

If you run your program under MS-DOS, you can arrange to have the
standard input connected somewhere else, such as to a file.  For
example, to run the command `foo' with the standard input connected to a
file, you enter the command line {\tt foo < input\_file}.  When {\tt
foo} calls {\tt scanf} or {\tt getchar}, the data that {\tt scanf} or
{\tt getchar} reads comes from the file {\tt input\_file} instead of
from the keyboard.  This is awfully handy---it means your program
doesn't have to know anything about files in order to operate on files.

Similarly, {\tt printf} sends its output to the {\em standard output}\/,
which is normally attached to the screen so that {\tt printf}'s output
appears on the screen.  But you can redirect the standard output to a
file also: {\tt foo > output\_file} attaches {\tt foo}'s standard output
to the file {\tt output\_file}, and everything that {\tt printf} writes
will go into the file instead of to the screen.

Your program is completely unaware that anything different is
happening when it gets run with its standard input or output redirected,
and that's good, because it means you don't have to write any extra code
to handle it.

\subsection{A File Copy Utility}

This trivial program copies data from the standard input to the standard
output: ({\tt putchar}'s argument is a character, which it writes out on
the standard output; {\tt putchar(c)} is identical to {\tt printf("\%c",
c)}.)

\begin{flushleft}
\verb% #include <stdio.h>% \\*
\verb% % \\*
\verb% int main(void)% \\*
\verb% { % \\*
\verb%   int c;% \\*
\verb% % \\*
\verb%   while ((c = getchar()) != EOF)% \\*
\verb%     putchar(c);% \\*
\verb%   return 0;% \\*
\verb% }% \\*
\end{flushleft}

Let's call it {\tt scopy}, for `simple copy'.  If you run {\tt scopy},
it echoes everything you type back at you, and perhaps that doesn't seem
too useful.  But you can use it to copy files: {\tt scopy < source\_file
> destination\_file} copies the data in {\tt source\_file} into {\tt
destination\_file}.  It's also a program that types the contents of a
file on the screen:  {\tt scopy < file} reads from {\tt file} but writes
to the screen.

\section{Operating on a Particular File}

Sometimes, though, you want to read data from or write data to a
particular file.  For example, the compiler needs to write its output
into a file with a particular name.  How do we do that?

\subsection{Opening a File}

First you have to ask the operating system to {\em open} the file for
you.  Opening a file means that you notify the operating system that you
want to use the file.  The operating system checks to make sure that the
file you've named exists, and that you have permission from the file's
owner to read or write the file.  It sets up variables in its own
blackboard space that keep track of how much data you've read from the
file and what part of the file the next byte is supposed to come from.
Sometimes it does other things too---on {\sc UNIX} the operating system
arranges that if someone erases a file that you're reading, the file
doesn't actually go away until you're done.

The way you open a file is with the {\tt fopen} function.  {\tt fopen}
accepts two arguments:  The first a string containing the name of the
file you want to open, and the second is a string that says whether you
want to read the file or write it.  For example:

\begin{flushleft}
\verb% fopen("cse110.log", "r"); % 
\end{flushleft}

\noindent says to open the file {\tt cse110.log} for reading.  ({\tt "r"}
means `read'; if it were {\tt "w"} it would open the file for writing.)

{\tt fopen}'s return value is a peculiar type: It's a pointer to an
object called a {\tt FILE}, which contains some data from the file,
information about whether you've reached {\tt EOF} in the file, and
other things that the standard I/O functions need to know to work
properly. You need this {\tt FILE *} value later on to tell the computer
what file you want to read or write from.  In some sense, the {\tt FILE
*} `represents' the file.

{\tt FILE *} values are often called {\em streams}\/.

The prototype for {\tt fopen} looks like this:

\begin{flushleft}
\verb% FILE * fopen(char *filename, char *type); % \\*
\end{flushleft}

If {\tt fopen} can't open the file for some reason, it returns the \np.

\subsection{Reading from a File with {\tt getc}}

Once you've got the file open, you can operate on it with functions that
are very much like the ones you're used to.  For example, {\tt getc} is
a function which takes one argument, a {\tt FILE *}.  It is exactly like
{\tt getchar} except that it reads its character from the source
represented by the {\tt FILE *} that is its argument, instead of from
the standard input.  Here's code to print out the contents of the file
{\tt mydata.txt} onto the standard output:

\begin{flushleft}
\verb% #include <stdio.h>% \\*
\verb% #define FILENAME "mydata.txt"% \\*
\verb% % \\*
\verb% int main(void)% \\*
\verb% {% \\*
\verb%   FILE * the_file;% \\*
\verb%   int c;% \\*
\verb% % \\*
\verb%   the_file = fopen(FILENAME, "r");% \\*
\verb%   if (the_file == NULL) { % \\*
\verb!     printf("Couldn't open the file %s.\n", FILENAME);! \\*
\verb%     return 1; % \\*
\verb%   }% \\*
\verb% % \\*
\verb%   while ((c = getc(the_file)) != EOF) % \\*
\verb%     putchar(c);% \\*
\verb% % \\*
\verb%   return 0;% \\*
\verb% }% \\*
\end{flushleft}

Actually we left out a detail here:  {\tt getc} and {\tt getchar} return
{\tt EOF} for two reasons:  because there was no more data for them to
read, or because there was some kind of error in reading the data.  (For
example, the disk failed in the middle.)  We really should have checked
whether the {\tt getc} above was returning {\tt EOF} for end-of-file or
for an error.  We'll see how to fix this soon.

\subsection{Stream Versions of {\tt printf} and {\tt scanf}}

The function {\tt fprintf} is just like {\tt printf}, except it has an
extra argument:  The first argument to {\tt fprintf} is a stream to
write its output to.  The second argument is a format string just like
{\tt printf}'s first argument, and the remaining arguments are values to
fill into the conversions in the format string, just like {\tt printf}'s
remaining arguments.  So, for example,

\begin{flushleft}
\verb! fprintf(the_file, "The value of %s is %d.\n", "fooey", fooey); ! \\*
\end{flushleft}

\noindent is exactly like 

\begin{flushleft}
\verb! printf("The value of %s is %d.\n", "fooey", fooey); ! \\*
\end{flushleft}

\noindent except that the {\tt fprintf} writes the output to the file
represented by {\tt the\_file}, while the {\tt printf} writes it to the
standard output.

Similarly, there is an {\tt fscanf} function, which is just like {\tt
scanf}.  Similarly, {\tt getchar} has {\tt getc} and {\tt putchar} has
{\tt putc}.  

\subsection{Closing a File}

When you're done with a file, you must {\em close}\/ it.  This tells the
operating system that you are done using the file.  That way the
operating system knows that it can forget all the details about the file
that it was keeping track of for you, such as how much data you'd read
out of it.  

To close a stream, you call {\tt fclose}.  {\tt fclose}'s argument is
the stream you want to close.  Once a stream is closed, you can't read
from it or write to it any more.  {\tt fclose} returns 0 if it succeeds
and {\tt EOF} if it fails.  One reason {\tt fclose} might fail is that
you tried to close a stream that you never opened.

\section{Command-Line Arguments}

When you run a program from MS-DOS, you can give it arguments.  For
example, when you run the {\tt copy} command, you  give it arguments
that say what files you want copied and where you want them copied to.
These command-line arguments get passed in to {\tt main} when the
operating system calls {\tt main}.

If we were writing {\tt copy} in C, we'd need some way to examine the
command-line arguments so that we'd know which files to copy.

\subsection{{\tt main}'s Header}

There are exactly two legal ways to write {\tt main}'s header. 

\begin{flushleft}
\verb% int main(void)% \\*
\end{flushleft}

\noindent we already know about---it means we're going to ignore the
command-line arguments.  The other header {\tt main} can have is

\begin{flushleft}
\verb% int main(int argc, char **argv)% \\*
\end{flushleft}

When the program gets run, the operating system collects the arguments
together.  Each argument gets put into a {\tt NUL}-terminated array of
characters.

The operating system gets the address of the first element each of these
arrays.  These addresses have type \p2\Char.  It takes the addresses and
assembles them into another array, an \ao{pointer to \Char}.  This array
is called the {\em argument vector}\/.

The operating system then takes the address of the first element in the
argument vector.  The first element is a \p2\Char, and so the address of
the first element has type \p2{pointer to \Char}.  The operating system
arranges that this value, the address of the argument vector, gets
passed to {\tt main} as its second argument, which is conventionally
called {\tt argv} for `argument vector', even though it's really a
pointer to the argument vector itself.

{\tt main} has the usual problem: It's got {\tt argv}, a pointer to the
first element of an array, so it can find the elements of the array with
no problem.  But how does it know when to stop?

{\tt main}'s first argument is the {\em argument count}\/,
conventionally called {\tt argc}.  It's an \int.  It says how many
arguments there are and therefore how long the argument vector is.

\subsection{A Thousand Words about {\tt argv} and {\tt argc}}

Here's a picture:

\vspace{3in}

\subsection{{\tt argc} and {\tt argv} in Practice}

It takes a while to get comfortable enough with pointers to understand
what's going on with {\tt argv}.  In the meantime, just remember this:
{\tt argv[0]} is a string which contains the program's first argument.
{\tt argv[1]} is a string which contains the program's second argument,
and so on, up to {\tt argv[argc-1]}, which is a string which contains
the program's last argument.

\subsection{The {\tt echo} Program}

Here's a program called {\tt echo}, which prints out its command-line
arguments onto the standard output:

\begin{flushleft}
\verb% #include <stdio.h>% \\*
\verb% % \\*
\verb% int main(int argc, char **argv)% \\*
\verb% { % \\*
\verb%   int i;% \\*
\verb% % \\*
\verb%   for (i=0; i<argc; i++)% \\*
\verb!     printf("%s ", argv[i]); ! \\*
\verb% % \\*
\verb%   printf("\n");% \\*
\verb%   return 0;% \\*
\verb% }% \\*
\end{flushleft}

\subsection{The {\tt type} Program}

Here's a program which types out the contents of the files named on its
command line.  It's like the MS-DOS {\tt type} command.

\input type.tex

Some notes: We used {\tt continue}, which we haven't seen before.  When
the computer encounters a {\tt continue} statement, it immediately
starts the next iteration of the smallest loop it's in.  Inside of {\tt
while} or {\tt do{\rm--}while} loop, {\tt continue} skips the rest of
the loop body and jumps right to the test.  In a {\tt for} statement,
{\tt continue} is the same, except it evaluates the update expression
before skipping to the test.  We use it here because if we can't open a
file, we don't want to bother with the rest of the loop, which is for
reading a file; we want to go on and try the next file immediately.
{\tt continue} starts the next pass through the loop immediately,
without executing the rest of the loop body.

We use the variable {\tt num\_errors} to keep track of the number of
problems we've had with the files so far, and return it when we're done.
Note that this is consistent with the convention of returning 0 if your
program was successful and nonzero if it encountered errors.  

{\tt argv[0]} conventionally contains the name of the command
that's being run---in this case, {\tt type}.  If we entered the command 
\begin{flushleft}
\verb% type foo bar baz% \\*
\end{flushleft}

\noindent to type out the files {\tt foo}, {\tt bar}, and {\tt baz}, then {\tt
argv[0]} would be {\tt type}, and {\tt argv[1]} would be {\tt foo}.
That's why we start opening files with the one named in {\tt argv[1]}.

Note that we're always careful to close files with {\tt fclose} when
we're done.  most systems enforce a limit on the number of files you can
have open at once, so we need to recycle our streams when we're done
using them.  

\end{document}
