%
% CSE 110 Lecture Notes
%
% Entire contents are copyright 1992 by Mark-Jason Dominus.
% All rights reserved.  Unauthorized reproduction prohobited.
%

\documentstyle[fancyheadings]{article}

% \def\brac#1{{\tt <}#1{\tt >}}
\def\brac#1{$<$#1$>$}
\def\Int{{\tt int}}
\def\int{\brac{\Int}}
\def\int{\brac{\Int}}
\def\Shortint{{\tt short~int}}
\def\shortint{\brac{\Shortint}}
\def\Longint{{\tt long~int}}
\def\longint{\brac{\Longint}}
\def\Float{{\tt float}}
\def\float{\brac{\Float}}
\def\Double{{\tt double}}
\def\double{\brac{\Double}}
\def\Char{{\tt char}}
\def\chr{\brac{\Char}}
\def\Void{{\tt void}}
\def\void{\brac{\Void}}

\def\p2#1{\brac{pointer~to #1}}

\parskip 8pt

% \def\baselinestretch{2}

\title{Lecture 5}
\author{CSE 110}
\date{7 July 1992}

\pagestyle{fancy}
\lhead{CSE 110 Lecture Notes}
\chead{Mark--Jason Dominus}
\rhead{\thepage}
\lfoot{Copyright \copyright 1992 Mark-Jason Dominus}
\cfoot{}
\rfoot{All rights reserved.}



\begin{document}

\maketitle

\section{More About {\tt if}}

\subsection {{\tt else}}

Consider this code:

\verb% %                  \\*
\verb% if ( x > 0 ) %                  \\*
\verb%   printf("X has a positive value\n");%       \\*
\verb% if ( ! (x > 0) ) %                  \\*
\verb%   printf("X does not have a positive value\n");%       

Suppose the value of {\tt x} is 0. We know that exactly one of the {\tt
printf} statements will always be executed, and so we know that if it
isn't the first one, it must be the second.  So as smart humans, we
don't have to perform both tests.  The computer, on the other hand, is
dumber, and we have to tell it explicitly that the two {\tt if}
statements here represent an exclusive partitioning of all possible
situations.  We do that with an {\tt else} clause:

\verb% %                  \\*
\verb% if ( x > 0 ) %                  \\*
\verb%   printf("X has a positive value\n");%       \\*
\verb% else          %                  \\*
\verb%   printf("X does not have a positive value\n");%       

This tells the computer to compute the value of the expression {\tt x >
0}, and if that expression is true, then to print {\tt X has a positive
value}.  Otherwise, the computer prints {\tt X does not have a positive
value}. Exactly one of the {\tt printf} statements is ever executed each
time through this code.

When we want to select one of many conditions, we use a similar construction:

\verb% %                  \\*
\verb% if ( profits < 0  ) %                  \\*
\verb%   printf("The company is losing money\n");%      \\* 
\verb% else if ( profits == 0 )          %                  \\*
\verb%   printf("The company exactly broke even\n");%       \\*
\verb% else                      %                           \\*
\verb%   printf("The company turned a profit\n");%       

The computer evaluates the conditions, one at a time, until it finds one
that is true.  Then it executes the associated statement, and then skips
all the rest of the clauses.  If none of the conditions are true, the
computer executes the statement associated with the {\tt else} clause,
if there is one.

\subsection{Nested {if{\rm --}else} Statements}

Here's some code for printing out information about average test grades:
{\tt total} holds the sum of all the grades in the class, and {\tt
count} holds the number of students in the class.

\verb% % \\*
\verb% if (count != 0) % \\*
\verb%   if ( total/count < 80 ) % \\*
\verb%     printf("This class is doing badly\n");%        \\*
\verb%   else % \\*
\verb%     printf("This class is not doing too badly\n");%        \\*
\verb% else % \\*
\verb%   printf("There are no students in this class\n"); % 

An {\tt if} construction or an {\tt if{\rm --}else} construction is a
statement, and can go anywhere a simple or a compound statement can.
Here, we first check to see if {\tt count} is not 0; if it is 0, we
print {\tt There are no students in this class}.  Otherwise, we compute
whether or not the average grade is below 80 and print a message
depending on whether it is or not.  This second {\tt if{\rm --}else}
statement is said to be {\em nested within}\/ the first.

We indent the nested statement to make our meaning clearer.  

No suppose the final {\tt else} and the last {\tt printf} weren't in the
example above.  How does the compiler know that we meant what we did,
rather than

\verb% % \\*
\verb% if (count != 0) % \\*
\verb%   if ( total/count < 80 ) % \\*
\verb%     printf("This class is doing badly\n");%        \\*
\verb% else % \\*
\verb%   printf("This class is not doing too badly\n");%        \\*

\noindent, which would print {\tt This class is not doing too badly}
whenever {\tt count} was equal to 0.  

If you said the compiler knows the difference because of the
indentation, you're mistaken:  the compiler ignores all white space
completely.  The indentation is only there for the benefit of human
readers.  

The C language resolves this ambiguaity by fiat:  The rule is that if
there's more than one {\tt if} that an {\tt else} could go with, it goes
with the nearest one.  So the {\tt else} above matches with the second
{\tt if}, not the first.  If for some reason you really wanted the
meaning suggested by the second indentation, you could write this:

\verb% % \\*
\verb% if (count != 0) { % \\*
\verb%   if ( total/count < 80 ) % \\*
\verb%     printf("This class is doing badly\n");%        \\*
\verb% } else % \\*
\verb%   printf("This class is not doing too badly\n");%        \\*

The curly braces here delimit the statement that follows the first {\tt
if}.  That statement doesn't include the {\tt else} clause, so the
compiler can't construe the {\tt else} as being under control of the
first {\tt if}---it must be parallel to it.

\section{Example Program: Solving Quadratic Equations}

Now we'll write a program which lets the user enter a quadratic
equation.  If the roots of the equation are real, the computer will find
them and print them out, and otherwise the computer will print a message
saying that the roots are complex.  We did this in class, so here's the
code:

\subsection{The Program}

\input quad.tex

\subsection{Notes About this Program}

The program's small enough that we didn't bother with any functions other
than {\tt main}.  

The declaration

\verb% double discriminant; %

\noindent is analogous to the {\tt int foo; } declarations we've seen,
but it allocates enough space for a variable of type \double\ instead of
one of type \int.  A \double\ is a double-precision floating-point
number and can represent a fraction or a number much larger than an
\int\ can.  We'll need that for this program.

We are going to need the {\tt sqrt()} function, which takes one \double\
argument and returns the \double\ which is its square root.  Someone
else wrote the {\tt sqrt()} function and put it in the library for us,
so we don't have to supply it.  But here's a problem: How does the
compiler know that {\tt sqrt()} takes a \double\ argument and returns a
\double\ result?  It needs to know, or else it won't be able to pass the
argument or interpret the return value properly.  The answer is that we
must tell it.  We {\tt\#include} the file {\tt <math.h>}, which contains
a line that tells the compiler what types the arguments and return value
of {\tt sqrt()} will have.  Such a line is called a {\em prototype}\/;
it looks just like a function header with no body:

\verb%  double sqrt(double x); %

This gives the compiler the information it needs to compile our calls to
{\tt sqrt()} corectly.  Otherwise it would assume that {\tt sqrt()}
returned an \int\ value, and all sorts of nastiness would ensue.  We
didn't have to write this, because the authors of {\tt sqrt()} put it in
{\tt <math.h>} for us, so all we have to do is {\tt\#include} that file.

The {\tt\#include <stdio.h>} is there for a similar reason, to tell the
compiler what kinds of arguments and return types {\tt printf()} and
{\tt scanf()} have.

We haven't seen {\tt scanf()} yet, and we'll talk about it in detail
later.  But it's the reverse of {\tt printf()}: {\tt printf()} prints
data out, and {\tt scanf()} reads data in.  {\tt scanf()}'s first
argument, like {\tt printf()}`s first argument, describes the format of
input to be read, and the remaining arguments, rather than getting
printed out, describe where to store the input values entered by the
user.  The {\tt "\%lf"} says that the input will be in the form of a
{\bf l}ong {\bf f}loat, which is an obsolete synonym for a \double.  The
{\tt \&a} is more interesting and important, and so it gets a section to
itself.

\subsection {Pointers and the {\tt\&} Operator}

We need to tell {\tt scanf()} where to store the value that the user
enters.  Say we want to store that value in the variable {\tt a}.  Now,
{\tt scanf()} can't normally do that because it doesn't know where {\tt
a} is.  If we wrote something like

\verb+ scanf("\%lf", a); +

\noindent that would be no good at all, because {\tt scanf()} would get
the {\em value}\/ of {\tt a} and would never find out what it really
needs to know, which is where {\tt a} actually is.

The {\tt \&} operator, called the {\em address-of}\/ operator, says
where a variable is.  The expression {\tt \&a} yields a value of a
special type, called a \p2\Double, because it `points to' a
\double---that is, it says where a certain \double\ object can be found.
So, in short, we are telling {\tt scanf()} where to find the variable in
which we want it to store its result.  Later on, we'll see more of {\tt
\&}, and of its complementary operation, {\tt *}, which takes a pointer and
finds to which it points.  The main point here is that since we
explicitly told {\tt scanf()} where {\tt a} is stored, {\tt scanf()} can
change the value of {\tt a}.


The call 

\verb+ scanf("\%lf", &a); +

\noindent instructs {\tt scanf()} to read input from the keyboard,
parsing and interpreting it as a floating-point value, and to store the
value that that data represents into the space pointed to by {\tt \&a}.

Note the parentheses in the assignment expressions, to preserve clarity
and also to get the compiler to do what we want.

Note that we use {\tt \%f} to get {\tt printf()} to print out a \double\
value, whereas we used {\tt \%d} to print out an \int.


\end{document}
