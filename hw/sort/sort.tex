
\documentstyle{article}

\title{Assignment \#2}
\author{Due Date: Tuesday, 28 July 1992}
\date{15 July 1992}

\parskip 5pt

\begin{document}
\maketitle

    You will write a program which sorts the lines in its input into
alphabetical order.

\section{Requirements}

    Your program will read lines of text from the input source.  Each
line will be separated from the next by a newline character, which is
not part of the line.  Your program will print out the input lines, but
in alphabetical order.

    If the input contains many lines, you do not have to read and output
them all, but you must process at least the first 500 lines in the
input.  If a line is very long, you may truncate it, but you must save
at least the first 80 characters of each line.

    Your program will examine its command-line arguments to see where
the input comes from and where the output must go to: The first argument
will name an input file, the second will name an output file.  If one or
both arguments are missing, the input or output will default to stdin
and stdout, respectively.  Your program must recover gracefully from
incorrect arguments, printing appropriate messages if (for example) the
user supplies too many arguments or names a file that does not exist.

\section{What to Hand in}

You should hand in a disk with your source code and an executable, and
any special instructions to me.  You may choose to supply a log file
that demonstrates your program, if you like.  I will not accept paper
copies of solutions to this assignment.

\section{Assignment-Specific Points}

Seven points for printing a correct sorted list.  Two points each for
handling very long lists and for handling long lines correctly.  Two
points for allowing command-line arguments to specify input and output
sources.  One point for recovering gracefully from argument errors.

\section{About Sorting}
\label{sorting}

The {\bf requirements} section says that you have to print out the input
lines out in alphabetical order, and to do that you will need to {\em
sort}\/ them.  To {\em sort}\/ data means to place it in some order,
alphabetical in this case.  The sorting is the big deal on this project.
All the other stuff is just decoration.

There are a vast number of different algorithms for sorting data.  Some
are easier to understand than others, and some require language
features we don't have yet.   Since I need to show you an algorithm that
is easy to understand and that is also possible to implement with what
we know now, I have a choice of {\em straight insertion} or {\em bubble}
sorting.  

\subsection{The Straight Insertion Sort}

The straight insertion sort is familiar to you already:  It's how you
sort a deck of cards into order.  To sort a deck of cards, you keep two
piles:  One pile initially has all the cards in it; it's called the
{\em source pile}\/.  The other pile is initially empty; that's the
{destination pile}\/.  To sort the deck, hunt through the source pile
until you find the lowest card---say 2$\clubsuit$.   Take that
card out and put it in the destination pile.  Now find the lowest card
that's left in the source pile---that'll be 3$\clubsuit$.  Take it out
and put it on the destination pile.  Continue in this way until you've
taken the last card, the A$\spadesuit$, from the source pile and put it
on the destination pile.  Now the destination pile is sorted. 

To do this with elements of an array is not much different.  You have a
source array, which initially contains the data you want to sort, and a
destination array of the same size.  Find the smallest element in the
source array.  Move it to the first empty space in the destination
array; this involves copying the element from the source to the
destination, and also obliterating it in the source so that you don't
find it again next time.  Repeat this until all the elements are moved
from the source to the destination; at this point all the source
elements have been obliterated, but copies are in the destination array
in sorted order.

\subsection{The Bubble Sort}

The bubble sort is often maligned and a little less familiar than the
straight insertion sort, but it's really no worse, and it has the
advantage that it doesn't require an extra destination array.

Lay out a dime, a nickel, a quarter, a penny and a bean, in that order,
on the table, and follow along with the description.  We'll sort these
things into order by value, so that the bean, which is worthless, is on
the left, and the quarter is on the right.

Compare the leftmost two coins, the first and second ones.  If they're
in the correct order, do nothing.  Otherwise, swap their positions.
Then compare the second and third coins and switch them if they're out
of order.  Continue doing this until you compare the last two coins,
which should be the quarter and the bean.

Each time we compared two coins, the greater one wound up on the right
and we used it in the next comparison. Somewhere along the line we
compared something with the quarter, and the quarter appeared in all the
comparisons after that; it kept `bubbling' over to the right.  The
quarter is now in the rightmost position of the line of coins, which is
where it should be.

Now repeat the process.  The dime will bubble right to the correct
position. Repeat it again, and the nickel will move to the correct
position.  Each time we run through the coins, at least one coin will
bubble into the correct position.  Therefore, if there are $n$ coins, we
needn't repeat the bubbling process more than $n-1$ times. 

\subsection{Stupid-Sort}

To stupid-sort a pack of cards, throw the cards down the stairs and then
gather them up again.  Examine the pack to see if the cards are in
order.  If they are, stop.  Otherwise, repeat the process. 

Stupid-sort is not a reasonable sorting algorithm, because you have to
throw the cards down the stairs about $8\cdot10^{67}$ 
% 80,658,175,170,943,878,571,660,636,856,403,766,975,289,
%505,440,883,277,824,000,000,000,000 
times before it works.

\subsection{What Sort to Use}

You can use any reasonable sorting algorithm you like, including the
bubble sort or the straight insertion sort.  You may not use
stupid-sort.  If you want to use some algorithm not discussed here, it
would probably be a good idea to discuss it with me first, but you don't
have to.

\end{document}


